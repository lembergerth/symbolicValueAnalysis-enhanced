\section{New options/optimizations}
\label{sec:newOptions}

We extended strengthening of the \valueAnalysisCPA\ by the \constraintsCPA\ to simplify symbolic expressions, as long as they are independent.
If none of the symbolic identifiers occurring in a symbolic expression are part of any constraint of the constraints state and no other program variable's assignment,
its value can be any number, independently of all other variable assignments.
In conclusion, any such expression can be replaced with a single symbolic identifier without losing any information.
Such independence can be checked easily by traversing through all constraints' operands.
By replacing potentially complex expressions by a simple single symbolic identifier, the occurrence of complex formulas that have to be solved in SAT checks are reduced to single variables.
 
%\subsubsection{Different merge operators}
By using configuration option \configOption{cpa.constraints.mergeType = SEP} or \configOption{JOIN}, either $\cpaMerge^{sep}$, as used in \cite{Lemberger2015}, or the new merge operator $\cpaMerge$ as defined in Section \ref{sec:newMerge} can be used in the \constraintsCPA.

%\subsubsection{Different less-or-equal operators}
%\paragraph{Subset operator}
%\paragraph{Aliased subset operator}
%\paragraph{Implication operator}
For choosing the less-or-equal operator to use with the \constraintsCPA, the property \configOption{cpa.constraints.lessOrEqualType}
with possible values \configOption{SUBSET}, \configOption{ALIASED\_SUBSET} and \configOption{IMPLICATION} exists.
Each less-or-equal operator behaves as described in Section \ref{sec:leqOperators}.
Keep in mind that \configOption{ALIASED\_SUBSET} can result in wrong behaviour.
%\subsubsection{Location/Frequency of SAT checks}
%\paragraph{Check after every assume edge}
%\paragraph{Check at target location}
