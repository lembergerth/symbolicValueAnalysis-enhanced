\section{Different Less-or-equal Operators}
%\subsubsection{Subset operator}
\label{sec:leqOperators}
The less-or-equal operator is the operator executed the most often during analyses as $\cpaStop^{sep}$ uses it once for every state in the reached set, at every iteration of the CPA algorithm.
In addition, it is responsible for determining whether a new state is already covered and analysis can be stopped at this point.
Although the implementation framework \cpaChecker\ only performs a termination check for reached states at the same location, 
its speed and precision can make a great difference for the performance of our analysis.

\paragraph*{Aliasing operator}
The  less-or-equal operators we used for \symbolicValueAnalysisCPA\ and \constraintsCPA\ in \cite{Lemberger2015} using an \aliasFunc\ function try to be more precise than a simple subset check.
Unfortunately, they can result in false behaviour because of their independent behaviour.
Consider the two pairs of value state and constraint state
$e = (v, a)$ with $v = \{x \rightarrow s1, y \rightarrow s2\}$, $a = \{s1 > 0\})$ and
$e' = (v', a')$ with $v' = \{x \rightarrow s2, y \rightarrow s1\}$, $a' = \{s1 > 0\})$.
When using the aliasing less-or-equal operators of the \symbolicValueAnalysisCPA\ and of the \constraintsCPA,
the \symbolicValueAnalysisCPA\ states
$v \lesserEqual v'$ for $\aliasFunc$ function $\aliasFunc(s1) = s2$, $\aliasFunc(s2) = s1$ and
the \constraintsCPA\ states
$a \lesserEqual a'$ for $\aliasFunc(s1) = s1$.
Because of this, $e \lesserEqual e'$, although the concrete states
$\llbracket e \rrbracket = \{ c \in C |\ c(x) > 0 \}$ and
$\llbracket e' \rrbracket = \{ c \in C |\ c(y) > 0 \}$
represented by $e$ and $e'$ are two different sets.
This violates the definition of the less-or-equal operator for abstract domains (Section~\ref{sec:abstractState}).
For this example, the less-or-equal operator of the \constraintsCPA\ actually behaves like the subset operator, since $\aliasFunc$ represents the identity.
This shows that the less-or-equal operator of the \symbolicValueAnalysisCPA\ cannot be used, regardless of the operator used by the \constraintsCPA.
Besides the default less-or-equal operator for the \constraintsCPA\ 
presented in Section~\ref{sec:constraintsCPA}, another operator might prove useful.

%\subsubsection{Implication operator}
\paragraph*{Implication operator}
Since a \constraintsCPA 's abstract state $a$ is interpreted as the conjunction of its constraints $\varphi_a$, it seems fit to use implication as the less-or-equal operator.
Remember that $\llbracket a \rrbracket = \{ c \in C |\ c \satisfies \varphi_a \}$.
If a formula $\varphi_a$ implies a formula $\varphi_{a'}$ and $c$ satisfies $\varphi_a$, then $c$ also satisfies $\varphi_{a'}$.
Because of this 
\[\llbracket a \rrbracket = \{ c \in C |\ c \satisfies \varphi_a \} \subseteq \{ c \in C |\ c \satisfies \varphi_{a'} \} = \llbracket a' \rrbracket \text{ if } \varphi_a \Rightarrow \varphi_{a'}.\]
The less-or-equal operator for the \constraintsCPA\ using implication is defined as $a \lesserEqualImpl a'$ if $\varphi_a \Rightarrow \varphi_{a'}$.
This operator has a higher precision than $\lesserEqualSub$ but requires SAT checks, which are definitely worse in performance than merely checking whether one set is the subset of another.

