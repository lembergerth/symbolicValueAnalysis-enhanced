\section{CEGAR for Symbolic Value Analysis + Constraints CPA}
For using the \symbolicExecutionCPA\ with CEGAR, we have to define a refinement procedure that returns a precision of type $\Pi_\symExCPA$ that fits the precision of the \symbolicExecutionCPA. We designed two such refinement procedures:
The first uses adjusted versions of the refinement and interpolation algorithms as used for abstract variable assignments,
with an adjusted strongest-post operator and a precision type that fits $\Pi_\symExCPA$.
The second refinement procedure extracts a precision for the \symbolicExecutionCPA\ from the precision created by the refinement of the \predicateCPA, which is based on interpolation in the domain of linear arithmetic.

\subsection{Refinement based on refinement for abstract variable assignments}
\label{sec:assignmentRefinement}
We adjust the refinement algorithm for abstract variable assignments (Alg. \ref{alg:refinementExplicit}).
The feasibility check of an error path $\sigma$ is performed by executing the \symbolicExecutionCPA\ with full precision for all program locations $l$.
If the error location of $\sigma$ is reached by the analysis, the path is feasible. It is infeasible, otherwise.

%\paragraph{New strongest post operator (VA-Transfer + ConstraintsPrec transfer + strengthening)}
We use a strongest-post operator that reflects the semantics of our \symbolicExecutionCPA\ by 
defining a composite operator $\strongestPostOpSym_{op} : V_\symValCPA \times 2^\goodconstraintsset \rightarrow V_\symValCPA \times 2^\goodconstraintsset$.
It is the composition of the transfer relations of the \symbolicValueAnalysisCPA\ and the \constraintsCPA, as well as the strengthen operator $\strengthenOp{\constrCPA}{\symValCPA}$ to create useful constraints states.
The result of $\strongestPostOpSym_{op}$ is contradicting if $\strengthenOp{\constrCPA}{\symValCPA}$ is not defined (that means that the conjunction of constraints are contradicting) or the transfer relation of the \symbolicValueAnalysisCPA\ produces a contradicting abstract variable assignment.
Formally:
\[ \strongestPostOpSym_{op} (v, a) = \begin{dcases}
(v', a'') & \parbox{.6\textwidth}{ if $v \gtransfer_\symValCPA v'$,  $a \gtransfer_\constrCPA a'$, $\strengthenOp{\constrCPA}{\symValCPA}(a', v')$ is defined with $\strengthenOp{\constrCPA}{\symValCPA}(a', v') = a''$ and $g = (\cdot, op, \cdot)$}\\
\bot & \text{ otherwise}
\end{dcases}\]
The contradiction $\bot$ represents the bottom element for both the \symbolicValueAnalysisCPA\ as well as for the \constraintsCPA.
Both the transfer relation of the \symbolicValueAnalysisCPA\ $\transfer_\symValCPA$ and the transfer relation of the \constraintsCPA\ $\transfer_\constrCPA$ always produce only one successor, so we can use them in our definition while keeping $\strongestPostOpSym_{op}$ unambiguous.
The performance of this strongest-post operator can be significantly worse than when only using abstract variable assignments since SAT checks have to be performed in the strengthen operation.
Because of this, the amount of calls of the strongest-post operator can make a notable difference in performance.

\begin{algorithm}[t]
\caption{$\interpolateSym(\prefix, \suffix)$, a modified version of Alg. \ref{alg:interpolateExplicit}}
\label{alg:interpolateSym}
\begin{algorithmic}[1]
\Input two constraint sequences $\prefix$ and $\suffix$, with $\prefix \logicAnd \suffix$ contradicting
\Output a constraint sequence $\Gamma$, which is an interpolant for $\prefix$ and $\suffix$
\Variables an abstract variable assignment $v$ and a constraints state $a$

\State $(v, a) \assign \strongestPostOpSym_\prefix(\varnothing)$

\ForAll{$p \in a$} \label{alg:interpolateSym:aInterpolationBegin}
	\If{$\strongestPostOpSym_\suffix (v, a \setminus \{ p \})$ is contradicting}
		\State $a \assign a \setminus \{ p \}$ \Comment $p$ not relevant, should not occur in interpolant
	\EndIf
\EndFor \label{alg:interpolateSym:aInterpolationEnd}
\ForAll{$x \in \defRange(v)$} \label{alg:interpolateSym:vInterpolationBegin}
	\If{$\strongestPostOpSym_\suffix(v_\restrictedTo{\defRange(v) \setminus \{x\}}, a)$ is contradicting}
		\State $v \assign v_\restrictedTo{\defRange(v) \setminus \{x\}}$ \Comment $x$ not relevant, should not occur in interpolant
	\EndIf
\EndFor\label{alg:interpolateSym:vInterpolationEnd}

\State $\Gamma \assign \langle \rangle$
\ForAll{$p \in a$} \label{alg:interpolateSym:itpAStart}
	\State $\Gamma \assign \Gamma \logicAnd \langle \assume(p) \rangle$
\EndFor \label{alg:interpolateSym:itpAEnd}
\ForAll{$x \in \defRange(v)$} \label{alg:interpolateSym:itpVStart}
	\State $\Gamma \assign \Gamma \logicAnd \langle x \assign v(x)\rangle$ \label{alg:interpolateSym:vBuildInterpolant}
\EndFor \label{alg:interpolateSym:itpVEnd}
\State %Empty state so return has new line and line number
\Return $\Gamma$
\end{algorithmic}
\end{algorithm}

Using this strongest-post operator for interpolation (Alg. \ref{alg:interpolateSym}) allows the computation of an interpolant for a prefix $\prefix$ and a suffix $\suffix$ at a specific program location based on the semantics of the \symbolicExecutionCPA.
Since we want to create an interpolant $\Gamma$ that contains all information necessary for proving that $\strongestPostOpSym_{\Gamma \logicAnd \suffix}$ is contradicting, not only abstract variable assignments but also constraints have to be considered.
First, we compute the strongest-post condition $(v, a)$ for the prefix $\prefix$ based on the initial state $\varnothing$ to get the complete prefix's strongest-post condition.
Similar to interpolation for abstract variable assignments, we then eliminate all constraints from $a$ that are not necessary for proving that $\suffix$ is contradicting.
Next, we remove all assignments from $v$ that are not required. This way we try to get the weakest interpolant possible.
We then build the interpolant from all left constraints in $a$ and all left assignments of $v$.
Contrary to Algorithm \ref{alg:interpolateSym}, we build the interpolant $\Gamma$ not by using $\assume$ operations for each $x \in \defRange(v)$, but assignment operations.
By using $\assume$ operations only for the constraints of $a$ we can easily distinguish between both domains.
It is not wrong to only use $\assume$ operations, but it would unnecessarily increase the precision for the \constraintsCPA\ and the constraints sets used in the inductive interpolations, as all assumptions are added here.
This could significantly decrease performance during interpolation due to the for-loop over all constraints in Line \ref{alg:interpolateSym:itpAStart} and the strongest-post operators bad performance.
It might even be more effective to just use all constraints and not perform these additional strongest-post computations
by omitting Lines \ref{alg:interpolateSym:aInterpolationBegin} - \ref{alg:interpolateSym:aInterpolationEnd} of the algorithm.
Since constraints made of assume edges whose occurring program variables have an unknown value are discarded in strengthening of the \constraintsCPA, the precision of the \valueAnalysisCPA\ indirectly influences the constraints the \constraintsCPA\ tracks.
Additionally, it is also possible to eliminate variable assignments first, and constraints second.
We will examine all three possibilities later in detail (Section~\ref{sec:eval:itpTechniques}). 

\begin{algorithm}[t]
\caption{$\refineSymbolic{\sigma}$, a modified version of Alg. \ref{alg:refinementExplicit}.}
\label{alg:refineSym1}
\begin{algorithmic}[1]
\Input infeasible error path $\sigma = \langle (op_1, l_1), ..., (op_n, l_n) \rangle$
\Output precision $(\pi_\symValCPA, \pi_\constrCPA) \in \Pi_\symExCPA$
\Variables interpolating constraint sequence $\Gamma$
\State $\Gamma \assign \langle \rangle$
\State $(\pi_\symValCPA(l), \pi_\constrCPA(l)) \assign (\varnothing, \varnothing)$ for all program locations $l$
\For{$i \assign 1$ to $n - 1$}
	\State $\suffix \assign \langle op_{i+1}, ..., op_n \rangle$
	\State $\Gamma \assign \interpolateSym(\Gamma \logicAnd \langle op_i \rangle, \suffix)$ \Comment inductive interpolation \label{alg:refinementExplicit:interpolation}
	\State $(\pi_\symValCPA(l_i), \pi_\constrCPA(l_i)) \assign \extractPrecisionSym{\Gamma}$
\EndFor\\
\Return $(\pi_\symValCPA, \pi_\constrCPA)$
\end{algorithmic}
\end{algorithm}

After interpolation is done, a precision of type $\Pi_\symExCPA$ must be extracted from this interpolant so that future executions of the CPA algorithm with the \symbolicExecutionCPA\ can prove the examined error path as infeasible.
To get a precision that consists of the two individual precisions of \symbolicValueAnalysisCPA\ and \constraintsCPA\ from an interpolant $\Gamma$, we define the $\extractPrecisionFunc$ function as the composition of two new functions, each of which extracts the precision for one of these CPAs based on $\Gamma$:
\[\extractPrecisionSym{\Gamma} = (\extractPrecisionSymVal{\Gamma}, \extractPrecisionConstr{\Gamma})\]
with
%\begin{align*}
\[\extractPrecisionSymVal{\Gamma} = \{ x |\ (x, z) \in v \text{, } z \neq \bot_\valueset \text{ and } \strongestPostOpSym_\Gamma (\varnothing) = (v, a) \}\]
and
\[\extractPrecisionConstr{\Gamma}  = a \text{ with } \strongestPostOpSym_\Gamma (\varnothing) = (v, a)\]
if the \constraintsCPA\ uses the precision $L \rightarrow 2^\goodconstraintsset$, or
\[\extractPrecisionConstrAlt{\Gamma} = \{ x |\ \exists p \in a : \text{p originates from an expression with $x$} \}\]
if the \constraintsCPA\ uses the precision $L \rightarrow 2^X$.
%\end{align*}
The \symbolicValueAnalysisCPA\ is based on the \valueAnalysisCPA, so it also works with the default refinement procedure for abstract variable assignments described in Section \ref{sec:cegarBasics}.
We simply use the existing $\extractPrecisionFunc$ for this CPA.
The precision of the \constraintsCPA\ is the set of all tracked constraints, so the constraints that result from applying the strongest-post operator to the interpolant provide the precision needed at the current location for proving the current error path as infeasible in future analysis.
The adjusted refinement procedure is shown in Algorithm \ref{alg:refineSym1}.

Since sliced prefix selection (Sec.~\ref{sec:assignmentRefinement}) only relies on a strongest-post operator and an existing refinement procedure, we can easily use it with our refinement procedure and $\strongestPostOpSym$.
Evaluation will show that prefix selection can boost the performance of symbolic execution with CEGAR significantly.

\subsection{Refinement based on refinement for \predicateCPA}
\label{sec:predicateRefinement}
Another possible way of refinement is to delegate the procedure to the refinement of the \predicateCPA\ and extract  a precision of type $\Pi_\symExCPA$ from the created predicate precision.
This might not always work as the \predicateCPA\ is able to handle more complex operations than the \symbolicExecutionCPA, for example non-deterministic arrays.
Even if an error path $\sigma$ is infeasible by using the \symbolicExecutionCPA\ with full precision, the interpolant (and resulting precision) computed by the \predicateCPA 's refinement could rely on unsupported operations.

Two other drawbacks exist due to the \predicateCPA's precision type:
In the current implementation of \CpaChecker, we are not able to create constraints out of the predicates of the precision created by the \predicateCPA's refinement, but can only extract program variables' names.
Due to this it is not possible to use the precision type $\Pi_\constrCPA = L \rightarrow 2^\goodconstraintsset$ as part of $\Pi_\symExCPA$, but only $\Pi_\symValCPA$ (see Section \ref{sec:constraintsCPA}).
Secondly, the precision type $\Pi_\predCPA = 2^P$ of the \predicateCPA\ is not location-based per definition. Because of this, we have to assign the same set of tracked program variables for each location.
This is not a problem in the implementation though, because a more detailed adjustment of the \predicateCPA\ is possible there.

Despite these problems this approach might still yield better performance then $\refineSymbolicFunc$.
Though both have to rely on SMT solvers, our own procedure performs one SAT check for every constraint, while the \predicateCPA's refinement utilizes a SMT solver to handle the complete interpolation process, which is presumably more performant due to its specialization.
Algorithm \ref{alg:refineSym2} shows the refinement procedure delegating to \predicateCPA's refinement.
After computing the precision $\pi'$ of the \predicateCPA, the variables used in the predicates of $\pi'$ are extracted and assigned as the precision $\pi_\symValCPA(l)$ for every location $l$. The precision $\pi_\symValCPA$ is then returned as precision for both \symbolicValueAnalysisCPA\ and \constraintsCPA.

Figure~\ref{fig:symExCegarComparison1} shows the benefit of using CEGAR with the \symbolicExecutionCPA.
The figure shows how analysis without CEGAR creates a lot of abstract states with information not necessary for proving that, increasing exponentially with the amount of assumptions in the program.
In contrast, analysis with CEGAR consists of two iterations: While the target state is reachable in the analysis with empty precision, it is infeasible in the analysis with refined precision.
Both runs themselves need far lesser abstract states than analysis without CEGAR, as only necessary information is tracked.

\begin{algorithm}[t!]
\caption{$\refineSymbolicPred{\sigma}$}
\label{alg:refineSym2}
\begin{algorithmic}[1]
\Input infeasible error path $\sigma = \langle (op_1, l_1), ..., (op_n, l_n) \rangle$
\Output precision $(\pi_\symValCPA, \pi_\constrCPA) \in \Pi_\symExCPA = \Pi_\symValCPA \times \Pi_\symValCPA$
\Variables predicate precision $\pi' \in \Pi_\predCPA = 2^P$
\State $(\pi_\symValCPA(l), \pi_\constrCPA(l)) \assign (\varnothing, \varnothing)$ for all program locations $l$
\State $\pi' = \refinePred{\gamma}$
\State $\pi_\symValCPA(l) \assign \extractPrecisionSymPred{\pi'}$ for all $l$
\State % empty state so return is in new line
\Return $(\pi_\symValCPA, \pi_\symValCPA)$
\end{algorithmic}
\end{algorithm}

\begin{figure}[t]
\begin{subfigure}{\linewidth}
\center
\begin{tikzpicture}[->,>=stealth, mynodes/.style={rectangle, draw, minimum height=0.2cm, minimum width=0.6cm}]

\node[mynodes] (root) {};
\node[mynodes] (l1) [below left = 0.4cm and 1.8cm of root] {};
\node[mynodes] (r1) [below right = 0.4cm and 1.8cm of root] {};
\node[mynodes] (l2) [below = 0.4cm of l1] {};
\node[mynodes] (r2) [below = 0.4cm of r1] {};
\node[mynodes] (ll3) [below left = 0.4cm and 0.6cm of l2] {};
\node[mynodes] (lr3) [below right = 0.4cm and 0.6cm of l2] {};
\node[mynodes] (rl3) [below left = 0.4cm and 0.6cm of r2] {};
\node[mynodes] (rr3) [below right = 0.4cm and 0.6cm of r2] {};
\node[mynodes] (ll4) [below = 0.4cm of ll3] {};
\node[mynodes] (lr4)[below = 0.4cm of lr3] {};
\node[mynodes] (rl4)[below = 0.4cm of rl3] {};
\node[mynodes] (rr4)[below = 0.4cm of rr3] {};
\node[mynodes] (lll5)[below left = 0.4cm and 0.1cm of ll4] {};
\node[mynodes] (llr5)[below right = 0.4cm and 0.1cm of ll4, very thick, draw=red, fill=gray] {};
\node[mynodes] (lrl5)[below left = 0.4cm and 0.1cm of lr4] {};
\node[mynodes] (lrr5)[below right = 0.4cm and 0.1cm of lr4, very thick, draw=red, fill=gray] {};
\node[mynodes] (rll5)[below left = 0.4cm and 0.1cm of rl4] {};
\node[mynodes] (rlr5)[below right = 0.4cm and 0.1cm of rl4, very thick, draw=red, fill=gray] {};
\node[mynodes] (rrl5)[below left = 0.4cm and 0.1cm of rr4] {}; % error state
\node[mynodes] (rrr5)[below right = 0.4cm and 0.1cm of rr4, very thick, draw=red, fill=gray] {};

\path
(root) edge (l1)
(root) edge (r1)
(l1) edge [draw, densely dotted] (l2)
(r1) edge [draw, densely dotted] (r2)
(l2) edge (ll3)
(l2) edge (lr3)
(r2) edge (rl3)
(r2) edge (rr3)
(ll3) edge [draw, densely dotted] (ll4)
(lr3) edge [draw, densely dotted] (lr4)
(rl3) edge [draw, densely dotted] (rl4)
(rr3) edge [draw, densely dotted] (rr4)
(ll4) edge [draw, densely dotted] (lll5)
(ll4) edge [draw, densely dotted] (llr5)
(lr4) edge [draw, densely dotted] (lrl5)
(lr4) edge [draw, densely dotted] (lrr5)
(rl4) edge [draw, densely dotted] (rll5)
(rl4) edge [draw, densely dotted] (rlr5)
(rr4) edge [draw, densely dotted] (rrl5)
(rr4) edge [draw, densely dotted] (rrr5)
;

%\draw let \p{l1}=(l1), \p{ll4}=(ll4) in
%[decorate, decoration={brace, amplitude=10pt, mirror}, -] (-4.65, \y{l1}) -- (-4.65, \y{ll4}) node [black, midway, xshift=-1.2cm] {\tiny $n$ \mbox{assumptions}};
\draw 
[decorate, decoration={brace, amplitude=10pt, mirror}, -, transform canvas={yshift=-0.2cm}] (lll5.south west) -- (rrr5.south east) node [black, midway, yshift=-0.5cm] {\scriptsize $n$ assumptions $\Rightarrow 2^n$ states};
\end{tikzpicture}
\caption{Analysis without CEGAR}
\label{fig:symExNoCegarEx1}
\end{subfigure}\\[5ex]
\begin{subfigure}{.48\textwidth}
\center
\begin{tikzpicture}[->,>=stealth, mynodes/.style={rectangle, draw, minimum height=0.2cm, minimum width=0.6cm,font=\scriptsize}]

\node[mynodes] (root){};
\node[mynodes] (l1) [below left = 0.4cm and 0.1cm of root] {};
\node[mynodes] (r1) [below right = 0.4cm and 0.1cm of root] {};
\node[mynodes] (c2) [below right = 0.4cm and 0.1cm of l1] {};
\node[mynodes] (l3) [below left = 0.4cm and 0.1cm of c2] {};
\node[mynodes] (r3) [below right = 0.4cm and 0.1cm of c2] {};
\node[mynodes] (c4) [below right = 0.4cm and 0.1cm of l3] {};
\node[mynodes] (l5) [below left = 0.4cm and 0.1cm of c4] {};
\node[mynodes] (r5) [below right = 0.4cm and 0.1cm of c4,  draw=red, very thick] {};

\path
(root) edge (l1)
(root) edge (r1)
(l1) edge [densely dotted] (c2)
(r1) edge [densely dotted] (c2)
(c2) edge (l3)
(c2) edge (r3)
(l3) edge [densely dotted] (c4)
(r3) edge [densely dotted] (c4)
(c4) edge (l5)
(c4) edge (r5)
;

\end{tikzpicture}
\caption{Analysis with CEGAR, first iteration: Empty precision}
\end{subfigure}%
\hfill
\begin{subfigure}{.48\textwidth}
\center
\begin{tikzpicture}[->,>=stealth, mynodes/.style={rectangle, draw, minimum height=0.2cm, minimum width=0.6cm, font=\scriptsize}]

\node[mynodes] (root){};
\node[mynodes] (l1) [below left = 0.4cm and 0.1cm of root] {};
\node[mynodes] (r1) [below right = 0.4cm and 0.1cm of root] {};
\node[mynodes] (c2) [below right = 0.4cm and 0.1cm of l1] {};
\node[mynodes] (l3) [below left = 0.4cm and 0.6cm of c2] {};
\node[mynodes] (r3) [below right = 0.4cm and 0.6cm of c2] {};
\node[mynodes] (l4) [below = 0.4cm of l3] {};
\node[mynodes] (r4) [below = 0.4cm of r3] {};
\node[mynodes] (ll5) [below left = 0.4cm and 0.1cm of l4] {};
\node[mynodes] (lr5) [below right = 0.4cm and 0.1cm of l4, fill = gray, draw=red, very thick] {};
\node[mynodes] (rl5) [below left = 0.4cm and 0.1cm of r4] {};
\node[mynodes] (rr5) [below right = 0.4cm and 0.1cm of r4, fill = gray, draw=red, very thick] {};

\path
(root) edge (l1)
(root) edge (r1)
(l1) edge [densely dotted] (c2)
(r1) edge [densely dotted] (c2)
(c2) edge (l3)
(c2) edge (r3)
(l3) edge [densely dotted] (l4)
(r3) edge [densely dotted] (r4)
(l4) edge (ll5)
(l4) edge (lr5)
(r4) edge (rl5)
(r4) edge (rr5)
;
\end{tikzpicture}
\caption{Analysis with CEGAR, second iteration: Refined precision}
\end{subfigure}
\caption{Symbolic execution analysis of a program with and without CEGAR.
The red rectangles are abstract states at an infeasible target location. If they are filled grey, the analysis sees them as infeasible.}
\label{fig:symExCegarComparison1}
\end{figure}
