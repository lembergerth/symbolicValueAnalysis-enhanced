\documentclass[abstracton,bibliography=totoc,version=last,fontsize=12pt,BCOR=5mm,footinclude=false,a4paper,final,ngerman]{scrreprt}
\usepackage[listings=false]{scrhack}
\usepackage[utf8]{inputenc}
\usepackage{wrapfig}
\usepackage{multicol}
\usepackage{dcolumn}
\newcolumntype{d}[1]{D{.}{.}{#1}}

%%%%%%%%%%%%%% USEFUL PACKAGES

% more flexible enumerate environments
\usepackage{enumerate}
\usepackage{multirow}
\usepackage{rotating}
% commands for changing line-height
\usepackage{setspace}


%%%%%%%%%%%%%% FONT

% better font-encoding and hyphenation for german text
\usepackage[T1]{fontenc}

% Palatino is a nice font
\usepackage{palatino}

% mathpazo provides good math support if you use Palatino
\usepackage{mathpazo}

% AMS Euler is also a nice math font together with Palatino
%\usepackage{eulervm}

% By default, Palatino is combined with Helvetica for headings.
% Optima would be perhaps nicer for headings, but
% you need to download and install it from
% http://www.ctan.org/tex-archive/fonts/urw/classico/
% If you have it, enable it with
%\renewcommand{\sfdefault}{uop}

% If you want a more classical font, use Latin Modern
%\usepackage{lmodern}


%%%%%%%%%%%%%% PAGE LAYOUT

% For the definition of custom page layouts and margins
% (use only if you have strict requirements)
% Disable \KOMAoptions{DIV=default} below!
%\usepackage{geometry}

% more space between lines
\onehalfspacing

% re-compute page layout for the current font and spacing
% disable if you use the geometry package
\KOMAoptions{DIV=default}


%%%%%%%%%%%%%% LANGUAGE

% support english and german language
\usepackage[english,ngerman]{babel}


%%%%%%%%%%%%%% MATH

% The basic math packages from AMS
\usepackage{amsmath,amssymb}

% some further symbols (e.g. \lbracket)
% for more see the "Comprehensive LaTeX symbol list" (symbols-a4.pdf)
\usepackage{stmaryrd}

% a package for writing algorithms in a nice way
%\usepackage[algo2e]{algorithm2e}
\usepackage{setspace}

% |x| (abs function)
\newcommand{\abs}[1]{\left| #1 \right|}

% shortcuts for the set of natural numbers, complex numbers etc.
% \NN* are the natural numbers without 0
\makeatletter
\newcommand{\NN}{\@ifstar{{\mathbb{N}\setminus\{0\}}}{{\mathbb{N}}}}
\makeatother
\newcommand{\BB}{\mathbb{B}}
\newcommand{\CC}{\mathbb{C}}
\newcommand{\DD}{\mathbb{D}}
\newcommand{\ZZ}{\mathbb{Z}}
\newcommand{\RR}{\mathbb{R}}
\DeclareMathOperator{\mergeop}{\mathsf{merge}}
\DeclareMathOperator{\reached}{\mathsf{reached}}
\DeclareMathOperator{\waitlist}{\mathsf{waitlist}}
\DeclareMathOperator{\mergejoin}{\mergeop^{join}}
\DeclareMathOperator{\mergesep}{\mergeop^{sep}}
\DeclareMathOperator{\stopop}{\mathsf{stop}}
\DeclareMathOperator{\stopsep}{\stopop^{sep}}
\DeclareMathOperator{\transabs}{\leadsto}
\DeclareMathOperator{\absop}{\mathsf{abs}}
\DeclareMathOperator{\abssbe}{\absop^{\mathsf{SBE}}}
\DeclareMathOperator{\abslbe}{\absop^{\mathsf{LBE}}}
\DeclareMathOperator{\sizeop}{\mathsf{size}}
\DeclareMathOperator{\SP}{\mathsf{SP}}
\DeclareMathOperator{\precop}{\mathsf{prec}}
\newcommand{\sem}[1]{\left\llbracket #1 \right\rrbracket}
\newcommand{\calE}{\mathcal{E}}
\newcommand{\preds}{\mathcal{P}}
\newcommand{\Ops}{Ops}
\newcommand{\op}{{op}}
\newcommand{\formulas}{\mathcal{F}}
\newcommand{\concrStates}{\mathcal{C}}
%\newcommand{\semilat}{\mathcal{V}}
\newcommand{\cartAbs}{\phi_\mathbb{C}^\pi}
\newcommand{\boolAbs}{\phi_\mathbb{B}^\pi}
\newcommand{\precision}{\Pi}

\newcommand{\CPAchecker}{\textsc{CPAchecker}}
\newcommand{\SLAM}{\textsc{Slam}}
\newcommand{\BLAST}{\textsc{Blast}}
\newcommand{\SatAbs}{\textsc{SatAbs}}
\newcommand{\CBMC}{\textsc{CBMC}}
\newcommand{\CIL}{\textsc{CIL}}
\newcommand{\CSIsat}{\textsc{CSIsat}}
\newcommand{\MathSAT}{\textsc{MathSAT}}

% nicer phi
%\renewcommand\phi\varphi % we use both \phi and \varphi, so do not do this

% If you want german numbers (comma as decimal separators),
% load this package so that no space is inserted after commas.
% You really should insert spaces in places where like (x, y) thou.
%\usepackage{icomma}


%%%%%%%%%%%%%% GRAPHICS

% basic package for including external graphics
\usepackage{graphicx}

% only use PDF files if no extension is specified,
% we strongly prefer vector images
%\DeclareGraphicsExtensions{.pdf}

% color support
\usepackage{color}

% for pictures in TeX
\usepackage{tikz}
\usetikzlibrary{shapes.geometric}
\usetikzlibrary{shadows}
\usepackage{pgfplots}

% for trees with tikz
% (get package from http://www.ctan.org/tex-archive/help/Catalogue/entries/tikz-qtree.html)
%\usepackage{tikz-qtree}

% for graphs with tikz
% (get package from http://altermundus.com/pages/tkz/graph/index.html)
\usepackage{tkz-graph}


%%%%%%%%%%%%%% TABLES

% for table rows with alignment of numbers at the decimal separator
\usepackage{dcolumn}

% cells that span several rows or columns
\usepackage{multicol}


%%%%%%%%%%%%%% CODE & ALGORITHMS

%\usepackage{program}
%\usepackage{algorithm2e}


% for writing algorithms in pseudo-code
%\usepackage[noend]{algorithmic} % we use algorithmicx

% float environment for algorithms
\usepackage{algorithm}

\newcommand\theHalgorithm{\Alph{algorithm}}

% for code with syntax highlighting
\usepackage{listings}
% a basic formatting style for C programs
\definecolor{javared}{rgb}{0.6,0,0} % for strings
\definecolor{javagreen}{rgb}{0.25,0.5,0.35} % comments
\definecolor{javapurple}{rgb}{0.5,0,0.35} % keywords
\definecolor{javadocblue}{rgb}{0.25,0.35,0.75} % javadoc
\lstdefinestyle{JAVA}{
    basicstyle=\ttfamily,
    keywordstyle=\color{javapurple}\bfseries,
    stringstyle=\color{javared},
    commentstyle=\color{javagreen},
    morecomment=[s][\color{javadocblue}]{/**}{*/},
    numberblanklines=false,
    flexiblecolumns=true,
    lineskip=1pt,
    numbers=left,
    numberstyle=\tiny,
    numberfirstline=true,
    firstnumber=1,
}

\lstdefinestyle{C}{
    basicstyle=\ttfamily,
    keywordstyle=\color{javapurple}\bfseries,
    stringstyle=\color{javared},
    commentstyle=\color{javagreen},
    morecomment=[s][\color{javadocblue}]{/**}{*/},
    numberblanklines=false,
    flexiblecolumns=true,
    lineskip=1pt,
    numbers=left,
    numberstyle=\tiny,
    numberfirstline=true,
    firstnumber=1,
}

% use our new style by default
\lstset{style=C}


\providecommand{\frontmatter}{}
\providecommand{\mainmatter}{}
\providecommand{\backmatter}{}

\usepackage{microtype}

\usepackage[overlay]{textpos}
\setlength\TPHorizModule{1cm}
\setlength\TPVertModule{1cm}


\usepackage[format=hang]{caption}

\usepackage{sidecap}
\sidecaptionvpos{figure}{t}
\captionsetup[SCfigure]{format=default}

\usepackage{subcaption} % has to be loaded after caption
\captionsetup[subfigure]{format=default}

\usepackage[above,below]{placeins}
\usepackage{hyperref}
\hypersetup{colorlinks=false, pdfborder=0 0 0, pdftitle=Efficient Symbolic Execution using compositional CEGAR, pdfauthor=Thomas Lemberger, pdfsubject=Bachelors Thesis in Internet Computing, pdfkeywords=,breaklinks}

\clubpenalty=10000
\widowpenalty=10000
\displaywidowpenalty=10000
%\hfuzz=0.2pt
\tolerance=400
%\emergencystretch=5pt
%\vbadness=2000

\renewcommand{\topfraction}{0.90}
\renewcommand{\textfraction}{0.1}

\addtocounter{tocdepth}{-1}


\deffootnote[1em]{1em}{1em}{\textsuperscript{\thefootnotemark}\ }


\bibliographystyle{my-alpha}
%\setcounter{tocdepth}{2}

\usetikzlibrary{positioning}
\usetikzlibrary{decorations.pathreplacing}
\usetikzlibrary{calc}
\usepackage{enumitem}
\usepackage{algorithmicx}
\usepackage{algpseudocode}

\usepackage[mathscr]{euscript}
\usepackage{mathtools}

%% Start of own commands
\algnewcommand\algorithmicinput{\textbf{Input:}}
\algnewcommand\Input{\item[\algorithmicinput]}
\algnewcommand\algorithmicoutput{\textbf{Output:}}
\algnewcommand\Output{\item[\algorithmicoutput]}
\algnewcommand\algorithmicvariables{\textbf{Variables:}}
\algnewcommand\Variables{\item[\algorithmicvariables]}
\algtext*{EndWhile}
\algtext*{EndIf}
\algtext*{EndFor}

\newcommand{\valueAnalysisCPA}{value analysis CPA}
\newcommand{\ValueAnalysisCPA}{Value analysis CPA}
\newcommand{\constantpropagationCPA}{constant propagation CPA}
\newcommand{\ConstantpropagationCPA}{Constant propagation CPA}
\newcommand{\predicateCPA}{predicate CPA}
\newcommand{\PredicateCPA}{Predicate CPA}
\newcommand{\symbolicExecutionCPA}{symbolic execution CPA}
\newcommand{\SymbolicExecutionCPA}{Symbolic execution CPA}
\newcommand{\symbolicValueAnalysisCPA}{symbolic value analysis CPA}
\newcommand{\SymbolicValueAnalysisCPA}{Symbolic value analysis CPA}
\newcommand{\constraintsCPA}{constraints CPA}
\newcommand{\ConstraintsCPA}{Constraints CPA}
\newcommand{\compositeCPA}{composite CPA}
\newcommand{\CompositeCPA}{Composite CPA}
\newcommand{\locationCPA}{location CPA}
\newcommand{\LocationCPA}{Location CPA}
\newcommand{\cpaChecker}{\CPAchecker}
\newcommand{\CpaChecker}{\CPAchecker}

\newcommand{\predCPA}{\mathbb{P}}
\newcommand{\valCPA}{\mathbb{C}}
\newcommand{\valCPAPlus}{\valCPA}
\newcommand{\locCPA}{\mathbb{L}}
\newcommand{\symValCPA}{\mathbb{C_{S}}}
\newcommand{\symExCPA}{\mathbb{S}}
\newcommand{\constrCPA}{\mathbb{A}}
\newcommand{\compCPA}{\mathcal{C}}

%% Commands specific to specification of CPAs
\newcommand{\cpa}{\mathbb{D}}
\newcommand{\cpaPlus}{\cpa}
\DeclareMathOperator{\cpaMerge}{\mergeop}
\DeclareMathOperator{\mergeAgree}{{\cpaMerge^{agree}}}
\DeclareMathOperator{\cpaStop}{\stopop}
\DeclareMathOperator{\cpaPrec}{\precop}
\DeclareMathOperator{\singletonPrec}{\widetilde{\cpaPrec}}
\DeclareMathOperator{\auxiliaryPrec}{\text{prec}}
\newcommand{\singletonPrecision}{{\widetilde{\pi}}}
\newcommand{\singletonPi}{{\widetilde{\Pi}}}
\newcommand{\reachedSet}{\texttt{reached}}
\newcommand{\targetRegion}{\sigma^{t}}
\newcommand{\waitlistSet}{\texttt{waitlist}}
\newcommand{\isTargetStateFunc}{\texttt{isTargetState}}
\newcommand{\isTargetState}[1]{\isTargetStateFunc(#1)}
\newcommand{\extractErrorPathFunc}{\texttt{extractErrorPath}}
\newcommand{\extractErrorPath}[1]{\extractErrorPathFunc(#1)}
\newcommand{\isFeasibleFunc}{\texttt{isFeasible}}
\newcommand{\isFeasible}[1]{\isFeasibleFunc(#1)}
\newcommand{\refineFunc}{\texttt{refine}}
\newcommand{\refine}[1]{\refineFunc(#1)}
\newcommand{\refineExplicitFunc}{\refineFunc}
\newcommand{\refineExplicit}[1]{\refineExplicitFunc(#1)}
\newcommand{\refinePredFunc}{\refineFunc_\predCPA}
\newcommand{\refinePred}[1]{\refinePredFunc(#1)}
\newcommand{\refineSymbolicFunc}{\refineFunc_\symExCPA}
\newcommand{\refineSymbolic}[1]{\refineSymbolicFunc(#1)}
\newcommand{\refineSymbolicPredFunc}{\refineFunc'_\symExCPA}
\newcommand{\refineSymbolicPred}[1]{\refineSymbolicPredFunc(#1)}
\DeclareMathOperator{\strongestPostOp}{\SP}
%\newcommand{\strongestPostOpBool}{\text{SP}^\booleanset} % We don't need this one
\DeclareMathOperator{\strongestPostOpExplicit}{\strongestPostOp}
\DeclareMathOperator{\strongestPostOpSym}{\strongestPostOp^\symExCPA}

\newcommand{\interpolate}{\texttt{interpolate}}
\newcommand{\interpolateExplicit}{\interpolate}
\newcommand{\interpolateSym}{\interpolate_\symExCPA}
\newcommand{\extractPrecisionFunc}{\texttt{extractPrecision}}
\newcommand{\extractPrecision}[1]{\texttt{extractPrecision}(#1)}
\newcommand{\extractPrecisionLoc}[1]{\extractPrecisionFunc_\locCPA(#1)}
\newcommand{\extractPrecisionSym}[1]{\extractPrecisionFunc_\symExCPA(#1)}
\newcommand{\extractPrecisionSymVal}[1]{\extractPrecisionFunc_\symValCPA(#1)}
\newcommand{\extractPrecisionConstr}[1]{\extractPrecisionFunc_\constrCPA(#1)}
\newcommand{\extractPrecisionConstrAlt}[1]{\extractPrecisionConstr{#1}}
\newcommand{\extractPrecisionSymPred}[1]{\extractPrecisionFunc'_\symExCPA(#1)}
\newcommand{\craigItp}{\psi}
\newcommand{\prefix}{{\gamma^{-}}}
\newcommand{\suffix}{{\gamma^{+}}}

\newcommand{\transfer}{\rightsquigarrow}
\newcommand{\gtransfer}{\overset{g}{\transfer}}
\newcommand{\strengthen}{\downarrow}
\newcommand{\strengthenOp}[2]{\strengthen_{{#1}, {#2}}}
\newcommand{\compare}{\preceq}

%\newcommand{\llbracket}{[\![} % Defined by stmaryrd
%\newcommand{\rrbracket}{]\!]} % Defined by stmaryrd
\newcommand{\concretization}{\llbracket \cdot \rrbracket}
\newcommand{\lesserEqual}{\sqsubseteq}
\newcommand{\lesserEqualSub}{{\lesserEqual_{sub}}}
\newcommand{\lesserEqualImpl}{{\lesserEqual_{impl}}}
\newcommand{\nlesserEqual}{\not\sqsubseteq}
\newcommand{\join}{\sqcup}
\newcommand{\bigjoin}{\bigsqcup}

\newcommand{\assign}{:=}
\newcommand{\logicAnd}{\wedge}
\newcommand{\biglogicAnd}{\bigwedge}
\newcommand{\logicOr}{\vee}
\newcommand{\biglogicOr}{\bigvee}

\newcommand{\satisfies}{\vDash}
% \newcommand{\superimpose}[2]{{\ooalign{$#1\@firstoftwo#2$\cr\hfil$#1\@secondoftwo#2$\hfil\cr}}}
% \newcommand{\partialFunc}{\mathrel{\mathpalette\superimpose{{\rightarrow}{\circ}}}}
\newcommand{\partialFunc}{\ooalign{\hfil $\circ$\hfil \cr $\to$\cr}}

\newcommand{\valueset}{\mathscr{Z}}
\newcommand{\integerset}{\ZZ}
\newcommand{\booleanset}{\BB}
\newcommand{\symvalset}{{\valueset_\symValCPA}}
\newcommand{\constraintsset}{\gamma}
\newcommand{\goodconstraintsset}{{\gamma^+}}

\newcommand{\semilattice}{\mathscr{E}}
\newcommand{\loclattice}{\mathscr{L}}
\newcommand{\vallattice}{\mathscr{V}}
\newcommand{\symvallattice}{\mathscr{E}}
\newcommand{\predicatelattice}{\mathscr{P}}
\newcommand{\constraintslattice}{\mathscr{A}}

% Commands for definition of predicate cpa
\newcommand{\strongestPost}{\texttt{post}}
\newcommand{\StrongestPost}{\texttt{Post}}
\newcommand{\weakestPre}{\texttt{pre}}
\newcommand{\WeakestPre}{\texttt{Pre}}

% Commands for definition of value analysis cpa
\newcommand{\valAssignment}{V}
\newcommand{\assume}{\texttt{assume}}
\newcommand{\using}[1]{{/#1}}
\newcommand{\restrictedTo}[1]{{|#1}}
\newcommand{\defRange}{\texttt{def}}

% Commands for definition of symbolic value analysis
\newcommand{\symValues}{S}
\newcommand{\symIds}{S_I}
\newcommand{\symExpressions}{S_E}
\newcommand{\symvalAssignment}{V_\symValCPA}
\newcommand{\aliasFunc}{\texttt{alias}}
\newcommand{\evalFunc}{\texttt{eval}}

\newcommand{\safe}{{$safe$}}
\newcommand{\unsafe}{{$unsafe$}}

% Commands for describing implementation
\newcommand{\configOption}[1]{\mbox{\texttt{#1}}}
\newcommand{\configParam}[1]{\mbox{\texttt{#1}}}
\newcommand{\objectName}[1]{\texttt{#1}}
\newcommand{\typeParam}[1]{\textbf{#1}}

% Commands for describing evaluation
\newcommand{\resultTrue}{TRUE}
\newcommand{\resultFalse}{FALSE}
\newcommand{\resultUnknown}{UNKNOWN}

\begin{document}
% language for the main part of the document
% (change to "ngerman" for a german document)
\selectlanguage{english}

\frontmatter

\titlehead{
\centering
\includegraphics[width=5cm]{unilogo}
}

\subject{Bachelor's Thesis\\in Computer Science}
\title{Efficient Symbolic Execution using compositional CEGAR}
\author{{\LARGE Thomas Lemberger}\\\mbox{}\\Supervisor:\\Prof. Dr. Dirk Beyer\\}
\publishers{\today}
\date{}

\maketitle

\begin{abstract}
\begin{addmargin}{1.5cm}

\end{addmargin}
\end{abstract}

\setcounter{tocdepth}{5}
\setcounter{secnumdepth}{5}
\tableofcontents
\listofalgorithms
\listoffigures
\listoftables

\mainmatter
\section{Definitions of newly introduced concepts}
\section{A different merge operator for \constraintsCPA}
\label{sec:newMerge}
For every operation $\assume(p)$ at a location $l$ that transfers the control flow to a location $l'$ there exists another operation $\assume(\neg p)$ at the same location transfering the control flow to a location $l'' \neq l'$.
In most programs it is probable that the two different program branches starting at $l'$ and $l''$ meet again, that means that for a later program location $l'''$ two abstract states $a, a'$ of the \constraintsCPA\ (in the following  called \emph{constraints states}) exist with $a$ containing $p$ and $a'$ containing $\neg p$.

If a constraint $p$ is part of an abstract state $a$, $p$ is true in all concrete states represented by $a$ (just like a predicate in an abstract state of the \predicateCPA\ \cite{Beyer2008}).
If for one program location $l$ two constraints states $a, a'$ exist with $p \in a$ and $\neg p \in a'$ and $a \setminus \{ p \} = a' \setminus \{ \neg p \}$,
then $a$ represents all concrete states for which $p \logicAnd a \setminus \{ p \}$ is true and $a'$ represents all concrete states for which $\neg p \logicAnd a \setminus \{ p \}$ is true.
At this point, the analysis will never be able to prove a program location as infeasible because of $p$ or $\neg p$.
If $a'$ reaches a program location and computes it as infeasible by using $p$, the abstract state $a$ will compute the same program location as feasible, if it reaches it.
Because of this, it seems legit to delete these obsolete constraints and only continue with one more abstract state instead of two more concrete ones by using the merge operator
\[ \cpaMerge (a, a', \pi) = \begin{dcases}
a' \setminus \neg Q & \text{ if } a \lesserEqual a' \setminus \neg Q \\
a' & \text{ otherwise}
\end{dcases} \]
with $\neg Q = \{\neg p |\ p \in a \logicAnd \neg p \in a' \}$ and $Q = \{ p |\ p \in a \logicAnd \neg p \in a' \}$.
It is not necessary that $a' \setminus \neg Q = a \setminus Q$.
If $a' \setminus \neg Q$ represents a super set of the concrete states represented by $a \setminus Q$, that is $a \setminus Q \lesserEqual a' \setminus \neg Q$, then the above condition is true, and $a \lesserEqual a \setminus Q$.

This condition is automatically checked by the $\mergeAgree$ operator, so we can simply use $\cpaMerge(a, a', \pi) = a' \setminus \neg Q$.

\section{Different Less-or-equal Operators}
%\subsubsection{Subset operator}
\label{sec:leqOperators}
The less-or-equal operator is the operator executed the most often during analyses as $\cpaStop^{sep}$ uses it once for every state in the reached set, at every iteration of the CPA algorithm.
In addition, it is responsible for determining whether a new state is already covered and analysis can be stopped at this point.
Although the implementation framework \cpaChecker\ only performs a termination check for reached states at the same location, 
its speed and precision can make a great difference for the performance of our analysis.

\paragraph*{Aliasing operator}
The  less-or-equal operators we used for \symbolicValueAnalysisCPA\ and \constraintsCPA\ in \cite{Lemberger2015} using an \aliasFunc\ function try to be more precise than a simple subset check.
Unfortunately, they can result in false behaviour because of their independent behaviour.
Consider the two pairs of value state and constraint state
$e = (v, a)$ with $v = \{x \rightarrow s1, y \rightarrow s2\}$, $a = \{s1 > 0\})$ and
$e' = (v', a')$ with $v' = \{x \rightarrow s2, y \rightarrow s1\}$, $a' = \{s1 > 0\})$.
When using the aliasing less-or-equal operators of the \symbolicValueAnalysisCPA\ and of the \constraintsCPA,
the \symbolicValueAnalysisCPA\ states
$v \lesserEqual v'$ for $\aliasFunc$ function $\aliasFunc(s1) = s2$, $\aliasFunc(s2) = s1$ and
the \constraintsCPA\ states
$a \lesserEqual a'$ for $\aliasFunc(s1) = s1$.
Because of this, $e \lesserEqual e'$, although the concrete states
$\llbracket e \rrbracket = \{ c \in C |\ c(x) > 0 \}$ and
$\llbracket e' \rrbracket = \{ c \in C |\ c(y) > 0 \}$
represented by $e$ and $e'$ are two different sets.
This violates the definition of the less-or-equal operator for abstract domains (Section~\ref{sec:abstractState}).
For this example, the less-or-equal operator of the \constraintsCPA\ actually behaves like the subset operator, since $\aliasFunc$ represents the identity.
This shows that the less-or-equal operator of the \symbolicValueAnalysisCPA\ cannot be used, regardless of the operator used by the \constraintsCPA.
Besides the default less-or-equal operator for the \constraintsCPA\ 
presented in Section~\ref{sec:constraintsCPA}, another operator might prove useful.

%\subsubsection{Implication operator}
\paragraph*{Implication operator}
Since a \constraintsCPA 's abstract state $a$ is interpreted as the conjunction of its constraints $\varphi_a$, it seems fit to use implication as the less-or-equal operator.
Remember that $\llbracket a \rrbracket = \{ c \in C |\ c \satisfies \varphi_a \}$.
If a formula $\varphi_a$ implies a formula $\varphi_{a'}$ and $c$ satisfies $\varphi_a$, then $c$ also satisfies $\varphi_{a'}$.
Because of this 
\[\llbracket a \rrbracket = \{ c \in C |\ c \satisfies \varphi_a \} \subseteq \{ c \in C |\ c \satisfies \varphi_{a'} \} = \llbracket a' \rrbracket \text{ if } \varphi_a \Rightarrow \varphi_{a'}.\]
The less-or-equal operator for the \constraintsCPA\ using implication is defined as $a \lesserEqualImpl a'$ if $\varphi_a \Rightarrow \varphi_{a'}$.
This operator has a higher precision than $\lesserEqualSub$ but requires SAT checks, which are definitely worse in performance than merely checking whether one set is the subset of another.


\subsection{Location/Frequency of SAT checks}
\subsubsection{After every assume}
\subsubsection{At every target location only}

\subsection{Basic CEGAR and its algorithm}
\subsubsection{CEGAR and interpolation in general}
Counterexample-guided abstraction refinement (CEGAR) \cite{Clarke2003} is a technique to find an abstraction that contains as few information as possible while retaining the possibility to prove or disprove a program's correctness.
This technique can greatly reduce the number of abstract states in a program's analysis and is considered ''the most general and flexible for handling the state explosion problem,''\cite{Clarke2003}\ the major problem we are facing with our \symbolicExecutionCPA.

The technique starts analysis with a coarse abstraction and refines it based on counterexamples. A counterexample is a witness of a property violation.\cite{Beyer2013}
If no error path is found by the analysis, it terminates and reports that no property violation exists.
If an error path is found, it is checked whether the path is feasible (i.e. a possible program execution) by repeating the analysis with full precision.
If the path is feasible, the analysis terminates and reports the found property violation.
If the error path is infeasible it was only found because the abstraction is too coarse. As a consequence, the abstraction is refined using the error path.
After this, the analysis starts again, using the new abstraction.

Since the problem of finding the coarsest possible refinement of an abstraction based on an error path is NP-hard, \cite{Clarke2003}\ good heuristics have to be used to find good refinements.
Interpolation \cite{Henzinger2004}\ is one such technique in a boolean context that is used for refinement of both the \predicateCPA\ and \valueAnalysisCPA.

\subsubsection{CEGAR and interpolation in the context of configurable software verification}
\label{sec:cegarBasics}
To apply CEGAR and interpolation to configurable software verification, a simple modification has to be made to the CPA algorithm.
Instead of passing it an initial state $e_0$ and an initial precision $\pi_0$, we use an initial reached set $R_0$ and initial waitlist $W_0$ (Alg. \ref{alg:cpaPlus}).
This way we can control at which point the analysis continues after a refinement was performed.

\begin{algorithm}[t]
\caption{$CPA(\cpaPlus, R_0, W_0)$, adapted from \cite{Beyer2013}}
\label{alg:cpaPlus}
\begin{algorithmic}[1]

\Input a CPA $\cpaPlus = (D, \Pi, \transfer, \cpaMerge, \cpaStop, \cpaPrec)$,
	    a set $R_0 \subseteq (E \times \Pi)$ of initial states with their precision and
	    a subset $W_0 \subseteq R_0$ of frontier abstract states with their precision,
	    with $E$ being the set of elements of $D$
\Output a set of abstract states reachable from $R_0$ with their precision and
	   a subset of frontier abstract states with their precision
\Variables \reachedSet\ and \waitlistSet , both subsets of $E \times \Pi$
\State $\reachedSet \assign R_0$
\State $\waitlistSet \assign W_0$
\While{$\waitlistSet \neq \varnothing$} \Comment from here on the same as before
\State ...
\EndWhile
\end{algorithmic}
\end{algorithm}

\begin{algorithm}[t]
\caption{$CEGAR(\cpaPlus, e_0, \pi_0)$, adapted from \cite{Beyer2013}}
\label{alg:cegar}
\begin{algorithmic}[1]
\Input a CPA $\cpaPlus = (D, \Pi, \transfer, \cpaMerge, \cpaStop, \cpaPrec)$ with dynamic precision adjustment,
	an initial abstract state $e_0 \in E$ with precision $\pi_0 \in \Pi$,
	with $E$ denoting the set of elements of the semi-lattice of $D$
\Output the verification result \safe\ or \unsafe
\Variables the sets \reachedSet\ and \waitlistSet\ of elements of $E \times \Pi$,
	      an error path $\sigma = \langle (op_1, l_1), ..., (op_n, l_n) \rangle$\\

\State $\reachedSet \assign \{ (e_0, \pi_0) \}$
\State $\waitlistSet \assign \{ e_0, \pi_0 \}$
\State $\pi \assign \pi_0$
\While{true}
	\State $(\reachedSet, \waitlistSet) \assign CPA(\cpaPlus, \reachedSet, \waitlistSet)$
	\If{$\waitlistSet = \varnothing$}
		\Return \safe
	\Else
		\State $\sigma \assign \extractErrorPath{\reachedSet}$ \label{alg:cegar:extraction}
		\If{\isFeasible{$\sigma$}} \Comment error path feasible: report bug \label{alg:cegar:feasibilityCheck}
			\State % empty state for new line after if
			\Return \unsafe 
		\Else \Comment error path infeasible: refine and restart from the beginning
			\State $\pi \assign \pi \cup \refine{\sigma}$
			\State $\reachedSet \assign (e_0, \pi)$
			\State $\waitlistSet \assign (e_0, \pi)$ \label{alg:cegar:end}
		\EndIf
	\EndIf
\EndWhile
\end{algorithmic}
\end{algorithm}

Now that the CPA algorithm is able to use precisions created in a refinement procedure, we use it as a part of our complete CEGAR algorithm.
Algorithm \ref{alg:cegar} uses a CPA using dynamic precision adjustment $\cpaPlus$,
an initial state $e_0$
and an initial precision $\pi_0$
to compute whether a property violation exists.

First, the $CPA$ algorithm is used to compute a set of reached abstract states ($\reachedSet$) and a subset of this set that contains all reached abstract states that have not been handled yet ($\waitlistSet$).
If $\waitlistSet$ is empty, the $CPA$ algorithm has handled all reachable states without encountering any target state.
If this is the case, no property violation was found and the algorithm can return \safe.
Otherwise, an error path is extracted from the reached set.
If the error path is reported as feasible, a property violation exists or the algorithm is not able to prove that none exists. It returns \unsafe.
If the error path is infeasible, the current precision is too abstract.
It is refined based on the infeasible error path by using $\refineFunc : \Sigma \rightarrow \Pi$ with $\Sigma$ being the set of all error paths, so that it can prove its infeasibility.
After this, the reached set and waitlist are reset to their initial values and the algorithm repeats analysis with the refined precision.
It is important to notice that the return type of $\refineFunc$ has to be equal to the precision type $\Pi$ used in $\cpaPlus$.
Because of this, CPAs are not exchangeable without changing refinement, too, in general.

%%%%%%%%%%%%%%%%%%%%%%%%%%
%%% Refinement in general
%%%%%%%%%%%%%%%%%%%%%%%%%%
For refinement, the priorly mentioned technique of interpolation is used to determine a location-specific precision that is strong enough for the CPA algorithm with precision adjustment to prove that a given error path is infeasible.
A boolean formula $\craigItp$ is a Craig interpolant \cite{Craig1957}\ for two boolean formulas $\prefix$ (called prefix) and $\suffix$ (called suffix), if the following three conditions are fulfilled:
\begin{enumerate}[label=\alph*)]
\item The prefix implies $\craigItp$, that is $\prefix \Rightarrow \craigItp$.
\item $\craigItp$ contradicts the suffix, that means $\craigItp \logicAnd \suffix$ is contradicting.
\item $\craigItp$ only contains variables occurring in \emph{both} prefix and suffix.
\end{enumerate}
It is proven that such an interpolant always exists in the domain of abstract variable assignments \cite{Beyer2013} as well as in the theory of linear arithmetics \cite{Craig1957}.

%\subsubsection{Refinement for the domain of abstract variable assignments}
Our work is strongly based on the refinement technique for abstract variable assignments.
The strongest-post operator $\strongestPostOp_{op}$ describes the semantics of an operation $op \in Ops$.
It is the analogy to the transfer relation in the domain of CPAs.
It maps a region of concrete states, implied by an abstract variable assignment, to the region of all concrete states that can be reached by executing $op$.
The semantics of a path $\sigma = \langle (l_1, op_1), ..., (l_n, op_n) \rangle$ is defined as the consecutive application of the strongest-post operator to its constraint sequence $\gamma_\sigma = \langle op_1, ..., op_n \rangle$:
$\strongestPostOp_{\gamma_\sigma}(v) = \strongestPostOp_{op_n}(\strongestPostOp_{op_{n-1}} (...\ \strongestPostOp_{op_1}(v) ... ))$.
We use strongest-post operators during interpolation and refinement to evaluate paths.

The strongest-post operator $\strongestPostOpExplicit_{op}$ is defined in the following way:
%\begin{enumerate}[label=\alph*)]
%\item
For an assignment operation $s \assign exp$, $\strongestPostOpExplicit_{s \assign exp}(v) = v_\restrictedTo{X \setminus \{ s \}} \logicAnd v_{s \assign exp}$ with $v_{s \assign exp} = \{ (s, exp_\using{v}) \}$ and $exp_\using{v}$ denoting the evaluation of $exp$ using the abstract variable assignment $v$, as defined in Section \ref{sec:valueAnalysis}.
%\item
For an assume operation $\assume(p)$, 
	$\strongestPostOpExplicit_{\assume(p)}(v) = v'$ with 
	\[ v'(x) = \begin{dcases}
		\bot & \text{ if } \exists y \in \defRange(v) : v(y) = \bot \text{ or } p_\using{v} \text{ is unsatisfiable}\\
		c & \text{ if $c$ is the only satisfying assignment of $p_\using{v}$ for $x$}\\
		v(x) & \text{ if none of the above and } x \in \defRange(v)
	\end{dcases}\]
	with $p_\using{v}$ as defined in Section \ref{sec:valueAnalysis}.
%\end{enumerate}

%\subsubsection{Interpolation for abstract variable assignments}
\begin{algorithm}[t]
\caption{$\interpolateExplicit(\prefix, \suffix)$, adapted from \cite{Beyer2013}}
\label{alg:interpolateExplicit}
\begin{algorithmic}[1]
\Input two constraint sequences $\prefix$ and $\suffix$, with $\prefix \logicAnd \suffix$ contradicting
\Output a constraint sequence $\Gamma$, which is an interpolant for $\prefix$ and $\suffix$
\Variables an abstract variable assignment $v$

\State $v \assign \strongestPostOpExplicit_\prefix(\varnothing)$
\ForAll{$x \in \defRange(v)$}
	\If{$\strongestPostOpExplicit_\suffix(v_\restrictedTo{\defRange(v) \setminus \{x\}})$ is contradicting}
		\State $v \assign v_\restrictedTo{\defRange(v) \setminus \{x\}}$ \Comment $x$ not relevant, should not occur in interpolant
	\EndIf
\EndFor
\State $\Gamma \assign \langle \rangle$
\ForAll{$x \in \defRange(v)$} \label{alg:interpolateExplicit:itpStart}
	\State $\Gamma \assign \Gamma \logicAnd \langle \assume(x = v(x))\rangle$
\EndFor\\ \label{alg:interpolateExplicit:itpFinish}
\Return $\Gamma$
\end{algorithmic}
\end{algorithm}

The algorithm for interpolation in the domain of abstract variable assignments is shown in Algorithm \ref{alg:interpolateExplicit}.
For a prefix $\prefix$ and a suffix $\prefix$, the abstract variable assignment $v$, that results from applying $\prefix$ to the initial abstract variable assignment $\varnothing$ is computed.
Next, for each variable assignment in $v$ it is checked whether the assignment is necessary to prove that $\suffix$ is contradicting.
If it is not, it can be removed from $v$.
After all variable assignments are checked, $v$ only contains variable assignments that are necessary to prove that $\suffix$ is contradicting.
From these, the interpolant is built (Lines \ref{alg:interpolateExplicit:itpStart} - \ref{alg:interpolateExplicit:itpFinish}).

\begin{algorithm}[t]
\caption{$\refineExplicit{\sigma}$, adapted from \cite{Beyer2015}}
\label{alg:refinementExplicit}
\begin{algorithmic}[1]
\Input infeasible error path $\sigma = \langle (op_1, l_1), ..., (op_n, l_n) \rangle$
\Output precision $\pi$
\Variables interpolating constraint sequence $\Gamma$
\State $\Gamma \assign \langle \rangle$
\State $\pi(l) \assign \varnothing$ for all program locations $l$
\For{$i \assign 1$ to $n - 1$}\label{alg:refinementExplicit:loopStart}
	\State $\suffix \assign \langle op_{i+1}, ..., op_n \rangle$
	\State $\Gamma \assign \interpolateExplicit(\Gamma \logicAnd \langle op_i \rangle, \suffix)$ \Comment inductive interpolation \label{alg:refinementExplicit:interpolation}
	\State $\pi(l_i) \assign \extractPrecision{\Gamma}$
\EndFor\\
\Return $\pi$
\end{algorithmic}
\end{algorithm}

The interpolants produced are used in the refinement of the precision (Alg. \ref{alg:refinementExplicit}).
We use a location-specific precision $\pi : L \rightarrow 2^X$ that returns for a program location $l \in L$ all program variables of $X$ which are relevant for the analysis at this location. This approach realizes the lazy abstraction technique \cite{Henzinger2002}.
The algorithm starts with an initial, empty interpolant $\Gamma$ and empty precision $\pi$ with $\pi(l) = \varnothing$ for all $l \in L$.
For each location $(l_i, op_i)$ on the error path, the suffix $\suffix$ of this location are set and the interpolant is computed inductively from the existing interpolant in conjunction with the current operation $op_i$ and the suffix (Line \ref{alg:refinementExplicit:interpolation}).
A precision for the current program location is then extracted from the interpolant.
One straightforward way to do this is by using all program variables with a valid assignment in the  abstract variable assignment resulting from the application of the strongest-post operator to our interpolant:
\[\extractPrecision{\Gamma} = \{ x |\ (x, z) \in \strongestPostOpExplicit_\Gamma (\varnothing ) \text{ and } z \neq \bot_\valueset \}.\]
It is not only sufficient, but also required to use $\Gamma \logicAnd \langle op_i \rangle$ instead of the full prefix $\prefix = \langle op_1, ..., op_1 \rangle$ for interpolation. The full prefix cannot be used as it has to be assured that the precision resulting from these consecutive interpolations proves the error path infeasible. All information necessary for proving the infeasibility of the remaining error path is present in the current interpolant and operation.

This refinement procedure can be used in CEGAR (Alg. \ref{alg:cegar}) in combination with a CPA with precision adjustment that expects these precision types, like the \valueAnalysisCPA\ in combination with refinement for abstract variable assignments.

%\subsubsection{Refinement for the domain of linear arithmetics}
Refinement in the domain of linear arithmetics, as used for the \predicateCPA, uses a standard approach to refinement based on lazy abstraction and Craig interpolation.
The task of interpolation is delegated to an off-the-shelf SMT solver.

%\subsubsection{\ValueAnalysisCPA\ with precision adjustment}
%The \valueAnalysisCPA\ with dynamic precision adjustment \cite{Beyer2013} \[\valCPAPlus = (D_\valCPA, \Pi_\valCPAPlus, \transfer_\valCPAPlus, \cpaMerge^{sep}, \cpaStop^{sep}, %\cpaPrec_\valCPAPlus)\] is a CPA that can be, and is, used with the refinement for abstract variable assignments as described above.
%It consists of:
%\begin{enumerate}[leftmargin=*, label=\arabic*.]
%\item The abstract domain $D_\valCPA$ as defined in Section \ref{sec:valueAnalysis}.
%\item The set of precisions $\Pi_\valCPAPlus = L \rightarrow 2^X$. A precision $\pi \in \Pi_\valCPAPlus$ specifies a subset of program variables of $X$ that are tracked.
%\item The transfer relation $\transfer_\valCPAPlus$ contains the transfer $v \transfer_\valCPAPlus (v', \pi)$ if $v \transfer_\valCPA v'$.
%\item The merge operator $\cpaMerge^{sep}$ that performs no merging.
%\item The termination check $\cpaStop^{sep}$ that checks every state individually.
%\item The precision adjustment $\cpaPrec_\valCPAPlus$. Given an abstract state $v$ and a precision $\pi$, all abstract assignments of variables that do not occur in $\pi$ are removed from %$v$. This is done by restricting the partial function: $\cpaPrec_\valCPAPlus(v, \pi) = (v_\restrictedTo{\pi}, \pi)$. The given precision is returned as it is.
%\end{enumerate}

In this chapter, we gave an overview of all theoretical concepts that are necessary to describe our own work. We introduced the concept of configurable software verification and configurable program analyses (CPAs), a very versatile approach to automated software verification. We introduced different CPAs we use in this work and CEGAR with precision refinement for both linear arithmetics and abstract variable assignments, which we will use when applying CEGAR to the \symbolicExecutionCPA.

%\subsection{Bounded loops with continuation after reached bound}
%\subsubsection{Idea}
%\subsubsection{Existing LoopstackCPA}
%\subsubsection{No continuation after reached bound: Existing AssumptionStorageCPA}
%\subsubsection{New: Jump out of loop at every possible exit and abstract information}


%%%%%%%%%%%%%%%%%%%%%%%%%%%%%%%%
\section{Definitions of newly introduced concepts}
\section{A different merge operator for \constraintsCPA}
\label{sec:newMerge}
For every operation $\assume(p)$ at a location $l$ that transfers the control flow to a location $l'$ there exists another operation $\assume(\neg p)$ at the same location transfering the control flow to a location $l'' \neq l'$.
In most programs it is probable that the two different program branches starting at $l'$ and $l''$ meet again, that means that for a later program location $l'''$ two abstract states $a, a'$ of the \constraintsCPA\ (in the following  called \emph{constraints states}) exist with $a$ containing $p$ and $a'$ containing $\neg p$.

If a constraint $p$ is part of an abstract state $a$, $p$ is true in all concrete states represented by $a$ (just like a predicate in an abstract state of the \predicateCPA\ \cite{Beyer2008}).
If for one program location $l$ two constraints states $a, a'$ exist with $p \in a$ and $\neg p \in a'$ and $a \setminus \{ p \} = a' \setminus \{ \neg p \}$,
then $a$ represents all concrete states for which $p \logicAnd a \setminus \{ p \}$ is true and $a'$ represents all concrete states for which $\neg p \logicAnd a \setminus \{ p \}$ is true.
At this point, the analysis will never be able to prove a program location as infeasible because of $p$ or $\neg p$.
If $a'$ reaches a program location and computes it as infeasible by using $p$, the abstract state $a$ will compute the same program location as feasible, if it reaches it.
Because of this, it seems legit to delete these obsolete constraints and only continue with one more abstract state instead of two more concrete ones by using the merge operator
\[ \cpaMerge (a, a', \pi) = \begin{dcases}
a' \setminus \neg Q & \text{ if } a \lesserEqual a' \setminus \neg Q \\
a' & \text{ otherwise}
\end{dcases} \]
with $\neg Q = \{\neg p |\ p \in a \logicAnd \neg p \in a' \}$ and $Q = \{ p |\ p \in a \logicAnd \neg p \in a' \}$.
It is not necessary that $a' \setminus \neg Q = a \setminus Q$.
If $a' \setminus \neg Q$ represents a super set of the concrete states represented by $a \setminus Q$, that is $a \setminus Q \lesserEqual a' \setminus \neg Q$, then the above condition is true, and $a \lesserEqual a \setminus Q$.

This condition is automatically checked by the $\mergeAgree$ operator, so we can simply use $\cpaMerge(a, a', \pi) = a' \setminus \neg Q$.

\section{Different Less-or-equal Operators}
%\subsubsection{Subset operator}
\label{sec:leqOperators}
The less-or-equal operator is the operator executed the most often during analyses as $\cpaStop^{sep}$ uses it once for every state in the reached set, at every iteration of the CPA algorithm.
In addition, it is responsible for determining whether a new state is already covered and analysis can be stopped at this point.
Although the implementation framework \cpaChecker\ only performs a termination check for reached states at the same location, 
its speed and precision can make a great difference for the performance of our analysis.

\paragraph*{Aliasing operator}
The  less-or-equal operators we used for \symbolicValueAnalysisCPA\ and \constraintsCPA\ in \cite{Lemberger2015} using an \aliasFunc\ function try to be more precise than a simple subset check.
Unfortunately, they can result in false behaviour because of their independent behaviour.
Consider the two pairs of value state and constraint state
$e = (v, a)$ with $v = \{x \rightarrow s1, y \rightarrow s2\}$, $a = \{s1 > 0\})$ and
$e' = (v', a')$ with $v' = \{x \rightarrow s2, y \rightarrow s1\}$, $a' = \{s1 > 0\})$.
When using the aliasing less-or-equal operators of the \symbolicValueAnalysisCPA\ and of the \constraintsCPA,
the \symbolicValueAnalysisCPA\ states
$v \lesserEqual v'$ for $\aliasFunc$ function $\aliasFunc(s1) = s2$, $\aliasFunc(s2) = s1$ and
the \constraintsCPA\ states
$a \lesserEqual a'$ for $\aliasFunc(s1) = s1$.
Because of this, $e \lesserEqual e'$, although the concrete states
$\llbracket e \rrbracket = \{ c \in C |\ c(x) > 0 \}$ and
$\llbracket e' \rrbracket = \{ c \in C |\ c(y) > 0 \}$
represented by $e$ and $e'$ are two different sets.
This violates the definition of the less-or-equal operator for abstract domains (Section~\ref{sec:abstractState}).
For this example, the less-or-equal operator of the \constraintsCPA\ actually behaves like the subset operator, since $\aliasFunc$ represents the identity.
This shows that the less-or-equal operator of the \symbolicValueAnalysisCPA\ cannot be used, regardless of the operator used by the \constraintsCPA.
Besides the default less-or-equal operator for the \constraintsCPA\ 
presented in Section~\ref{sec:constraintsCPA}, another operator might prove useful.

%\subsubsection{Implication operator}
\paragraph*{Implication operator}
Since a \constraintsCPA 's abstract state $a$ is interpreted as the conjunction of its constraints $\varphi_a$, it seems fit to use implication as the less-or-equal operator.
Remember that $\llbracket a \rrbracket = \{ c \in C |\ c \satisfies \varphi_a \}$.
If a formula $\varphi_a$ implies a formula $\varphi_{a'}$ and $c$ satisfies $\varphi_a$, then $c$ also satisfies $\varphi_{a'}$.
Because of this 
\[\llbracket a \rrbracket = \{ c \in C |\ c \satisfies \varphi_a \} \subseteq \{ c \in C |\ c \satisfies \varphi_{a'} \} = \llbracket a' \rrbracket \text{ if } \varphi_a \Rightarrow \varphi_{a'}.\]
The less-or-equal operator for the \constraintsCPA\ using implication is defined as $a \lesserEqualImpl a'$ if $\varphi_a \Rightarrow \varphi_{a'}$.
This operator has a higher precision than $\lesserEqualSub$ but requires SAT checks, which are definitely worse in performance than merely checking whether one set is the subset of another.


\subsection{Location/Frequency of SAT checks}
\subsubsection{After every assume}
\subsubsection{At every target location only}

\subsection{Basic CEGAR and its algorithm}
\subsubsection{CEGAR and interpolation in general}
Counterexample-guided abstraction refinement (CEGAR) \cite{Clarke2003} is a technique to find an abstraction that contains as few information as possible while retaining the possibility to prove or disprove a program's correctness.
This technique can greatly reduce the number of abstract states in a program's analysis and is considered ''the most general and flexible for handling the state explosion problem,''\cite{Clarke2003}\ the major problem we are facing with our \symbolicExecutionCPA.

The technique starts analysis with a coarse abstraction and refines it based on counterexamples. A counterexample is a witness of a property violation.\cite{Beyer2013}
If no error path is found by the analysis, it terminates and reports that no property violation exists.
If an error path is found, it is checked whether the path is feasible (i.e. a possible program execution) by repeating the analysis with full precision.
If the path is feasible, the analysis terminates and reports the found property violation.
If the error path is infeasible it was only found because the abstraction is too coarse. As a consequence, the abstraction is refined using the error path.
After this, the analysis starts again, using the new abstraction.

Since the problem of finding the coarsest possible refinement of an abstraction based on an error path is NP-hard, \cite{Clarke2003}\ good heuristics have to be used to find good refinements.
Interpolation \cite{Henzinger2004}\ is one such technique in a boolean context that is used for refinement of both the \predicateCPA\ and \valueAnalysisCPA.

\subsubsection{CEGAR and interpolation in the context of configurable software verification}
\label{sec:cegarBasics}
To apply CEGAR and interpolation to configurable software verification, a simple modification has to be made to the CPA algorithm.
Instead of passing it an initial state $e_0$ and an initial precision $\pi_0$, we use an initial reached set $R_0$ and initial waitlist $W_0$ (Alg. \ref{alg:cpaPlus}).
This way we can control at which point the analysis continues after a refinement was performed.

\begin{algorithm}[t]
\caption{$CPA(\cpaPlus, R_0, W_0)$, adapted from \cite{Beyer2013}}
\label{alg:cpaPlus}
\begin{algorithmic}[1]

\Input a CPA $\cpaPlus = (D, \Pi, \transfer, \cpaMerge, \cpaStop, \cpaPrec)$,
	    a set $R_0 \subseteq (E \times \Pi)$ of initial states with their precision and
	    a subset $W_0 \subseteq R_0$ of frontier abstract states with their precision,
	    with $E$ being the set of elements of $D$
\Output a set of abstract states reachable from $R_0$ with their precision and
	   a subset of frontier abstract states with their precision
\Variables \reachedSet\ and \waitlistSet , both subsets of $E \times \Pi$
\State $\reachedSet \assign R_0$
\State $\waitlistSet \assign W_0$
\While{$\waitlistSet \neq \varnothing$} \Comment from here on the same as before
\State ...
\EndWhile
\end{algorithmic}
\end{algorithm}

\begin{algorithm}[t]
\caption{$CEGAR(\cpaPlus, e_0, \pi_0)$, adapted from \cite{Beyer2013}}
\label{alg:cegar}
\begin{algorithmic}[1]
\Input a CPA $\cpaPlus = (D, \Pi, \transfer, \cpaMerge, \cpaStop, \cpaPrec)$ with dynamic precision adjustment,
	an initial abstract state $e_0 \in E$ with precision $\pi_0 \in \Pi$,
	with $E$ denoting the set of elements of the semi-lattice of $D$
\Output the verification result \safe\ or \unsafe
\Variables the sets \reachedSet\ and \waitlistSet\ of elements of $E \times \Pi$,
	      an error path $\sigma = \langle (op_1, l_1), ..., (op_n, l_n) \rangle$\\

\State $\reachedSet \assign \{ (e_0, \pi_0) \}$
\State $\waitlistSet \assign \{ e_0, \pi_0 \}$
\State $\pi \assign \pi_0$
\While{true}
	\State $(\reachedSet, \waitlistSet) \assign CPA(\cpaPlus, \reachedSet, \waitlistSet)$
	\If{$\waitlistSet = \varnothing$}
		\Return \safe
	\Else
		\State $\sigma \assign \extractErrorPath{\reachedSet}$ \label{alg:cegar:extraction}
		\If{\isFeasible{$\sigma$}} \Comment error path feasible: report bug \label{alg:cegar:feasibilityCheck}
			\State % empty state for new line after if
			\Return \unsafe 
		\Else \Comment error path infeasible: refine and restart from the beginning
			\State $\pi \assign \pi \cup \refine{\sigma}$
			\State $\reachedSet \assign (e_0, \pi)$
			\State $\waitlistSet \assign (e_0, \pi)$ \label{alg:cegar:end}
		\EndIf
	\EndIf
\EndWhile
\end{algorithmic}
\end{algorithm}

Now that the CPA algorithm is able to use precisions created in a refinement procedure, we use it as a part of our complete CEGAR algorithm.
Algorithm \ref{alg:cegar} uses a CPA using dynamic precision adjustment $\cpaPlus$,
an initial state $e_0$
and an initial precision $\pi_0$
to compute whether a property violation exists.

First, the $CPA$ algorithm is used to compute a set of reached abstract states ($\reachedSet$) and a subset of this set that contains all reached abstract states that have not been handled yet ($\waitlistSet$).
If $\waitlistSet$ is empty, the $CPA$ algorithm has handled all reachable states without encountering any target state.
If this is the case, no property violation was found and the algorithm can return \safe.
Otherwise, an error path is extracted from the reached set.
If the error path is reported as feasible, a property violation exists or the algorithm is not able to prove that none exists. It returns \unsafe.
If the error path is infeasible, the current precision is too abstract.
It is refined based on the infeasible error path by using $\refineFunc : \Sigma \rightarrow \Pi$ with $\Sigma$ being the set of all error paths, so that it can prove its infeasibility.
After this, the reached set and waitlist are reset to their initial values and the algorithm repeats analysis with the refined precision.
It is important to notice that the return type of $\refineFunc$ has to be equal to the precision type $\Pi$ used in $\cpaPlus$.
Because of this, CPAs are not exchangeable without changing refinement, too, in general.

%%%%%%%%%%%%%%%%%%%%%%%%%%
%%% Refinement in general
%%%%%%%%%%%%%%%%%%%%%%%%%%
For refinement, the priorly mentioned technique of interpolation is used to determine a location-specific precision that is strong enough for the CPA algorithm with precision adjustment to prove that a given error path is infeasible.
A boolean formula $\craigItp$ is a Craig interpolant \cite{Craig1957}\ for two boolean formulas $\prefix$ (called prefix) and $\suffix$ (called suffix), if the following three conditions are fulfilled:
\begin{enumerate}[label=\alph*)]
\item The prefix implies $\craigItp$, that is $\prefix \Rightarrow \craigItp$.
\item $\craigItp$ contradicts the suffix, that means $\craigItp \logicAnd \suffix$ is contradicting.
\item $\craigItp$ only contains variables occurring in \emph{both} prefix and suffix.
\end{enumerate}
It is proven that such an interpolant always exists in the domain of abstract variable assignments \cite{Beyer2013} as well as in the theory of linear arithmetics \cite{Craig1957}.

%\subsubsection{Refinement for the domain of abstract variable assignments}
Our work is strongly based on the refinement technique for abstract variable assignments.
The strongest-post operator $\strongestPostOp_{op}$ describes the semantics of an operation $op \in Ops$.
It is the analogy to the transfer relation in the domain of CPAs.
It maps a region of concrete states, implied by an abstract variable assignment, to the region of all concrete states that can be reached by executing $op$.
The semantics of a path $\sigma = \langle (l_1, op_1), ..., (l_n, op_n) \rangle$ is defined as the consecutive application of the strongest-post operator to its constraint sequence $\gamma_\sigma = \langle op_1, ..., op_n \rangle$:
$\strongestPostOp_{\gamma_\sigma}(v) = \strongestPostOp_{op_n}(\strongestPostOp_{op_{n-1}} (...\ \strongestPostOp_{op_1}(v) ... ))$.
We use strongest-post operators during interpolation and refinement to evaluate paths.

The strongest-post operator $\strongestPostOpExplicit_{op}$ is defined in the following way:
%\begin{enumerate}[label=\alph*)]
%\item
For an assignment operation $s \assign exp$, $\strongestPostOpExplicit_{s \assign exp}(v) = v_\restrictedTo{X \setminus \{ s \}} \logicAnd v_{s \assign exp}$ with $v_{s \assign exp} = \{ (s, exp_\using{v}) \}$ and $exp_\using{v}$ denoting the evaluation of $exp$ using the abstract variable assignment $v$, as defined in Section \ref{sec:valueAnalysis}.
%\item
For an assume operation $\assume(p)$, 
	$\strongestPostOpExplicit_{\assume(p)}(v) = v'$ with 
	\[ v'(x) = \begin{dcases}
		\bot & \text{ if } \exists y \in \defRange(v) : v(y) = \bot \text{ or } p_\using{v} \text{ is unsatisfiable}\\
		c & \text{ if $c$ is the only satisfying assignment of $p_\using{v}$ for $x$}\\
		v(x) & \text{ if none of the above and } x \in \defRange(v)
	\end{dcases}\]
	with $p_\using{v}$ as defined in Section \ref{sec:valueAnalysis}.
%\end{enumerate}

%\subsubsection{Interpolation for abstract variable assignments}
\begin{algorithm}[t]
\caption{$\interpolateExplicit(\prefix, \suffix)$, adapted from \cite{Beyer2013}}
\label{alg:interpolateExplicit}
\begin{algorithmic}[1]
\Input two constraint sequences $\prefix$ and $\suffix$, with $\prefix \logicAnd \suffix$ contradicting
\Output a constraint sequence $\Gamma$, which is an interpolant for $\prefix$ and $\suffix$
\Variables an abstract variable assignment $v$

\State $v \assign \strongestPostOpExplicit_\prefix(\varnothing)$
\ForAll{$x \in \defRange(v)$}
	\If{$\strongestPostOpExplicit_\suffix(v_\restrictedTo{\defRange(v) \setminus \{x\}})$ is contradicting}
		\State $v \assign v_\restrictedTo{\defRange(v) \setminus \{x\}}$ \Comment $x$ not relevant, should not occur in interpolant
	\EndIf
\EndFor
\State $\Gamma \assign \langle \rangle$
\ForAll{$x \in \defRange(v)$} \label{alg:interpolateExplicit:itpStart}
	\State $\Gamma \assign \Gamma \logicAnd \langle \assume(x = v(x))\rangle$
\EndFor\\ \label{alg:interpolateExplicit:itpFinish}
\Return $\Gamma$
\end{algorithmic}
\end{algorithm}

The algorithm for interpolation in the domain of abstract variable assignments is shown in Algorithm \ref{alg:interpolateExplicit}.
For a prefix $\prefix$ and a suffix $\prefix$, the abstract variable assignment $v$, that results from applying $\prefix$ to the initial abstract variable assignment $\varnothing$ is computed.
Next, for each variable assignment in $v$ it is checked whether the assignment is necessary to prove that $\suffix$ is contradicting.
If it is not, it can be removed from $v$.
After all variable assignments are checked, $v$ only contains variable assignments that are necessary to prove that $\suffix$ is contradicting.
From these, the interpolant is built (Lines \ref{alg:interpolateExplicit:itpStart} - \ref{alg:interpolateExplicit:itpFinish}).

\begin{algorithm}[t]
\caption{$\refineExplicit{\sigma}$, adapted from \cite{Beyer2015}}
\label{alg:refinementExplicit}
\begin{algorithmic}[1]
\Input infeasible error path $\sigma = \langle (op_1, l_1), ..., (op_n, l_n) \rangle$
\Output precision $\pi$
\Variables interpolating constraint sequence $\Gamma$
\State $\Gamma \assign \langle \rangle$
\State $\pi(l) \assign \varnothing$ for all program locations $l$
\For{$i \assign 1$ to $n - 1$}\label{alg:refinementExplicit:loopStart}
	\State $\suffix \assign \langle op_{i+1}, ..., op_n \rangle$
	\State $\Gamma \assign \interpolateExplicit(\Gamma \logicAnd \langle op_i \rangle, \suffix)$ \Comment inductive interpolation \label{alg:refinementExplicit:interpolation}
	\State $\pi(l_i) \assign \extractPrecision{\Gamma}$
\EndFor\\
\Return $\pi$
\end{algorithmic}
\end{algorithm}

The interpolants produced are used in the refinement of the precision (Alg. \ref{alg:refinementExplicit}).
We use a location-specific precision $\pi : L \rightarrow 2^X$ that returns for a program location $l \in L$ all program variables of $X$ which are relevant for the analysis at this location. This approach realizes the lazy abstraction technique \cite{Henzinger2002}.
The algorithm starts with an initial, empty interpolant $\Gamma$ and empty precision $\pi$ with $\pi(l) = \varnothing$ for all $l \in L$.
For each location $(l_i, op_i)$ on the error path, the suffix $\suffix$ of this location are set and the interpolant is computed inductively from the existing interpolant in conjunction with the current operation $op_i$ and the suffix (Line \ref{alg:refinementExplicit:interpolation}).
A precision for the current program location is then extracted from the interpolant.
One straightforward way to do this is by using all program variables with a valid assignment in the  abstract variable assignment resulting from the application of the strongest-post operator to our interpolant:
\[\extractPrecision{\Gamma} = \{ x |\ (x, z) \in \strongestPostOpExplicit_\Gamma (\varnothing ) \text{ and } z \neq \bot_\valueset \}.\]
It is not only sufficient, but also required to use $\Gamma \logicAnd \langle op_i \rangle$ instead of the full prefix $\prefix = \langle op_1, ..., op_1 \rangle$ for interpolation. The full prefix cannot be used as it has to be assured that the precision resulting from these consecutive interpolations proves the error path infeasible. All information necessary for proving the infeasibility of the remaining error path is present in the current interpolant and operation.

This refinement procedure can be used in CEGAR (Alg. \ref{alg:cegar}) in combination with a CPA with precision adjustment that expects these precision types, like the \valueAnalysisCPA\ in combination with refinement for abstract variable assignments.

%\subsubsection{Refinement for the domain of linear arithmetics}
Refinement in the domain of linear arithmetics, as used for the \predicateCPA, uses a standard approach to refinement based on lazy abstraction and Craig interpolation.
The task of interpolation is delegated to an off-the-shelf SMT solver.

%\subsubsection{\ValueAnalysisCPA\ with precision adjustment}
%The \valueAnalysisCPA\ with dynamic precision adjustment \cite{Beyer2013} \[\valCPAPlus = (D_\valCPA, \Pi_\valCPAPlus, \transfer_\valCPAPlus, \cpaMerge^{sep}, \cpaStop^{sep}, %\cpaPrec_\valCPAPlus)\] is a CPA that can be, and is, used with the refinement for abstract variable assignments as described above.
%It consists of:
%\begin{enumerate}[leftmargin=*, label=\arabic*.]
%\item The abstract domain $D_\valCPA$ as defined in Section \ref{sec:valueAnalysis}.
%\item The set of precisions $\Pi_\valCPAPlus = L \rightarrow 2^X$. A precision $\pi \in \Pi_\valCPAPlus$ specifies a subset of program variables of $X$ that are tracked.
%\item The transfer relation $\transfer_\valCPAPlus$ contains the transfer $v \transfer_\valCPAPlus (v', \pi)$ if $v \transfer_\valCPA v'$.
%\item The merge operator $\cpaMerge^{sep}$ that performs no merging.
%\item The termination check $\cpaStop^{sep}$ that checks every state individually.
%\item The precision adjustment $\cpaPrec_\valCPAPlus$. Given an abstract state $v$ and a precision $\pi$, all abstract assignments of variables that do not occur in $\pi$ are removed from %$v$. This is done by restricting the partial function: $\cpaPrec_\valCPAPlus(v, \pi) = (v_\restrictedTo{\pi}, \pi)$. The given precision is returned as it is.
%\end{enumerate}

In this chapter, we gave an overview of all theoretical concepts that are necessary to describe our own work. We introduced the concept of configurable software verification and configurable program analyses (CPAs), a very versatile approach to automated software verification. We introduced different CPAs we use in this work and CEGAR with precision refinement for both linear arithmetics and abstract variable assignments, which we will use when applying CEGAR to the \symbolicExecutionCPA.

%\subsection{Bounded loops with continuation after reached bound}
%\subsubsection{Idea}
%\subsubsection{Existing LoopstackCPA}
%\subsubsection{No continuation after reached bound: Existing AssumptionStorageCPA}
%\subsubsection{New: Jump out of loop at every possible exit and abstract information}


%%%%%%%%%%%%%%%%%%%%%%%%%%%%%%%%
\section{Definitions of newly introduced concepts}
\section{A different merge operator for \constraintsCPA}
\label{sec:newMerge}
For every operation $\assume(p)$ at a location $l$ that transfers the control flow to a location $l'$ there exists another operation $\assume(\neg p)$ at the same location transfering the control flow to a location $l'' \neq l'$.
In most programs it is probable that the two different program branches starting at $l'$ and $l''$ meet again, that means that for a later program location $l'''$ two abstract states $a, a'$ of the \constraintsCPA\ (in the following  called \emph{constraints states}) exist with $a$ containing $p$ and $a'$ containing $\neg p$.

If a constraint $p$ is part of an abstract state $a$, $p$ is true in all concrete states represented by $a$ (just like a predicate in an abstract state of the \predicateCPA\ \cite{Beyer2008}).
If for one program location $l$ two constraints states $a, a'$ exist with $p \in a$ and $\neg p \in a'$ and $a \setminus \{ p \} = a' \setminus \{ \neg p \}$,
then $a$ represents all concrete states for which $p \logicAnd a \setminus \{ p \}$ is true and $a'$ represents all concrete states for which $\neg p \logicAnd a \setminus \{ p \}$ is true.
At this point, the analysis will never be able to prove a program location as infeasible because of $p$ or $\neg p$.
If $a'$ reaches a program location and computes it as infeasible by using $p$, the abstract state $a$ will compute the same program location as feasible, if it reaches it.
Because of this, it seems legit to delete these obsolete constraints and only continue with one more abstract state instead of two more concrete ones by using the merge operator
\[ \cpaMerge (a, a', \pi) = \begin{dcases}
a' \setminus \neg Q & \text{ if } a \lesserEqual a' \setminus \neg Q \\
a' & \text{ otherwise}
\end{dcases} \]
with $\neg Q = \{\neg p |\ p \in a \logicAnd \neg p \in a' \}$ and $Q = \{ p |\ p \in a \logicAnd \neg p \in a' \}$.
It is not necessary that $a' \setminus \neg Q = a \setminus Q$.
If $a' \setminus \neg Q$ represents a super set of the concrete states represented by $a \setminus Q$, that is $a \setminus Q \lesserEqual a' \setminus \neg Q$, then the above condition is true, and $a \lesserEqual a \setminus Q$.

This condition is automatically checked by the $\mergeAgree$ operator, so we can simply use $\cpaMerge(a, a', \pi) = a' \setminus \neg Q$.

\section{Different Less-or-equal Operators}
%\subsubsection{Subset operator}
\label{sec:leqOperators}
The less-or-equal operator is the operator executed the most often during analyses as $\cpaStop^{sep}$ uses it once for every state in the reached set, at every iteration of the CPA algorithm.
In addition, it is responsible for determining whether a new state is already covered and analysis can be stopped at this point.
Although the implementation framework \cpaChecker\ only performs a termination check for reached states at the same location, 
its speed and precision can make a great difference for the performance of our analysis.

\paragraph*{Aliasing operator}
The  less-or-equal operators we used for \symbolicValueAnalysisCPA\ and \constraintsCPA\ in \cite{Lemberger2015} using an \aliasFunc\ function try to be more precise than a simple subset check.
Unfortunately, they can result in false behaviour because of their independent behaviour.
Consider the two pairs of value state and constraint state
$e = (v, a)$ with $v = \{x \rightarrow s1, y \rightarrow s2\}$, $a = \{s1 > 0\})$ and
$e' = (v', a')$ with $v' = \{x \rightarrow s2, y \rightarrow s1\}$, $a' = \{s1 > 0\})$.
When using the aliasing less-or-equal operators of the \symbolicValueAnalysisCPA\ and of the \constraintsCPA,
the \symbolicValueAnalysisCPA\ states
$v \lesserEqual v'$ for $\aliasFunc$ function $\aliasFunc(s1) = s2$, $\aliasFunc(s2) = s1$ and
the \constraintsCPA\ states
$a \lesserEqual a'$ for $\aliasFunc(s1) = s1$.
Because of this, $e \lesserEqual e'$, although the concrete states
$\llbracket e \rrbracket = \{ c \in C |\ c(x) > 0 \}$ and
$\llbracket e' \rrbracket = \{ c \in C |\ c(y) > 0 \}$
represented by $e$ and $e'$ are two different sets.
This violates the definition of the less-or-equal operator for abstract domains (Section~\ref{sec:abstractState}).
For this example, the less-or-equal operator of the \constraintsCPA\ actually behaves like the subset operator, since $\aliasFunc$ represents the identity.
This shows that the less-or-equal operator of the \symbolicValueAnalysisCPA\ cannot be used, regardless of the operator used by the \constraintsCPA.
Besides the default less-or-equal operator for the \constraintsCPA\ 
presented in Section~\ref{sec:constraintsCPA}, another operator might prove useful.

%\subsubsection{Implication operator}
\paragraph*{Implication operator}
Since a \constraintsCPA 's abstract state $a$ is interpreted as the conjunction of its constraints $\varphi_a$, it seems fit to use implication as the less-or-equal operator.
Remember that $\llbracket a \rrbracket = \{ c \in C |\ c \satisfies \varphi_a \}$.
If a formula $\varphi_a$ implies a formula $\varphi_{a'}$ and $c$ satisfies $\varphi_a$, then $c$ also satisfies $\varphi_{a'}$.
Because of this 
\[\llbracket a \rrbracket = \{ c \in C |\ c \satisfies \varphi_a \} \subseteq \{ c \in C |\ c \satisfies \varphi_{a'} \} = \llbracket a' \rrbracket \text{ if } \varphi_a \Rightarrow \varphi_{a'}.\]
The less-or-equal operator for the \constraintsCPA\ using implication is defined as $a \lesserEqualImpl a'$ if $\varphi_a \Rightarrow \varphi_{a'}$.
This operator has a higher precision than $\lesserEqualSub$ but requires SAT checks, which are definitely worse in performance than merely checking whether one set is the subset of another.


\subsection{Location/Frequency of SAT checks}
\subsubsection{After every assume}
\subsubsection{At every target location only}

\subsection{Basic CEGAR and its algorithm}
\subsubsection{CEGAR and interpolation in general}
Counterexample-guided abstraction refinement (CEGAR) \cite{Clarke2003} is a technique to find an abstraction that contains as few information as possible while retaining the possibility to prove or disprove a program's correctness.
This technique can greatly reduce the number of abstract states in a program's analysis and is considered ''the most general and flexible for handling the state explosion problem,''\cite{Clarke2003}\ the major problem we are facing with our \symbolicExecutionCPA.

The technique starts analysis with a coarse abstraction and refines it based on counterexamples. A counterexample is a witness of a property violation.\cite{Beyer2013}
If no error path is found by the analysis, it terminates and reports that no property violation exists.
If an error path is found, it is checked whether the path is feasible (i.e. a possible program execution) by repeating the analysis with full precision.
If the path is feasible, the analysis terminates and reports the found property violation.
If the error path is infeasible it was only found because the abstraction is too coarse. As a consequence, the abstraction is refined using the error path.
After this, the analysis starts again, using the new abstraction.

Since the problem of finding the coarsest possible refinement of an abstraction based on an error path is NP-hard, \cite{Clarke2003}\ good heuristics have to be used to find good refinements.
Interpolation \cite{Henzinger2004}\ is one such technique in a boolean context that is used for refinement of both the \predicateCPA\ and \valueAnalysisCPA.

\subsubsection{CEGAR and interpolation in the context of configurable software verification}
\label{sec:cegarBasics}
To apply CEGAR and interpolation to configurable software verification, a simple modification has to be made to the CPA algorithm.
Instead of passing it an initial state $e_0$ and an initial precision $\pi_0$, we use an initial reached set $R_0$ and initial waitlist $W_0$ (Alg. \ref{alg:cpaPlus}).
This way we can control at which point the analysis continues after a refinement was performed.

\begin{algorithm}[t]
\caption{$CPA(\cpaPlus, R_0, W_0)$, adapted from \cite{Beyer2013}}
\label{alg:cpaPlus}
\begin{algorithmic}[1]

\Input a CPA $\cpaPlus = (D, \Pi, \transfer, \cpaMerge, \cpaStop, \cpaPrec)$,
	    a set $R_0 \subseteq (E \times \Pi)$ of initial states with their precision and
	    a subset $W_0 \subseteq R_0$ of frontier abstract states with their precision,
	    with $E$ being the set of elements of $D$
\Output a set of abstract states reachable from $R_0$ with their precision and
	   a subset of frontier abstract states with their precision
\Variables \reachedSet\ and \waitlistSet , both subsets of $E \times \Pi$
\State $\reachedSet \assign R_0$
\State $\waitlistSet \assign W_0$
\While{$\waitlistSet \neq \varnothing$} \Comment from here on the same as before
\State ...
\EndWhile
\end{algorithmic}
\end{algorithm}

\begin{algorithm}[t]
\caption{$CEGAR(\cpaPlus, e_0, \pi_0)$, adapted from \cite{Beyer2013}}
\label{alg:cegar}
\begin{algorithmic}[1]
\Input a CPA $\cpaPlus = (D, \Pi, \transfer, \cpaMerge, \cpaStop, \cpaPrec)$ with dynamic precision adjustment,
	an initial abstract state $e_0 \in E$ with precision $\pi_0 \in \Pi$,
	with $E$ denoting the set of elements of the semi-lattice of $D$
\Output the verification result \safe\ or \unsafe
\Variables the sets \reachedSet\ and \waitlistSet\ of elements of $E \times \Pi$,
	      an error path $\sigma = \langle (op_1, l_1), ..., (op_n, l_n) \rangle$\\

\State $\reachedSet \assign \{ (e_0, \pi_0) \}$
\State $\waitlistSet \assign \{ e_0, \pi_0 \}$
\State $\pi \assign \pi_0$
\While{true}
	\State $(\reachedSet, \waitlistSet) \assign CPA(\cpaPlus, \reachedSet, \waitlistSet)$
	\If{$\waitlistSet = \varnothing$}
		\Return \safe
	\Else
		\State $\sigma \assign \extractErrorPath{\reachedSet}$ \label{alg:cegar:extraction}
		\If{\isFeasible{$\sigma$}} \Comment error path feasible: report bug \label{alg:cegar:feasibilityCheck}
			\State % empty state for new line after if
			\Return \unsafe 
		\Else \Comment error path infeasible: refine and restart from the beginning
			\State $\pi \assign \pi \cup \refine{\sigma}$
			\State $\reachedSet \assign (e_0, \pi)$
			\State $\waitlistSet \assign (e_0, \pi)$ \label{alg:cegar:end}
		\EndIf
	\EndIf
\EndWhile
\end{algorithmic}
\end{algorithm}

Now that the CPA algorithm is able to use precisions created in a refinement procedure, we use it as a part of our complete CEGAR algorithm.
Algorithm \ref{alg:cegar} uses a CPA using dynamic precision adjustment $\cpaPlus$,
an initial state $e_0$
and an initial precision $\pi_0$
to compute whether a property violation exists.

First, the $CPA$ algorithm is used to compute a set of reached abstract states ($\reachedSet$) and a subset of this set that contains all reached abstract states that have not been handled yet ($\waitlistSet$).
If $\waitlistSet$ is empty, the $CPA$ algorithm has handled all reachable states without encountering any target state.
If this is the case, no property violation was found and the algorithm can return \safe.
Otherwise, an error path is extracted from the reached set.
If the error path is reported as feasible, a property violation exists or the algorithm is not able to prove that none exists. It returns \unsafe.
If the error path is infeasible, the current precision is too abstract.
It is refined based on the infeasible error path by using $\refineFunc : \Sigma \rightarrow \Pi$ with $\Sigma$ being the set of all error paths, so that it can prove its infeasibility.
After this, the reached set and waitlist are reset to their initial values and the algorithm repeats analysis with the refined precision.
It is important to notice that the return type of $\refineFunc$ has to be equal to the precision type $\Pi$ used in $\cpaPlus$.
Because of this, CPAs are not exchangeable without changing refinement, too, in general.

%%%%%%%%%%%%%%%%%%%%%%%%%%
%%% Refinement in general
%%%%%%%%%%%%%%%%%%%%%%%%%%
For refinement, the priorly mentioned technique of interpolation is used to determine a location-specific precision that is strong enough for the CPA algorithm with precision adjustment to prove that a given error path is infeasible.
A boolean formula $\craigItp$ is a Craig interpolant \cite{Craig1957}\ for two boolean formulas $\prefix$ (called prefix) and $\suffix$ (called suffix), if the following three conditions are fulfilled:
\begin{enumerate}[label=\alph*)]
\item The prefix implies $\craigItp$, that is $\prefix \Rightarrow \craigItp$.
\item $\craigItp$ contradicts the suffix, that means $\craigItp \logicAnd \suffix$ is contradicting.
\item $\craigItp$ only contains variables occurring in \emph{both} prefix and suffix.
\end{enumerate}
It is proven that such an interpolant always exists in the domain of abstract variable assignments \cite{Beyer2013} as well as in the theory of linear arithmetics \cite{Craig1957}.

%\subsubsection{Refinement for the domain of abstract variable assignments}
Our work is strongly based on the refinement technique for abstract variable assignments.
The strongest-post operator $\strongestPostOp_{op}$ describes the semantics of an operation $op \in Ops$.
It is the analogy to the transfer relation in the domain of CPAs.
It maps a region of concrete states, implied by an abstract variable assignment, to the region of all concrete states that can be reached by executing $op$.
The semantics of a path $\sigma = \langle (l_1, op_1), ..., (l_n, op_n) \rangle$ is defined as the consecutive application of the strongest-post operator to its constraint sequence $\gamma_\sigma = \langle op_1, ..., op_n \rangle$:
$\strongestPostOp_{\gamma_\sigma}(v) = \strongestPostOp_{op_n}(\strongestPostOp_{op_{n-1}} (...\ \strongestPostOp_{op_1}(v) ... ))$.
We use strongest-post operators during interpolation and refinement to evaluate paths.

The strongest-post operator $\strongestPostOpExplicit_{op}$ is defined in the following way:
%\begin{enumerate}[label=\alph*)]
%\item
For an assignment operation $s \assign exp$, $\strongestPostOpExplicit_{s \assign exp}(v) = v_\restrictedTo{X \setminus \{ s \}} \logicAnd v_{s \assign exp}$ with $v_{s \assign exp} = \{ (s, exp_\using{v}) \}$ and $exp_\using{v}$ denoting the evaluation of $exp$ using the abstract variable assignment $v$, as defined in Section \ref{sec:valueAnalysis}.
%\item
For an assume operation $\assume(p)$, 
	$\strongestPostOpExplicit_{\assume(p)}(v) = v'$ with 
	\[ v'(x) = \begin{dcases}
		\bot & \text{ if } \exists y \in \defRange(v) : v(y) = \bot \text{ or } p_\using{v} \text{ is unsatisfiable}\\
		c & \text{ if $c$ is the only satisfying assignment of $p_\using{v}$ for $x$}\\
		v(x) & \text{ if none of the above and } x \in \defRange(v)
	\end{dcases}\]
	with $p_\using{v}$ as defined in Section \ref{sec:valueAnalysis}.
%\end{enumerate}

%\subsubsection{Interpolation for abstract variable assignments}
\begin{algorithm}[t]
\caption{$\interpolateExplicit(\prefix, \suffix)$, adapted from \cite{Beyer2013}}
\label{alg:interpolateExplicit}
\begin{algorithmic}[1]
\Input two constraint sequences $\prefix$ and $\suffix$, with $\prefix \logicAnd \suffix$ contradicting
\Output a constraint sequence $\Gamma$, which is an interpolant for $\prefix$ and $\suffix$
\Variables an abstract variable assignment $v$

\State $v \assign \strongestPostOpExplicit_\prefix(\varnothing)$
\ForAll{$x \in \defRange(v)$}
	\If{$\strongestPostOpExplicit_\suffix(v_\restrictedTo{\defRange(v) \setminus \{x\}})$ is contradicting}
		\State $v \assign v_\restrictedTo{\defRange(v) \setminus \{x\}}$ \Comment $x$ not relevant, should not occur in interpolant
	\EndIf
\EndFor
\State $\Gamma \assign \langle \rangle$
\ForAll{$x \in \defRange(v)$} \label{alg:interpolateExplicit:itpStart}
	\State $\Gamma \assign \Gamma \logicAnd \langle \assume(x = v(x))\rangle$
\EndFor\\ \label{alg:interpolateExplicit:itpFinish}
\Return $\Gamma$
\end{algorithmic}
\end{algorithm}

The algorithm for interpolation in the domain of abstract variable assignments is shown in Algorithm \ref{alg:interpolateExplicit}.
For a prefix $\prefix$ and a suffix $\prefix$, the abstract variable assignment $v$, that results from applying $\prefix$ to the initial abstract variable assignment $\varnothing$ is computed.
Next, for each variable assignment in $v$ it is checked whether the assignment is necessary to prove that $\suffix$ is contradicting.
If it is not, it can be removed from $v$.
After all variable assignments are checked, $v$ only contains variable assignments that are necessary to prove that $\suffix$ is contradicting.
From these, the interpolant is built (Lines \ref{alg:interpolateExplicit:itpStart} - \ref{alg:interpolateExplicit:itpFinish}).

\begin{algorithm}[t]
\caption{$\refineExplicit{\sigma}$, adapted from \cite{Beyer2015}}
\label{alg:refinementExplicit}
\begin{algorithmic}[1]
\Input infeasible error path $\sigma = \langle (op_1, l_1), ..., (op_n, l_n) \rangle$
\Output precision $\pi$
\Variables interpolating constraint sequence $\Gamma$
\State $\Gamma \assign \langle \rangle$
\State $\pi(l) \assign \varnothing$ for all program locations $l$
\For{$i \assign 1$ to $n - 1$}\label{alg:refinementExplicit:loopStart}
	\State $\suffix \assign \langle op_{i+1}, ..., op_n \rangle$
	\State $\Gamma \assign \interpolateExplicit(\Gamma \logicAnd \langle op_i \rangle, \suffix)$ \Comment inductive interpolation \label{alg:refinementExplicit:interpolation}
	\State $\pi(l_i) \assign \extractPrecision{\Gamma}$
\EndFor\\
\Return $\pi$
\end{algorithmic}
\end{algorithm}

The interpolants produced are used in the refinement of the precision (Alg. \ref{alg:refinementExplicit}).
We use a location-specific precision $\pi : L \rightarrow 2^X$ that returns for a program location $l \in L$ all program variables of $X$ which are relevant for the analysis at this location. This approach realizes the lazy abstraction technique \cite{Henzinger2002}.
The algorithm starts with an initial, empty interpolant $\Gamma$ and empty precision $\pi$ with $\pi(l) = \varnothing$ for all $l \in L$.
For each location $(l_i, op_i)$ on the error path, the suffix $\suffix$ of this location are set and the interpolant is computed inductively from the existing interpolant in conjunction with the current operation $op_i$ and the suffix (Line \ref{alg:refinementExplicit:interpolation}).
A precision for the current program location is then extracted from the interpolant.
One straightforward way to do this is by using all program variables with a valid assignment in the  abstract variable assignment resulting from the application of the strongest-post operator to our interpolant:
\[\extractPrecision{\Gamma} = \{ x |\ (x, z) \in \strongestPostOpExplicit_\Gamma (\varnothing ) \text{ and } z \neq \bot_\valueset \}.\]
It is not only sufficient, but also required to use $\Gamma \logicAnd \langle op_i \rangle$ instead of the full prefix $\prefix = \langle op_1, ..., op_1 \rangle$ for interpolation. The full prefix cannot be used as it has to be assured that the precision resulting from these consecutive interpolations proves the error path infeasible. All information necessary for proving the infeasibility of the remaining error path is present in the current interpolant and operation.

This refinement procedure can be used in CEGAR (Alg. \ref{alg:cegar}) in combination with a CPA with precision adjustment that expects these precision types, like the \valueAnalysisCPA\ in combination with refinement for abstract variable assignments.

%\subsubsection{Refinement for the domain of linear arithmetics}
Refinement in the domain of linear arithmetics, as used for the \predicateCPA, uses a standard approach to refinement based on lazy abstraction and Craig interpolation.
The task of interpolation is delegated to an off-the-shelf SMT solver.

%\subsubsection{\ValueAnalysisCPA\ with precision adjustment}
%The \valueAnalysisCPA\ with dynamic precision adjustment \cite{Beyer2013} \[\valCPAPlus = (D_\valCPA, \Pi_\valCPAPlus, \transfer_\valCPAPlus, \cpaMerge^{sep}, \cpaStop^{sep}, %\cpaPrec_\valCPAPlus)\] is a CPA that can be, and is, used with the refinement for abstract variable assignments as described above.
%It consists of:
%\begin{enumerate}[leftmargin=*, label=\arabic*.]
%\item The abstract domain $D_\valCPA$ as defined in Section \ref{sec:valueAnalysis}.
%\item The set of precisions $\Pi_\valCPAPlus = L \rightarrow 2^X$. A precision $\pi \in \Pi_\valCPAPlus$ specifies a subset of program variables of $X$ that are tracked.
%\item The transfer relation $\transfer_\valCPAPlus$ contains the transfer $v \transfer_\valCPAPlus (v', \pi)$ if $v \transfer_\valCPA v'$.
%\item The merge operator $\cpaMerge^{sep}$ that performs no merging.
%\item The termination check $\cpaStop^{sep}$ that checks every state individually.
%\item The precision adjustment $\cpaPrec_\valCPAPlus$. Given an abstract state $v$ and a precision $\pi$, all abstract assignments of variables that do not occur in $\pi$ are removed from %$v$. This is done by restricting the partial function: $\cpaPrec_\valCPAPlus(v, \pi) = (v_\restrictedTo{\pi}, \pi)$. The given precision is returned as it is.
%\end{enumerate}

In this chapter, we gave an overview of all theoretical concepts that are necessary to describe our own work. We introduced the concept of configurable software verification and configurable program analyses (CPAs), a very versatile approach to automated software verification. We introduced different CPAs we use in this work and CEGAR with precision refinement for both linear arithmetics and abstract variable assignments, which we will use when applying CEGAR to the \symbolicExecutionCPA.

%\subsection{Bounded loops with continuation after reached bound}
%\subsubsection{Idea}
%\subsubsection{Existing LoopstackCPA}
%\subsubsection{No continuation after reached bound: Existing AssumptionStorageCPA}
%\subsubsection{New: Jump out of loop at every possible exit and abstract information}


%%%%%%%%%%%%%%%%%%%%%%%%%%%%%%%%
\section{Definitions of newly introduced concepts}
\section{A different merge operator for \constraintsCPA}
\label{sec:newMerge}
For every operation $\assume(p)$ at a location $l$ that transfers the control flow to a location $l'$ there exists another operation $\assume(\neg p)$ at the same location transfering the control flow to a location $l'' \neq l'$.
In most programs it is probable that the two different program branches starting at $l'$ and $l''$ meet again, that means that for a later program location $l'''$ two abstract states $a, a'$ of the \constraintsCPA\ (in the following  called \emph{constraints states}) exist with $a$ containing $p$ and $a'$ containing $\neg p$.

If a constraint $p$ is part of an abstract state $a$, $p$ is true in all concrete states represented by $a$ (just like a predicate in an abstract state of the \predicateCPA\ \cite{Beyer2008}).
If for one program location $l$ two constraints states $a, a'$ exist with $p \in a$ and $\neg p \in a'$ and $a \setminus \{ p \} = a' \setminus \{ \neg p \}$,
then $a$ represents all concrete states for which $p \logicAnd a \setminus \{ p \}$ is true and $a'$ represents all concrete states for which $\neg p \logicAnd a \setminus \{ p \}$ is true.
At this point, the analysis will never be able to prove a program location as infeasible because of $p$ or $\neg p$.
If $a'$ reaches a program location and computes it as infeasible by using $p$, the abstract state $a$ will compute the same program location as feasible, if it reaches it.
Because of this, it seems legit to delete these obsolete constraints and only continue with one more abstract state instead of two more concrete ones by using the merge operator
\[ \cpaMerge (a, a', \pi) = \begin{dcases}
a' \setminus \neg Q & \text{ if } a \lesserEqual a' \setminus \neg Q \\
a' & \text{ otherwise}
\end{dcases} \]
with $\neg Q = \{\neg p |\ p \in a \logicAnd \neg p \in a' \}$ and $Q = \{ p |\ p \in a \logicAnd \neg p \in a' \}$.
It is not necessary that $a' \setminus \neg Q = a \setminus Q$.
If $a' \setminus \neg Q$ represents a super set of the concrete states represented by $a \setminus Q$, that is $a \setminus Q \lesserEqual a' \setminus \neg Q$, then the above condition is true, and $a \lesserEqual a \setminus Q$.

This condition is automatically checked by the $\mergeAgree$ operator, so we can simply use $\cpaMerge(a, a', \pi) = a' \setminus \neg Q$.

\section{Different Less-or-equal Operators}
%\subsubsection{Subset operator}
\label{sec:leqOperators}
The less-or-equal operator is the operator executed the most often during analyses as $\cpaStop^{sep}$ uses it once for every state in the reached set, at every iteration of the CPA algorithm.
In addition, it is responsible for determining whether a new state is already covered and analysis can be stopped at this point.
Although the implementation framework \cpaChecker\ only performs a termination check for reached states at the same location, 
its speed and precision can make a great difference for the performance of our analysis.

\paragraph*{Aliasing operator}
The  less-or-equal operators we used for \symbolicValueAnalysisCPA\ and \constraintsCPA\ in \cite{Lemberger2015} using an \aliasFunc\ function try to be more precise than a simple subset check.
Unfortunately, they can result in false behaviour because of their independent behaviour.
Consider the two pairs of value state and constraint state
$e = (v, a)$ with $v = \{x \rightarrow s1, y \rightarrow s2\}$, $a = \{s1 > 0\})$ and
$e' = (v', a')$ with $v' = \{x \rightarrow s2, y \rightarrow s1\}$, $a' = \{s1 > 0\})$.
When using the aliasing less-or-equal operators of the \symbolicValueAnalysisCPA\ and of the \constraintsCPA,
the \symbolicValueAnalysisCPA\ states
$v \lesserEqual v'$ for $\aliasFunc$ function $\aliasFunc(s1) = s2$, $\aliasFunc(s2) = s1$ and
the \constraintsCPA\ states
$a \lesserEqual a'$ for $\aliasFunc(s1) = s1$.
Because of this, $e \lesserEqual e'$, although the concrete states
$\llbracket e \rrbracket = \{ c \in C |\ c(x) > 0 \}$ and
$\llbracket e' \rrbracket = \{ c \in C |\ c(y) > 0 \}$
represented by $e$ and $e'$ are two different sets.
This violates the definition of the less-or-equal operator for abstract domains (Section~\ref{sec:abstractState}).
For this example, the less-or-equal operator of the \constraintsCPA\ actually behaves like the subset operator, since $\aliasFunc$ represents the identity.
This shows that the less-or-equal operator of the \symbolicValueAnalysisCPA\ cannot be used, regardless of the operator used by the \constraintsCPA.
Besides the default less-or-equal operator for the \constraintsCPA\ 
presented in Section~\ref{sec:constraintsCPA}, another operator might prove useful.

%\subsubsection{Implication operator}
\paragraph*{Implication operator}
Since a \constraintsCPA 's abstract state $a$ is interpreted as the conjunction of its constraints $\varphi_a$, it seems fit to use implication as the less-or-equal operator.
Remember that $\llbracket a \rrbracket = \{ c \in C |\ c \satisfies \varphi_a \}$.
If a formula $\varphi_a$ implies a formula $\varphi_{a'}$ and $c$ satisfies $\varphi_a$, then $c$ also satisfies $\varphi_{a'}$.
Because of this 
\[\llbracket a \rrbracket = \{ c \in C |\ c \satisfies \varphi_a \} \subseteq \{ c \in C |\ c \satisfies \varphi_{a'} \} = \llbracket a' \rrbracket \text{ if } \varphi_a \Rightarrow \varphi_{a'}.\]
The less-or-equal operator for the \constraintsCPA\ using implication is defined as $a \lesserEqualImpl a'$ if $\varphi_a \Rightarrow \varphi_{a'}$.
This operator has a higher precision than $\lesserEqualSub$ but requires SAT checks, which are definitely worse in performance than merely checking whether one set is the subset of another.


\subsection{Location/Frequency of SAT checks}
\subsubsection{After every assume}
\subsubsection{At every target location only}

\subsection{Basic CEGAR and its algorithm}
\subsubsection{CEGAR and interpolation in general}
Counterexample-guided abstraction refinement (CEGAR) \cite{Clarke2003} is a technique to find an abstraction that contains as few information as possible while retaining the possibility to prove or disprove a program's correctness.
This technique can greatly reduce the number of abstract states in a program's analysis and is considered ''the most general and flexible for handling the state explosion problem,''\cite{Clarke2003}\ the major problem we are facing with our \symbolicExecutionCPA.

The technique starts analysis with a coarse abstraction and refines it based on counterexamples. A counterexample is a witness of a property violation.\cite{Beyer2013}
If no error path is found by the analysis, it terminates and reports that no property violation exists.
If an error path is found, it is checked whether the path is feasible (i.e. a possible program execution) by repeating the analysis with full precision.
If the path is feasible, the analysis terminates and reports the found property violation.
If the error path is infeasible it was only found because the abstraction is too coarse. As a consequence, the abstraction is refined using the error path.
After this, the analysis starts again, using the new abstraction.

Since the problem of finding the coarsest possible refinement of an abstraction based on an error path is NP-hard, \cite{Clarke2003}\ good heuristics have to be used to find good refinements.
Interpolation \cite{Henzinger2004}\ is one such technique in a boolean context that is used for refinement of both the \predicateCPA\ and \valueAnalysisCPA.

\subsubsection{CEGAR and interpolation in the context of configurable software verification}
\label{sec:cegarBasics}
To apply CEGAR and interpolation to configurable software verification, a simple modification has to be made to the CPA algorithm.
Instead of passing it an initial state $e_0$ and an initial precision $\pi_0$, we use an initial reached set $R_0$ and initial waitlist $W_0$ (Alg. \ref{alg:cpaPlus}).
This way we can control at which point the analysis continues after a refinement was performed.

\begin{algorithm}[t]
\caption{$CPA(\cpaPlus, R_0, W_0)$, adapted from \cite{Beyer2013}}
\label{alg:cpaPlus}
\begin{algorithmic}[1]

\Input a CPA $\cpaPlus = (D, \Pi, \transfer, \cpaMerge, \cpaStop, \cpaPrec)$,
	    a set $R_0 \subseteq (E \times \Pi)$ of initial states with their precision and
	    a subset $W_0 \subseteq R_0$ of frontier abstract states with their precision,
	    with $E$ being the set of elements of $D$
\Output a set of abstract states reachable from $R_0$ with their precision and
	   a subset of frontier abstract states with their precision
\Variables \reachedSet\ and \waitlistSet , both subsets of $E \times \Pi$
\State $\reachedSet \assign R_0$
\State $\waitlistSet \assign W_0$
\While{$\waitlistSet \neq \varnothing$} \Comment from here on the same as before
\State ...
\EndWhile
\end{algorithmic}
\end{algorithm}

\begin{algorithm}[t]
\caption{$CEGAR(\cpaPlus, e_0, \pi_0)$, adapted from \cite{Beyer2013}}
\label{alg:cegar}
\begin{algorithmic}[1]
\Input a CPA $\cpaPlus = (D, \Pi, \transfer, \cpaMerge, \cpaStop, \cpaPrec)$ with dynamic precision adjustment,
	an initial abstract state $e_0 \in E$ with precision $\pi_0 \in \Pi$,
	with $E$ denoting the set of elements of the semi-lattice of $D$
\Output the verification result \safe\ or \unsafe
\Variables the sets \reachedSet\ and \waitlistSet\ of elements of $E \times \Pi$,
	      an error path $\sigma = \langle (op_1, l_1), ..., (op_n, l_n) \rangle$\\

\State $\reachedSet \assign \{ (e_0, \pi_0) \}$
\State $\waitlistSet \assign \{ e_0, \pi_0 \}$
\State $\pi \assign \pi_0$
\While{true}
	\State $(\reachedSet, \waitlistSet) \assign CPA(\cpaPlus, \reachedSet, \waitlistSet)$
	\If{$\waitlistSet = \varnothing$}
		\Return \safe
	\Else
		\State $\sigma \assign \extractErrorPath{\reachedSet}$ \label{alg:cegar:extraction}
		\If{\isFeasible{$\sigma$}} \Comment error path feasible: report bug \label{alg:cegar:feasibilityCheck}
			\State % empty state for new line after if
			\Return \unsafe 
		\Else \Comment error path infeasible: refine and restart from the beginning
			\State $\pi \assign \pi \cup \refine{\sigma}$
			\State $\reachedSet \assign (e_0, \pi)$
			\State $\waitlistSet \assign (e_0, \pi)$ \label{alg:cegar:end}
		\EndIf
	\EndIf
\EndWhile
\end{algorithmic}
\end{algorithm}

Now that the CPA algorithm is able to use precisions created in a refinement procedure, we use it as a part of our complete CEGAR algorithm.
Algorithm \ref{alg:cegar} uses a CPA using dynamic precision adjustment $\cpaPlus$,
an initial state $e_0$
and an initial precision $\pi_0$
to compute whether a property violation exists.

First, the $CPA$ algorithm is used to compute a set of reached abstract states ($\reachedSet$) and a subset of this set that contains all reached abstract states that have not been handled yet ($\waitlistSet$).
If $\waitlistSet$ is empty, the $CPA$ algorithm has handled all reachable states without encountering any target state.
If this is the case, no property violation was found and the algorithm can return \safe.
Otherwise, an error path is extracted from the reached set.
If the error path is reported as feasible, a property violation exists or the algorithm is not able to prove that none exists. It returns \unsafe.
If the error path is infeasible, the current precision is too abstract.
It is refined based on the infeasible error path by using $\refineFunc : \Sigma \rightarrow \Pi$ with $\Sigma$ being the set of all error paths, so that it can prove its infeasibility.
After this, the reached set and waitlist are reset to their initial values and the algorithm repeats analysis with the refined precision.
It is important to notice that the return type of $\refineFunc$ has to be equal to the precision type $\Pi$ used in $\cpaPlus$.
Because of this, CPAs are not exchangeable without changing refinement, too, in general.

%%%%%%%%%%%%%%%%%%%%%%%%%%
%%% Refinement in general
%%%%%%%%%%%%%%%%%%%%%%%%%%
For refinement, the priorly mentioned technique of interpolation is used to determine a location-specific precision that is strong enough for the CPA algorithm with precision adjustment to prove that a given error path is infeasible.
A boolean formula $\craigItp$ is a Craig interpolant \cite{Craig1957}\ for two boolean formulas $\prefix$ (called prefix) and $\suffix$ (called suffix), if the following three conditions are fulfilled:
\begin{enumerate}[label=\alph*)]
\item The prefix implies $\craigItp$, that is $\prefix \Rightarrow \craigItp$.
\item $\craigItp$ contradicts the suffix, that means $\craigItp \logicAnd \suffix$ is contradicting.
\item $\craigItp$ only contains variables occurring in \emph{both} prefix and suffix.
\end{enumerate}
It is proven that such an interpolant always exists in the domain of abstract variable assignments \cite{Beyer2013} as well as in the theory of linear arithmetics \cite{Craig1957}.

%\subsubsection{Refinement for the domain of abstract variable assignments}
Our work is strongly based on the refinement technique for abstract variable assignments.
The strongest-post operator $\strongestPostOp_{op}$ describes the semantics of an operation $op \in Ops$.
It is the analogy to the transfer relation in the domain of CPAs.
It maps a region of concrete states, implied by an abstract variable assignment, to the region of all concrete states that can be reached by executing $op$.
The semantics of a path $\sigma = \langle (l_1, op_1), ..., (l_n, op_n) \rangle$ is defined as the consecutive application of the strongest-post operator to its constraint sequence $\gamma_\sigma = \langle op_1, ..., op_n \rangle$:
$\strongestPostOp_{\gamma_\sigma}(v) = \strongestPostOp_{op_n}(\strongestPostOp_{op_{n-1}} (...\ \strongestPostOp_{op_1}(v) ... ))$.
We use strongest-post operators during interpolation and refinement to evaluate paths.

The strongest-post operator $\strongestPostOpExplicit_{op}$ is defined in the following way:
%\begin{enumerate}[label=\alph*)]
%\item
For an assignment operation $s \assign exp$, $\strongestPostOpExplicit_{s \assign exp}(v) = v_\restrictedTo{X \setminus \{ s \}} \logicAnd v_{s \assign exp}$ with $v_{s \assign exp} = \{ (s, exp_\using{v}) \}$ and $exp_\using{v}$ denoting the evaluation of $exp$ using the abstract variable assignment $v$, as defined in Section \ref{sec:valueAnalysis}.
%\item
For an assume operation $\assume(p)$, 
	$\strongestPostOpExplicit_{\assume(p)}(v) = v'$ with 
	\[ v'(x) = \begin{dcases}
		\bot & \text{ if } \exists y \in \defRange(v) : v(y) = \bot \text{ or } p_\using{v} \text{ is unsatisfiable}\\
		c & \text{ if $c$ is the only satisfying assignment of $p_\using{v}$ for $x$}\\
		v(x) & \text{ if none of the above and } x \in \defRange(v)
	\end{dcases}\]
	with $p_\using{v}$ as defined in Section \ref{sec:valueAnalysis}.
%\end{enumerate}

%\subsubsection{Interpolation for abstract variable assignments}
\begin{algorithm}[t]
\caption{$\interpolateExplicit(\prefix, \suffix)$, adapted from \cite{Beyer2013}}
\label{alg:interpolateExplicit}
\begin{algorithmic}[1]
\Input two constraint sequences $\prefix$ and $\suffix$, with $\prefix \logicAnd \suffix$ contradicting
\Output a constraint sequence $\Gamma$, which is an interpolant for $\prefix$ and $\suffix$
\Variables an abstract variable assignment $v$

\State $v \assign \strongestPostOpExplicit_\prefix(\varnothing)$
\ForAll{$x \in \defRange(v)$}
	\If{$\strongestPostOpExplicit_\suffix(v_\restrictedTo{\defRange(v) \setminus \{x\}})$ is contradicting}
		\State $v \assign v_\restrictedTo{\defRange(v) \setminus \{x\}}$ \Comment $x$ not relevant, should not occur in interpolant
	\EndIf
\EndFor
\State $\Gamma \assign \langle \rangle$
\ForAll{$x \in \defRange(v)$} \label{alg:interpolateExplicit:itpStart}
	\State $\Gamma \assign \Gamma \logicAnd \langle \assume(x = v(x))\rangle$
\EndFor\\ \label{alg:interpolateExplicit:itpFinish}
\Return $\Gamma$
\end{algorithmic}
\end{algorithm}

The algorithm for interpolation in the domain of abstract variable assignments is shown in Algorithm \ref{alg:interpolateExplicit}.
For a prefix $\prefix$ and a suffix $\prefix$, the abstract variable assignment $v$, that results from applying $\prefix$ to the initial abstract variable assignment $\varnothing$ is computed.
Next, for each variable assignment in $v$ it is checked whether the assignment is necessary to prove that $\suffix$ is contradicting.
If it is not, it can be removed from $v$.
After all variable assignments are checked, $v$ only contains variable assignments that are necessary to prove that $\suffix$ is contradicting.
From these, the interpolant is built (Lines \ref{alg:interpolateExplicit:itpStart} - \ref{alg:interpolateExplicit:itpFinish}).

\begin{algorithm}[t]
\caption{$\refineExplicit{\sigma}$, adapted from \cite{Beyer2015}}
\label{alg:refinementExplicit}
\begin{algorithmic}[1]
\Input infeasible error path $\sigma = \langle (op_1, l_1), ..., (op_n, l_n) \rangle$
\Output precision $\pi$
\Variables interpolating constraint sequence $\Gamma$
\State $\Gamma \assign \langle \rangle$
\State $\pi(l) \assign \varnothing$ for all program locations $l$
\For{$i \assign 1$ to $n - 1$}\label{alg:refinementExplicit:loopStart}
	\State $\suffix \assign \langle op_{i+1}, ..., op_n \rangle$
	\State $\Gamma \assign \interpolateExplicit(\Gamma \logicAnd \langle op_i \rangle, \suffix)$ \Comment inductive interpolation \label{alg:refinementExplicit:interpolation}
	\State $\pi(l_i) \assign \extractPrecision{\Gamma}$
\EndFor\\
\Return $\pi$
\end{algorithmic}
\end{algorithm}

The interpolants produced are used in the refinement of the precision (Alg. \ref{alg:refinementExplicit}).
We use a location-specific precision $\pi : L \rightarrow 2^X$ that returns for a program location $l \in L$ all program variables of $X$ which are relevant for the analysis at this location. This approach realizes the lazy abstraction technique \cite{Henzinger2002}.
The algorithm starts with an initial, empty interpolant $\Gamma$ and empty precision $\pi$ with $\pi(l) = \varnothing$ for all $l \in L$.
For each location $(l_i, op_i)$ on the error path, the suffix $\suffix$ of this location are set and the interpolant is computed inductively from the existing interpolant in conjunction with the current operation $op_i$ and the suffix (Line \ref{alg:refinementExplicit:interpolation}).
A precision for the current program location is then extracted from the interpolant.
One straightforward way to do this is by using all program variables with a valid assignment in the  abstract variable assignment resulting from the application of the strongest-post operator to our interpolant:
\[\extractPrecision{\Gamma} = \{ x |\ (x, z) \in \strongestPostOpExplicit_\Gamma (\varnothing ) \text{ and } z \neq \bot_\valueset \}.\]
It is not only sufficient, but also required to use $\Gamma \logicAnd \langle op_i \rangle$ instead of the full prefix $\prefix = \langle op_1, ..., op_1 \rangle$ for interpolation. The full prefix cannot be used as it has to be assured that the precision resulting from these consecutive interpolations proves the error path infeasible. All information necessary for proving the infeasibility of the remaining error path is present in the current interpolant and operation.

This refinement procedure can be used in CEGAR (Alg. \ref{alg:cegar}) in combination with a CPA with precision adjustment that expects these precision types, like the \valueAnalysisCPA\ in combination with refinement for abstract variable assignments.

%\subsubsection{Refinement for the domain of linear arithmetics}
Refinement in the domain of linear arithmetics, as used for the \predicateCPA, uses a standard approach to refinement based on lazy abstraction and Craig interpolation.
The task of interpolation is delegated to an off-the-shelf SMT solver.

%\subsubsection{\ValueAnalysisCPA\ with precision adjustment}
%The \valueAnalysisCPA\ with dynamic precision adjustment \cite{Beyer2013} \[\valCPAPlus = (D_\valCPA, \Pi_\valCPAPlus, \transfer_\valCPAPlus, \cpaMerge^{sep}, \cpaStop^{sep}, %\cpaPrec_\valCPAPlus)\] is a CPA that can be, and is, used with the refinement for abstract variable assignments as described above.
%It consists of:
%\begin{enumerate}[leftmargin=*, label=\arabic*.]
%\item The abstract domain $D_\valCPA$ as defined in Section \ref{sec:valueAnalysis}.
%\item The set of precisions $\Pi_\valCPAPlus = L \rightarrow 2^X$. A precision $\pi \in \Pi_\valCPAPlus$ specifies a subset of program variables of $X$ that are tracked.
%\item The transfer relation $\transfer_\valCPAPlus$ contains the transfer $v \transfer_\valCPAPlus (v', \pi)$ if $v \transfer_\valCPA v'$.
%\item The merge operator $\cpaMerge^{sep}$ that performs no merging.
%\item The termination check $\cpaStop^{sep}$ that checks every state individually.
%\item The precision adjustment $\cpaPrec_\valCPAPlus$. Given an abstract state $v$ and a precision $\pi$, all abstract assignments of variables that do not occur in $\pi$ are removed from %$v$. This is done by restricting the partial function: $\cpaPrec_\valCPAPlus(v, \pi) = (v_\restrictedTo{\pi}, \pi)$. The given precision is returned as it is.
%\end{enumerate}

In this chapter, we gave an overview of all theoretical concepts that are necessary to describe our own work. We introduced the concept of configurable software verification and configurable program analyses (CPAs), a very versatile approach to automated software verification. We introduced different CPAs we use in this work and CEGAR with precision refinement for both linear arithmetics and abstract variable assignments, which we will use when applying CEGAR to the \symbolicExecutionCPA.

%\subsection{Bounded loops with continuation after reached bound}
%\subsubsection{Idea}
%\subsubsection{Existing LoopstackCPA}
%\subsubsection{No continuation after reached bound: Existing AssumptionStorageCPA}
%\subsubsection{New: Jump out of loop at every possible exit and abstract information}


%%%%%%%%%%%%%%%%%%%%%%%%%%%%%%%%
\section{Definitions of newly introduced concepts}
\section{A different merge operator for \constraintsCPA}
\label{sec:newMerge}
For every operation $\assume(p)$ at a location $l$ that transfers the control flow to a location $l'$ there exists another operation $\assume(\neg p)$ at the same location transfering the control flow to a location $l'' \neq l'$.
In most programs it is probable that the two different program branches starting at $l'$ and $l''$ meet again, that means that for a later program location $l'''$ two abstract states $a, a'$ of the \constraintsCPA\ (in the following  called \emph{constraints states}) exist with $a$ containing $p$ and $a'$ containing $\neg p$.

If a constraint $p$ is part of an abstract state $a$, $p$ is true in all concrete states represented by $a$ (just like a predicate in an abstract state of the \predicateCPA\ \cite{Beyer2008}).
If for one program location $l$ two constraints states $a, a'$ exist with $p \in a$ and $\neg p \in a'$ and $a \setminus \{ p \} = a' \setminus \{ \neg p \}$,
then $a$ represents all concrete states for which $p \logicAnd a \setminus \{ p \}$ is true and $a'$ represents all concrete states for which $\neg p \logicAnd a \setminus \{ p \}$ is true.
At this point, the analysis will never be able to prove a program location as infeasible because of $p$ or $\neg p$.
If $a'$ reaches a program location and computes it as infeasible by using $p$, the abstract state $a$ will compute the same program location as feasible, if it reaches it.
Because of this, it seems legit to delete these obsolete constraints and only continue with one more abstract state instead of two more concrete ones by using the merge operator
\[ \cpaMerge (a, a', \pi) = \begin{dcases}
a' \setminus \neg Q & \text{ if } a \lesserEqual a' \setminus \neg Q \\
a' & \text{ otherwise}
\end{dcases} \]
with $\neg Q = \{\neg p |\ p \in a \logicAnd \neg p \in a' \}$ and $Q = \{ p |\ p \in a \logicAnd \neg p \in a' \}$.
It is not necessary that $a' \setminus \neg Q = a \setminus Q$.
If $a' \setminus \neg Q$ represents a super set of the concrete states represented by $a \setminus Q$, that is $a \setminus Q \lesserEqual a' \setminus \neg Q$, then the above condition is true, and $a \lesserEqual a \setminus Q$.

This condition is automatically checked by the $\mergeAgree$ operator, so we can simply use $\cpaMerge(a, a', \pi) = a' \setminus \neg Q$.

\section{Different Less-or-equal Operators}
%\subsubsection{Subset operator}
\label{sec:leqOperators}
The less-or-equal operator is the operator executed the most often during analyses as $\cpaStop^{sep}$ uses it once for every state in the reached set, at every iteration of the CPA algorithm.
In addition, it is responsible for determining whether a new state is already covered and analysis can be stopped at this point.
Although the implementation framework \cpaChecker\ only performs a termination check for reached states at the same location, 
its speed and precision can make a great difference for the performance of our analysis.

\paragraph*{Aliasing operator}
The  less-or-equal operators we used for \symbolicValueAnalysisCPA\ and \constraintsCPA\ in \cite{Lemberger2015} using an \aliasFunc\ function try to be more precise than a simple subset check.
Unfortunately, they can result in false behaviour because of their independent behaviour.
Consider the two pairs of value state and constraint state
$e = (v, a)$ with $v = \{x \rightarrow s1, y \rightarrow s2\}$, $a = \{s1 > 0\})$ and
$e' = (v', a')$ with $v' = \{x \rightarrow s2, y \rightarrow s1\}$, $a' = \{s1 > 0\})$.
When using the aliasing less-or-equal operators of the \symbolicValueAnalysisCPA\ and of the \constraintsCPA,
the \symbolicValueAnalysisCPA\ states
$v \lesserEqual v'$ for $\aliasFunc$ function $\aliasFunc(s1) = s2$, $\aliasFunc(s2) = s1$ and
the \constraintsCPA\ states
$a \lesserEqual a'$ for $\aliasFunc(s1) = s1$.
Because of this, $e \lesserEqual e'$, although the concrete states
$\llbracket e \rrbracket = \{ c \in C |\ c(x) > 0 \}$ and
$\llbracket e' \rrbracket = \{ c \in C |\ c(y) > 0 \}$
represented by $e$ and $e'$ are two different sets.
This violates the definition of the less-or-equal operator for abstract domains (Section~\ref{sec:abstractState}).
For this example, the less-or-equal operator of the \constraintsCPA\ actually behaves like the subset operator, since $\aliasFunc$ represents the identity.
This shows that the less-or-equal operator of the \symbolicValueAnalysisCPA\ cannot be used, regardless of the operator used by the \constraintsCPA.
Besides the default less-or-equal operator for the \constraintsCPA\ 
presented in Section~\ref{sec:constraintsCPA}, another operator might prove useful.

%\subsubsection{Implication operator}
\paragraph*{Implication operator}
Since a \constraintsCPA 's abstract state $a$ is interpreted as the conjunction of its constraints $\varphi_a$, it seems fit to use implication as the less-or-equal operator.
Remember that $\llbracket a \rrbracket = \{ c \in C |\ c \satisfies \varphi_a \}$.
If a formula $\varphi_a$ implies a formula $\varphi_{a'}$ and $c$ satisfies $\varphi_a$, then $c$ also satisfies $\varphi_{a'}$.
Because of this 
\[\llbracket a \rrbracket = \{ c \in C |\ c \satisfies \varphi_a \} \subseteq \{ c \in C |\ c \satisfies \varphi_{a'} \} = \llbracket a' \rrbracket \text{ if } \varphi_a \Rightarrow \varphi_{a'}.\]
The less-or-equal operator for the \constraintsCPA\ using implication is defined as $a \lesserEqualImpl a'$ if $\varphi_a \Rightarrow \varphi_{a'}$.
This operator has a higher precision than $\lesserEqualSub$ but requires SAT checks, which are definitely worse in performance than merely checking whether one set is the subset of another.


\subsection{Location/Frequency of SAT checks}
\subsubsection{After every assume}
\subsubsection{At every target location only}

\subsection{Basic CEGAR and its algorithm}
\subsubsection{CEGAR and interpolation in general}
Counterexample-guided abstraction refinement (CEGAR) \cite{Clarke2003} is a technique to find an abstraction that contains as few information as possible while retaining the possibility to prove or disprove a program's correctness.
This technique can greatly reduce the number of abstract states in a program's analysis and is considered ''the most general and flexible for handling the state explosion problem,''\cite{Clarke2003}\ the major problem we are facing with our \symbolicExecutionCPA.

The technique starts analysis with a coarse abstraction and refines it based on counterexamples. A counterexample is a witness of a property violation.\cite{Beyer2013}
If no error path is found by the analysis, it terminates and reports that no property violation exists.
If an error path is found, it is checked whether the path is feasible (i.e. a possible program execution) by repeating the analysis with full precision.
If the path is feasible, the analysis terminates and reports the found property violation.
If the error path is infeasible it was only found because the abstraction is too coarse. As a consequence, the abstraction is refined using the error path.
After this, the analysis starts again, using the new abstraction.

Since the problem of finding the coarsest possible refinement of an abstraction based on an error path is NP-hard, \cite{Clarke2003}\ good heuristics have to be used to find good refinements.
Interpolation \cite{Henzinger2004}\ is one such technique in a boolean context that is used for refinement of both the \predicateCPA\ and \valueAnalysisCPA.

\subsubsection{CEGAR and interpolation in the context of configurable software verification}
\label{sec:cegarBasics}
To apply CEGAR and interpolation to configurable software verification, a simple modification has to be made to the CPA algorithm.
Instead of passing it an initial state $e_0$ and an initial precision $\pi_0$, we use an initial reached set $R_0$ and initial waitlist $W_0$ (Alg. \ref{alg:cpaPlus}).
This way we can control at which point the analysis continues after a refinement was performed.

\begin{algorithm}[t]
\caption{$CPA(\cpaPlus, R_0, W_0)$, adapted from \cite{Beyer2013}}
\label{alg:cpaPlus}
\begin{algorithmic}[1]

\Input a CPA $\cpaPlus = (D, \Pi, \transfer, \cpaMerge, \cpaStop, \cpaPrec)$,
	    a set $R_0 \subseteq (E \times \Pi)$ of initial states with their precision and
	    a subset $W_0 \subseteq R_0$ of frontier abstract states with their precision,
	    with $E$ being the set of elements of $D$
\Output a set of abstract states reachable from $R_0$ with their precision and
	   a subset of frontier abstract states with their precision
\Variables \reachedSet\ and \waitlistSet , both subsets of $E \times \Pi$
\State $\reachedSet \assign R_0$
\State $\waitlistSet \assign W_0$
\While{$\waitlistSet \neq \varnothing$} \Comment from here on the same as before
\State ...
\EndWhile
\end{algorithmic}
\end{algorithm}

\begin{algorithm}[t]
\caption{$CEGAR(\cpaPlus, e_0, \pi_0)$, adapted from \cite{Beyer2013}}
\label{alg:cegar}
\begin{algorithmic}[1]
\Input a CPA $\cpaPlus = (D, \Pi, \transfer, \cpaMerge, \cpaStop, \cpaPrec)$ with dynamic precision adjustment,
	an initial abstract state $e_0 \in E$ with precision $\pi_0 \in \Pi$,
	with $E$ denoting the set of elements of the semi-lattice of $D$
\Output the verification result \safe\ or \unsafe
\Variables the sets \reachedSet\ and \waitlistSet\ of elements of $E \times \Pi$,
	      an error path $\sigma = \langle (op_1, l_1), ..., (op_n, l_n) \rangle$\\

\State $\reachedSet \assign \{ (e_0, \pi_0) \}$
\State $\waitlistSet \assign \{ e_0, \pi_0 \}$
\State $\pi \assign \pi_0$
\While{true}
	\State $(\reachedSet, \waitlistSet) \assign CPA(\cpaPlus, \reachedSet, \waitlistSet)$
	\If{$\waitlistSet = \varnothing$}
		\Return \safe
	\Else
		\State $\sigma \assign \extractErrorPath{\reachedSet}$ \label{alg:cegar:extraction}
		\If{\isFeasible{$\sigma$}} \Comment error path feasible: report bug \label{alg:cegar:feasibilityCheck}
			\State % empty state for new line after if
			\Return \unsafe 
		\Else \Comment error path infeasible: refine and restart from the beginning
			\State $\pi \assign \pi \cup \refine{\sigma}$
			\State $\reachedSet \assign (e_0, \pi)$
			\State $\waitlistSet \assign (e_0, \pi)$ \label{alg:cegar:end}
		\EndIf
	\EndIf
\EndWhile
\end{algorithmic}
\end{algorithm}

Now that the CPA algorithm is able to use precisions created in a refinement procedure, we use it as a part of our complete CEGAR algorithm.
Algorithm \ref{alg:cegar} uses a CPA using dynamic precision adjustment $\cpaPlus$,
an initial state $e_0$
and an initial precision $\pi_0$
to compute whether a property violation exists.

First, the $CPA$ algorithm is used to compute a set of reached abstract states ($\reachedSet$) and a subset of this set that contains all reached abstract states that have not been handled yet ($\waitlistSet$).
If $\waitlistSet$ is empty, the $CPA$ algorithm has handled all reachable states without encountering any target state.
If this is the case, no property violation was found and the algorithm can return \safe.
Otherwise, an error path is extracted from the reached set.
If the error path is reported as feasible, a property violation exists or the algorithm is not able to prove that none exists. It returns \unsafe.
If the error path is infeasible, the current precision is too abstract.
It is refined based on the infeasible error path by using $\refineFunc : \Sigma \rightarrow \Pi$ with $\Sigma$ being the set of all error paths, so that it can prove its infeasibility.
After this, the reached set and waitlist are reset to their initial values and the algorithm repeats analysis with the refined precision.
It is important to notice that the return type of $\refineFunc$ has to be equal to the precision type $\Pi$ used in $\cpaPlus$.
Because of this, CPAs are not exchangeable without changing refinement, too, in general.

%%%%%%%%%%%%%%%%%%%%%%%%%%
%%% Refinement in general
%%%%%%%%%%%%%%%%%%%%%%%%%%
For refinement, the priorly mentioned technique of interpolation is used to determine a location-specific precision that is strong enough for the CPA algorithm with precision adjustment to prove that a given error path is infeasible.
A boolean formula $\craigItp$ is a Craig interpolant \cite{Craig1957}\ for two boolean formulas $\prefix$ (called prefix) and $\suffix$ (called suffix), if the following three conditions are fulfilled:
\begin{enumerate}[label=\alph*)]
\item The prefix implies $\craigItp$, that is $\prefix \Rightarrow \craigItp$.
\item $\craigItp$ contradicts the suffix, that means $\craigItp \logicAnd \suffix$ is contradicting.
\item $\craigItp$ only contains variables occurring in \emph{both} prefix and suffix.
\end{enumerate}
It is proven that such an interpolant always exists in the domain of abstract variable assignments \cite{Beyer2013} as well as in the theory of linear arithmetics \cite{Craig1957}.

%\subsubsection{Refinement for the domain of abstract variable assignments}
Our work is strongly based on the refinement technique for abstract variable assignments.
The strongest-post operator $\strongestPostOp_{op}$ describes the semantics of an operation $op \in Ops$.
It is the analogy to the transfer relation in the domain of CPAs.
It maps a region of concrete states, implied by an abstract variable assignment, to the region of all concrete states that can be reached by executing $op$.
The semantics of a path $\sigma = \langle (l_1, op_1), ..., (l_n, op_n) \rangle$ is defined as the consecutive application of the strongest-post operator to its constraint sequence $\gamma_\sigma = \langle op_1, ..., op_n \rangle$:
$\strongestPostOp_{\gamma_\sigma}(v) = \strongestPostOp_{op_n}(\strongestPostOp_{op_{n-1}} (...\ \strongestPostOp_{op_1}(v) ... ))$.
We use strongest-post operators during interpolation and refinement to evaluate paths.

The strongest-post operator $\strongestPostOpExplicit_{op}$ is defined in the following way:
%\begin{enumerate}[label=\alph*)]
%\item
For an assignment operation $s \assign exp$, $\strongestPostOpExplicit_{s \assign exp}(v) = v_\restrictedTo{X \setminus \{ s \}} \logicAnd v_{s \assign exp}$ with $v_{s \assign exp} = \{ (s, exp_\using{v}) \}$ and $exp_\using{v}$ denoting the evaluation of $exp$ using the abstract variable assignment $v$, as defined in Section \ref{sec:valueAnalysis}.
%\item
For an assume operation $\assume(p)$, 
	$\strongestPostOpExplicit_{\assume(p)}(v) = v'$ with 
	\[ v'(x) = \begin{dcases}
		\bot & \text{ if } \exists y \in \defRange(v) : v(y) = \bot \text{ or } p_\using{v} \text{ is unsatisfiable}\\
		c & \text{ if $c$ is the only satisfying assignment of $p_\using{v}$ for $x$}\\
		v(x) & \text{ if none of the above and } x \in \defRange(v)
	\end{dcases}\]
	with $p_\using{v}$ as defined in Section \ref{sec:valueAnalysis}.
%\end{enumerate}

%\subsubsection{Interpolation for abstract variable assignments}
\begin{algorithm}[t]
\caption{$\interpolateExplicit(\prefix, \suffix)$, adapted from \cite{Beyer2013}}
\label{alg:interpolateExplicit}
\begin{algorithmic}[1]
\Input two constraint sequences $\prefix$ and $\suffix$, with $\prefix \logicAnd \suffix$ contradicting
\Output a constraint sequence $\Gamma$, which is an interpolant for $\prefix$ and $\suffix$
\Variables an abstract variable assignment $v$

\State $v \assign \strongestPostOpExplicit_\prefix(\varnothing)$
\ForAll{$x \in \defRange(v)$}
	\If{$\strongestPostOpExplicit_\suffix(v_\restrictedTo{\defRange(v) \setminus \{x\}})$ is contradicting}
		\State $v \assign v_\restrictedTo{\defRange(v) \setminus \{x\}}$ \Comment $x$ not relevant, should not occur in interpolant
	\EndIf
\EndFor
\State $\Gamma \assign \langle \rangle$
\ForAll{$x \in \defRange(v)$} \label{alg:interpolateExplicit:itpStart}
	\State $\Gamma \assign \Gamma \logicAnd \langle \assume(x = v(x))\rangle$
\EndFor\\ \label{alg:interpolateExplicit:itpFinish}
\Return $\Gamma$
\end{algorithmic}
\end{algorithm}

The algorithm for interpolation in the domain of abstract variable assignments is shown in Algorithm \ref{alg:interpolateExplicit}.
For a prefix $\prefix$ and a suffix $\prefix$, the abstract variable assignment $v$, that results from applying $\prefix$ to the initial abstract variable assignment $\varnothing$ is computed.
Next, for each variable assignment in $v$ it is checked whether the assignment is necessary to prove that $\suffix$ is contradicting.
If it is not, it can be removed from $v$.
After all variable assignments are checked, $v$ only contains variable assignments that are necessary to prove that $\suffix$ is contradicting.
From these, the interpolant is built (Lines \ref{alg:interpolateExplicit:itpStart} - \ref{alg:interpolateExplicit:itpFinish}).

\begin{algorithm}[t]
\caption{$\refineExplicit{\sigma}$, adapted from \cite{Beyer2015}}
\label{alg:refinementExplicit}
\begin{algorithmic}[1]
\Input infeasible error path $\sigma = \langle (op_1, l_1), ..., (op_n, l_n) \rangle$
\Output precision $\pi$
\Variables interpolating constraint sequence $\Gamma$
\State $\Gamma \assign \langle \rangle$
\State $\pi(l) \assign \varnothing$ for all program locations $l$
\For{$i \assign 1$ to $n - 1$}\label{alg:refinementExplicit:loopStart}
	\State $\suffix \assign \langle op_{i+1}, ..., op_n \rangle$
	\State $\Gamma \assign \interpolateExplicit(\Gamma \logicAnd \langle op_i \rangle, \suffix)$ \Comment inductive interpolation \label{alg:refinementExplicit:interpolation}
	\State $\pi(l_i) \assign \extractPrecision{\Gamma}$
\EndFor\\
\Return $\pi$
\end{algorithmic}
\end{algorithm}

The interpolants produced are used in the refinement of the precision (Alg. \ref{alg:refinementExplicit}).
We use a location-specific precision $\pi : L \rightarrow 2^X$ that returns for a program location $l \in L$ all program variables of $X$ which are relevant for the analysis at this location. This approach realizes the lazy abstraction technique \cite{Henzinger2002}.
The algorithm starts with an initial, empty interpolant $\Gamma$ and empty precision $\pi$ with $\pi(l) = \varnothing$ for all $l \in L$.
For each location $(l_i, op_i)$ on the error path, the suffix $\suffix$ of this location are set and the interpolant is computed inductively from the existing interpolant in conjunction with the current operation $op_i$ and the suffix (Line \ref{alg:refinementExplicit:interpolation}).
A precision for the current program location is then extracted from the interpolant.
One straightforward way to do this is by using all program variables with a valid assignment in the  abstract variable assignment resulting from the application of the strongest-post operator to our interpolant:
\[\extractPrecision{\Gamma} = \{ x |\ (x, z) \in \strongestPostOpExplicit_\Gamma (\varnothing ) \text{ and } z \neq \bot_\valueset \}.\]
It is not only sufficient, but also required to use $\Gamma \logicAnd \langle op_i \rangle$ instead of the full prefix $\prefix = \langle op_1, ..., op_1 \rangle$ for interpolation. The full prefix cannot be used as it has to be assured that the precision resulting from these consecutive interpolations proves the error path infeasible. All information necessary for proving the infeasibility of the remaining error path is present in the current interpolant and operation.

This refinement procedure can be used in CEGAR (Alg. \ref{alg:cegar}) in combination with a CPA with precision adjustment that expects these precision types, like the \valueAnalysisCPA\ in combination with refinement for abstract variable assignments.

%\subsubsection{Refinement for the domain of linear arithmetics}
Refinement in the domain of linear arithmetics, as used for the \predicateCPA, uses a standard approach to refinement based on lazy abstraction and Craig interpolation.
The task of interpolation is delegated to an off-the-shelf SMT solver.

%\subsubsection{\ValueAnalysisCPA\ with precision adjustment}
%The \valueAnalysisCPA\ with dynamic precision adjustment \cite{Beyer2013} \[\valCPAPlus = (D_\valCPA, \Pi_\valCPAPlus, \transfer_\valCPAPlus, \cpaMerge^{sep}, \cpaStop^{sep}, %\cpaPrec_\valCPAPlus)\] is a CPA that can be, and is, used with the refinement for abstract variable assignments as described above.
%It consists of:
%\begin{enumerate}[leftmargin=*, label=\arabic*.]
%\item The abstract domain $D_\valCPA$ as defined in Section \ref{sec:valueAnalysis}.
%\item The set of precisions $\Pi_\valCPAPlus = L \rightarrow 2^X$. A precision $\pi \in \Pi_\valCPAPlus$ specifies a subset of program variables of $X$ that are tracked.
%\item The transfer relation $\transfer_\valCPAPlus$ contains the transfer $v \transfer_\valCPAPlus (v', \pi)$ if $v \transfer_\valCPA v'$.
%\item The merge operator $\cpaMerge^{sep}$ that performs no merging.
%\item The termination check $\cpaStop^{sep}$ that checks every state individually.
%\item The precision adjustment $\cpaPrec_\valCPAPlus$. Given an abstract state $v$ and a precision $\pi$, all abstract assignments of variables that do not occur in $\pi$ are removed from %$v$. This is done by restricting the partial function: $\cpaPrec_\valCPAPlus(v, \pi) = (v_\restrictedTo{\pi}, \pi)$. The given precision is returned as it is.
%\end{enumerate}

In this chapter, we gave an overview of all theoretical concepts that are necessary to describe our own work. We introduced the concept of configurable software verification and configurable program analyses (CPAs), a very versatile approach to automated software verification. We introduced different CPAs we use in this work and CEGAR with precision refinement for both linear arithmetics and abstract variable assignments, which we will use when applying CEGAR to the \symbolicExecutionCPA.

%\subsection{Bounded loops with continuation after reached bound}
%\subsubsection{Idea}
%\subsubsection{Existing LoopstackCPA}
%\subsubsection{No continuation after reached bound: Existing AssumptionStorageCPA}
%\subsubsection{New: Jump out of loop at every possible exit and abstract information}


%%%%%%%%%%%%%%%%%%%%%%%%%%%%%%%
\section{Definitions of newly introduced concepts}
\section{A different merge operator for \constraintsCPA}
\label{sec:newMerge}
For every operation $\assume(p)$ at a location $l$ that transfers the control flow to a location $l'$ there exists another operation $\assume(\neg p)$ at the same location transfering the control flow to a location $l'' \neq l'$.
In most programs it is probable that the two different program branches starting at $l'$ and $l''$ meet again, that means that for a later program location $l'''$ two abstract states $a, a'$ of the \constraintsCPA\ (in the following  called \emph{constraints states}) exist with $a$ containing $p$ and $a'$ containing $\neg p$.

If a constraint $p$ is part of an abstract state $a$, $p$ is true in all concrete states represented by $a$ (just like a predicate in an abstract state of the \predicateCPA\ \cite{Beyer2008}).
If for one program location $l$ two constraints states $a, a'$ exist with $p \in a$ and $\neg p \in a'$ and $a \setminus \{ p \} = a' \setminus \{ \neg p \}$,
then $a$ represents all concrete states for which $p \logicAnd a \setminus \{ p \}$ is true and $a'$ represents all concrete states for which $\neg p \logicAnd a \setminus \{ p \}$ is true.
At this point, the analysis will never be able to prove a program location as infeasible because of $p$ or $\neg p$.
If $a'$ reaches a program location and computes it as infeasible by using $p$, the abstract state $a$ will compute the same program location as feasible, if it reaches it.
Because of this, it seems legit to delete these obsolete constraints and only continue with one more abstract state instead of two more concrete ones by using the merge operator
\[ \cpaMerge (a, a', \pi) = \begin{dcases}
a' \setminus \neg Q & \text{ if } a \lesserEqual a' \setminus \neg Q \\
a' & \text{ otherwise}
\end{dcases} \]
with $\neg Q = \{\neg p |\ p \in a \logicAnd \neg p \in a' \}$ and $Q = \{ p |\ p \in a \logicAnd \neg p \in a' \}$.
It is not necessary that $a' \setminus \neg Q = a \setminus Q$.
If $a' \setminus \neg Q$ represents a super set of the concrete states represented by $a \setminus Q$, that is $a \setminus Q \lesserEqual a' \setminus \neg Q$, then the above condition is true, and $a \lesserEqual a \setminus Q$.

This condition is automatically checked by the $\mergeAgree$ operator, so we can simply use $\cpaMerge(a, a', \pi) = a' \setminus \neg Q$.

\section{Different Less-or-equal Operators}
%\subsubsection{Subset operator}
\label{sec:leqOperators}
The less-or-equal operator is the operator executed the most often during analyses as $\cpaStop^{sep}$ uses it once for every state in the reached set, at every iteration of the CPA algorithm.
In addition, it is responsible for determining whether a new state is already covered and analysis can be stopped at this point.
Although the implementation framework \cpaChecker\ only performs a termination check for reached states at the same location, 
its speed and precision can make a great difference for the performance of our analysis.

\paragraph*{Aliasing operator}
The  less-or-equal operators we used for \symbolicValueAnalysisCPA\ and \constraintsCPA\ in \cite{Lemberger2015} using an \aliasFunc\ function try to be more precise than a simple subset check.
Unfortunately, they can result in false behaviour because of their independent behaviour.
Consider the two pairs of value state and constraint state
$e = (v, a)$ with $v = \{x \rightarrow s1, y \rightarrow s2\}$, $a = \{s1 > 0\})$ and
$e' = (v', a')$ with $v' = \{x \rightarrow s2, y \rightarrow s1\}$, $a' = \{s1 > 0\})$.
When using the aliasing less-or-equal operators of the \symbolicValueAnalysisCPA\ and of the \constraintsCPA,
the \symbolicValueAnalysisCPA\ states
$v \lesserEqual v'$ for $\aliasFunc$ function $\aliasFunc(s1) = s2$, $\aliasFunc(s2) = s1$ and
the \constraintsCPA\ states
$a \lesserEqual a'$ for $\aliasFunc(s1) = s1$.
Because of this, $e \lesserEqual e'$, although the concrete states
$\llbracket e \rrbracket = \{ c \in C |\ c(x) > 0 \}$ and
$\llbracket e' \rrbracket = \{ c \in C |\ c(y) > 0 \}$
represented by $e$ and $e'$ are two different sets.
This violates the definition of the less-or-equal operator for abstract domains (Section~\ref{sec:abstractState}).
For this example, the less-or-equal operator of the \constraintsCPA\ actually behaves like the subset operator, since $\aliasFunc$ represents the identity.
This shows that the less-or-equal operator of the \symbolicValueAnalysisCPA\ cannot be used, regardless of the operator used by the \constraintsCPA.
Besides the default less-or-equal operator for the \constraintsCPA\ 
presented in Section~\ref{sec:constraintsCPA}, another operator might prove useful.

%\subsubsection{Implication operator}
\paragraph*{Implication operator}
Since a \constraintsCPA 's abstract state $a$ is interpreted as the conjunction of its constraints $\varphi_a$, it seems fit to use implication as the less-or-equal operator.
Remember that $\llbracket a \rrbracket = \{ c \in C |\ c \satisfies \varphi_a \}$.
If a formula $\varphi_a$ implies a formula $\varphi_{a'}$ and $c$ satisfies $\varphi_a$, then $c$ also satisfies $\varphi_{a'}$.
Because of this 
\[\llbracket a \rrbracket = \{ c \in C |\ c \satisfies \varphi_a \} \subseteq \{ c \in C |\ c \satisfies \varphi_{a'} \} = \llbracket a' \rrbracket \text{ if } \varphi_a \Rightarrow \varphi_{a'}.\]
The less-or-equal operator for the \constraintsCPA\ using implication is defined as $a \lesserEqualImpl a'$ if $\varphi_a \Rightarrow \varphi_{a'}$.
This operator has a higher precision than $\lesserEqualSub$ but requires SAT checks, which are definitely worse in performance than merely checking whether one set is the subset of another.


\subsection{Location/Frequency of SAT checks}
\subsubsection{After every assume}
\subsubsection{At every target location only}

\subsection{Basic CEGAR and its algorithm}
\subsubsection{CEGAR and interpolation in general}
Counterexample-guided abstraction refinement (CEGAR) \cite{Clarke2003} is a technique to find an abstraction that contains as few information as possible while retaining the possibility to prove or disprove a program's correctness.
This technique can greatly reduce the number of abstract states in a program's analysis and is considered ''the most general and flexible for handling the state explosion problem,''\cite{Clarke2003}\ the major problem we are facing with our \symbolicExecutionCPA.

The technique starts analysis with a coarse abstraction and refines it based on counterexamples. A counterexample is a witness of a property violation.\cite{Beyer2013}
If no error path is found by the analysis, it terminates and reports that no property violation exists.
If an error path is found, it is checked whether the path is feasible (i.e. a possible program execution) by repeating the analysis with full precision.
If the path is feasible, the analysis terminates and reports the found property violation.
If the error path is infeasible it was only found because the abstraction is too coarse. As a consequence, the abstraction is refined using the error path.
After this, the analysis starts again, using the new abstraction.

Since the problem of finding the coarsest possible refinement of an abstraction based on an error path is NP-hard, \cite{Clarke2003}\ good heuristics have to be used to find good refinements.
Interpolation \cite{Henzinger2004}\ is one such technique in a boolean context that is used for refinement of both the \predicateCPA\ and \valueAnalysisCPA.

\subsubsection{CEGAR and interpolation in the context of configurable software verification}
\label{sec:cegarBasics}
To apply CEGAR and interpolation to configurable software verification, a simple modification has to be made to the CPA algorithm.
Instead of passing it an initial state $e_0$ and an initial precision $\pi_0$, we use an initial reached set $R_0$ and initial waitlist $W_0$ (Alg. \ref{alg:cpaPlus}).
This way we can control at which point the analysis continues after a refinement was performed.

\begin{algorithm}[t]
\caption{$CPA(\cpaPlus, R_0, W_0)$, adapted from \cite{Beyer2013}}
\label{alg:cpaPlus}
\begin{algorithmic}[1]

\Input a CPA $\cpaPlus = (D, \Pi, \transfer, \cpaMerge, \cpaStop, \cpaPrec)$,
	    a set $R_0 \subseteq (E \times \Pi)$ of initial states with their precision and
	    a subset $W_0 \subseteq R_0$ of frontier abstract states with their precision,
	    with $E$ being the set of elements of $D$
\Output a set of abstract states reachable from $R_0$ with their precision and
	   a subset of frontier abstract states with their precision
\Variables \reachedSet\ and \waitlistSet , both subsets of $E \times \Pi$
\State $\reachedSet \assign R_0$
\State $\waitlistSet \assign W_0$
\While{$\waitlistSet \neq \varnothing$} \Comment from here on the same as before
\State ...
\EndWhile
\end{algorithmic}
\end{algorithm}

\begin{algorithm}[t]
\caption{$CEGAR(\cpaPlus, e_0, \pi_0)$, adapted from \cite{Beyer2013}}
\label{alg:cegar}
\begin{algorithmic}[1]
\Input a CPA $\cpaPlus = (D, \Pi, \transfer, \cpaMerge, \cpaStop, \cpaPrec)$ with dynamic precision adjustment,
	an initial abstract state $e_0 \in E$ with precision $\pi_0 \in \Pi$,
	with $E$ denoting the set of elements of the semi-lattice of $D$
\Output the verification result \safe\ or \unsafe
\Variables the sets \reachedSet\ and \waitlistSet\ of elements of $E \times \Pi$,
	      an error path $\sigma = \langle (op_1, l_1), ..., (op_n, l_n) \rangle$\\

\State $\reachedSet \assign \{ (e_0, \pi_0) \}$
\State $\waitlistSet \assign \{ e_0, \pi_0 \}$
\State $\pi \assign \pi_0$
\While{true}
	\State $(\reachedSet, \waitlistSet) \assign CPA(\cpaPlus, \reachedSet, \waitlistSet)$
	\If{$\waitlistSet = \varnothing$}
		\Return \safe
	\Else
		\State $\sigma \assign \extractErrorPath{\reachedSet}$ \label{alg:cegar:extraction}
		\If{\isFeasible{$\sigma$}} \Comment error path feasible: report bug \label{alg:cegar:feasibilityCheck}
			\State % empty state for new line after if
			\Return \unsafe 
		\Else \Comment error path infeasible: refine and restart from the beginning
			\State $\pi \assign \pi \cup \refine{\sigma}$
			\State $\reachedSet \assign (e_0, \pi)$
			\State $\waitlistSet \assign (e_0, \pi)$ \label{alg:cegar:end}
		\EndIf
	\EndIf
\EndWhile
\end{algorithmic}
\end{algorithm}

Now that the CPA algorithm is able to use precisions created in a refinement procedure, we use it as a part of our complete CEGAR algorithm.
Algorithm \ref{alg:cegar} uses a CPA using dynamic precision adjustment $\cpaPlus$,
an initial state $e_0$
and an initial precision $\pi_0$
to compute whether a property violation exists.

First, the $CPA$ algorithm is used to compute a set of reached abstract states ($\reachedSet$) and a subset of this set that contains all reached abstract states that have not been handled yet ($\waitlistSet$).
If $\waitlistSet$ is empty, the $CPA$ algorithm has handled all reachable states without encountering any target state.
If this is the case, no property violation was found and the algorithm can return \safe.
Otherwise, an error path is extracted from the reached set.
If the error path is reported as feasible, a property violation exists or the algorithm is not able to prove that none exists. It returns \unsafe.
If the error path is infeasible, the current precision is too abstract.
It is refined based on the infeasible error path by using $\refineFunc : \Sigma \rightarrow \Pi$ with $\Sigma$ being the set of all error paths, so that it can prove its infeasibility.
After this, the reached set and waitlist are reset to their initial values and the algorithm repeats analysis with the refined precision.
It is important to notice that the return type of $\refineFunc$ has to be equal to the precision type $\Pi$ used in $\cpaPlus$.
Because of this, CPAs are not exchangeable without changing refinement, too, in general.

%%%%%%%%%%%%%%%%%%%%%%%%%%
%%% Refinement in general
%%%%%%%%%%%%%%%%%%%%%%%%%%
For refinement, the priorly mentioned technique of interpolation is used to determine a location-specific precision that is strong enough for the CPA algorithm with precision adjustment to prove that a given error path is infeasible.
A boolean formula $\craigItp$ is a Craig interpolant \cite{Craig1957}\ for two boolean formulas $\prefix$ (called prefix) and $\suffix$ (called suffix), if the following three conditions are fulfilled:
\begin{enumerate}[label=\alph*)]
\item The prefix implies $\craigItp$, that is $\prefix \Rightarrow \craigItp$.
\item $\craigItp$ contradicts the suffix, that means $\craigItp \logicAnd \suffix$ is contradicting.
\item $\craigItp$ only contains variables occurring in \emph{both} prefix and suffix.
\end{enumerate}
It is proven that such an interpolant always exists in the domain of abstract variable assignments \cite{Beyer2013} as well as in the theory of linear arithmetics \cite{Craig1957}.

%\subsubsection{Refinement for the domain of abstract variable assignments}
Our work is strongly based on the refinement technique for abstract variable assignments.
The strongest-post operator $\strongestPostOp_{op}$ describes the semantics of an operation $op \in Ops$.
It is the analogy to the transfer relation in the domain of CPAs.
It maps a region of concrete states, implied by an abstract variable assignment, to the region of all concrete states that can be reached by executing $op$.
The semantics of a path $\sigma = \langle (l_1, op_1), ..., (l_n, op_n) \rangle$ is defined as the consecutive application of the strongest-post operator to its constraint sequence $\gamma_\sigma = \langle op_1, ..., op_n \rangle$:
$\strongestPostOp_{\gamma_\sigma}(v) = \strongestPostOp_{op_n}(\strongestPostOp_{op_{n-1}} (...\ \strongestPostOp_{op_1}(v) ... ))$.
We use strongest-post operators during interpolation and refinement to evaluate paths.

The strongest-post operator $\strongestPostOpExplicit_{op}$ is defined in the following way:
%\begin{enumerate}[label=\alph*)]
%\item
For an assignment operation $s \assign exp$, $\strongestPostOpExplicit_{s \assign exp}(v) = v_\restrictedTo{X \setminus \{ s \}} \logicAnd v_{s \assign exp}$ with $v_{s \assign exp} = \{ (s, exp_\using{v}) \}$ and $exp_\using{v}$ denoting the evaluation of $exp$ using the abstract variable assignment $v$, as defined in Section \ref{sec:valueAnalysis}.
%\item
For an assume operation $\assume(p)$, 
	$\strongestPostOpExplicit_{\assume(p)}(v) = v'$ with 
	\[ v'(x) = \begin{dcases}
		\bot & \text{ if } \exists y \in \defRange(v) : v(y) = \bot \text{ or } p_\using{v} \text{ is unsatisfiable}\\
		c & \text{ if $c$ is the only satisfying assignment of $p_\using{v}$ for $x$}\\
		v(x) & \text{ if none of the above and } x \in \defRange(v)
	\end{dcases}\]
	with $p_\using{v}$ as defined in Section \ref{sec:valueAnalysis}.
%\end{enumerate}

%\subsubsection{Interpolation for abstract variable assignments}
\begin{algorithm}[t]
\caption{$\interpolateExplicit(\prefix, \suffix)$, adapted from \cite{Beyer2013}}
\label{alg:interpolateExplicit}
\begin{algorithmic}[1]
\Input two constraint sequences $\prefix$ and $\suffix$, with $\prefix \logicAnd \suffix$ contradicting
\Output a constraint sequence $\Gamma$, which is an interpolant for $\prefix$ and $\suffix$
\Variables an abstract variable assignment $v$

\State $v \assign \strongestPostOpExplicit_\prefix(\varnothing)$
\ForAll{$x \in \defRange(v)$}
	\If{$\strongestPostOpExplicit_\suffix(v_\restrictedTo{\defRange(v) \setminus \{x\}})$ is contradicting}
		\State $v \assign v_\restrictedTo{\defRange(v) \setminus \{x\}}$ \Comment $x$ not relevant, should not occur in interpolant
	\EndIf
\EndFor
\State $\Gamma \assign \langle \rangle$
\ForAll{$x \in \defRange(v)$} \label{alg:interpolateExplicit:itpStart}
	\State $\Gamma \assign \Gamma \logicAnd \langle \assume(x = v(x))\rangle$
\EndFor\\ \label{alg:interpolateExplicit:itpFinish}
\Return $\Gamma$
\end{algorithmic}
\end{algorithm}

The algorithm for interpolation in the domain of abstract variable assignments is shown in Algorithm \ref{alg:interpolateExplicit}.
For a prefix $\prefix$ and a suffix $\prefix$, the abstract variable assignment $v$, that results from applying $\prefix$ to the initial abstract variable assignment $\varnothing$ is computed.
Next, for each variable assignment in $v$ it is checked whether the assignment is necessary to prove that $\suffix$ is contradicting.
If it is not, it can be removed from $v$.
After all variable assignments are checked, $v$ only contains variable assignments that are necessary to prove that $\suffix$ is contradicting.
From these, the interpolant is built (Lines \ref{alg:interpolateExplicit:itpStart} - \ref{alg:interpolateExplicit:itpFinish}).

\begin{algorithm}[t]
\caption{$\refineExplicit{\sigma}$, adapted from \cite{Beyer2015}}
\label{alg:refinementExplicit}
\begin{algorithmic}[1]
\Input infeasible error path $\sigma = \langle (op_1, l_1), ..., (op_n, l_n) \rangle$
\Output precision $\pi$
\Variables interpolating constraint sequence $\Gamma$
\State $\Gamma \assign \langle \rangle$
\State $\pi(l) \assign \varnothing$ for all program locations $l$
\For{$i \assign 1$ to $n - 1$}\label{alg:refinementExplicit:loopStart}
	\State $\suffix \assign \langle op_{i+1}, ..., op_n \rangle$
	\State $\Gamma \assign \interpolateExplicit(\Gamma \logicAnd \langle op_i \rangle, \suffix)$ \Comment inductive interpolation \label{alg:refinementExplicit:interpolation}
	\State $\pi(l_i) \assign \extractPrecision{\Gamma}$
\EndFor\\
\Return $\pi$
\end{algorithmic}
\end{algorithm}

The interpolants produced are used in the refinement of the precision (Alg. \ref{alg:refinementExplicit}).
We use a location-specific precision $\pi : L \rightarrow 2^X$ that returns for a program location $l \in L$ all program variables of $X$ which are relevant for the analysis at this location. This approach realizes the lazy abstraction technique \cite{Henzinger2002}.
The algorithm starts with an initial, empty interpolant $\Gamma$ and empty precision $\pi$ with $\pi(l) = \varnothing$ for all $l \in L$.
For each location $(l_i, op_i)$ on the error path, the suffix $\suffix$ of this location are set and the interpolant is computed inductively from the existing interpolant in conjunction with the current operation $op_i$ and the suffix (Line \ref{alg:refinementExplicit:interpolation}).
A precision for the current program location is then extracted from the interpolant.
One straightforward way to do this is by using all program variables with a valid assignment in the  abstract variable assignment resulting from the application of the strongest-post operator to our interpolant:
\[\extractPrecision{\Gamma} = \{ x |\ (x, z) \in \strongestPostOpExplicit_\Gamma (\varnothing ) \text{ and } z \neq \bot_\valueset \}.\]
It is not only sufficient, but also required to use $\Gamma \logicAnd \langle op_i \rangle$ instead of the full prefix $\prefix = \langle op_1, ..., op_1 \rangle$ for interpolation. The full prefix cannot be used as it has to be assured that the precision resulting from these consecutive interpolations proves the error path infeasible. All information necessary for proving the infeasibility of the remaining error path is present in the current interpolant and operation.

This refinement procedure can be used in CEGAR (Alg. \ref{alg:cegar}) in combination with a CPA with precision adjustment that expects these precision types, like the \valueAnalysisCPA\ in combination with refinement for abstract variable assignments.

%\subsubsection{Refinement for the domain of linear arithmetics}
Refinement in the domain of linear arithmetics, as used for the \predicateCPA, uses a standard approach to refinement based on lazy abstraction and Craig interpolation.
The task of interpolation is delegated to an off-the-shelf SMT solver.

%\subsubsection{\ValueAnalysisCPA\ with precision adjustment}
%The \valueAnalysisCPA\ with dynamic precision adjustment \cite{Beyer2013} \[\valCPAPlus = (D_\valCPA, \Pi_\valCPAPlus, \transfer_\valCPAPlus, \cpaMerge^{sep}, \cpaStop^{sep}, %\cpaPrec_\valCPAPlus)\] is a CPA that can be, and is, used with the refinement for abstract variable assignments as described above.
%It consists of:
%\begin{enumerate}[leftmargin=*, label=\arabic*.]
%\item The abstract domain $D_\valCPA$ as defined in Section \ref{sec:valueAnalysis}.
%\item The set of precisions $\Pi_\valCPAPlus = L \rightarrow 2^X$. A precision $\pi \in \Pi_\valCPAPlus$ specifies a subset of program variables of $X$ that are tracked.
%\item The transfer relation $\transfer_\valCPAPlus$ contains the transfer $v \transfer_\valCPAPlus (v', \pi)$ if $v \transfer_\valCPA v'$.
%\item The merge operator $\cpaMerge^{sep}$ that performs no merging.
%\item The termination check $\cpaStop^{sep}$ that checks every state individually.
%\item The precision adjustment $\cpaPrec_\valCPAPlus$. Given an abstract state $v$ and a precision $\pi$, all abstract assignments of variables that do not occur in $\pi$ are removed from %$v$. This is done by restricting the partial function: $\cpaPrec_\valCPAPlus(v, \pi) = (v_\restrictedTo{\pi}, \pi)$. The given precision is returned as it is.
%\end{enumerate}

In this chapter, we gave an overview of all theoretical concepts that are necessary to describe our own work. We introduced the concept of configurable software verification and configurable program analyses (CPAs), a very versatile approach to automated software verification. We introduced different CPAs we use in this work and CEGAR with precision refinement for both linear arithmetics and abstract variable assignments, which we will use when applying CEGAR to the \symbolicExecutionCPA.

%\subsection{Bounded loops with continuation after reached bound}
%\subsubsection{Idea}
%\subsubsection{Existing LoopstackCPA}
%\subsubsection{No continuation after reached bound: Existing AssumptionStorageCPA}
%\subsubsection{New: Jump out of loop at every possible exit and abstract information}


%%%%%%%%%%%%%%%%%%%%%%%%
%\section{Definitions of newly introduced concepts}
\section{A different merge operator for \constraintsCPA}
\label{sec:newMerge}
For every operation $\assume(p)$ at a location $l$ that transfers the control flow to a location $l'$ there exists another operation $\assume(\neg p)$ at the same location transfering the control flow to a location $l'' \neq l'$.
In most programs it is probable that the two different program branches starting at $l'$ and $l''$ meet again, that means that for a later program location $l'''$ two abstract states $a, a'$ of the \constraintsCPA\ (in the following  called \emph{constraints states}) exist with $a$ containing $p$ and $a'$ containing $\neg p$.

If a constraint $p$ is part of an abstract state $a$, $p$ is true in all concrete states represented by $a$ (just like a predicate in an abstract state of the \predicateCPA\ \cite{Beyer2008}).
If for one program location $l$ two constraints states $a, a'$ exist with $p \in a$ and $\neg p \in a'$ and $a \setminus \{ p \} = a' \setminus \{ \neg p \}$,
then $a$ represents all concrete states for which $p \logicAnd a \setminus \{ p \}$ is true and $a'$ represents all concrete states for which $\neg p \logicAnd a \setminus \{ p \}$ is true.
At this point, the analysis will never be able to prove a program location as infeasible because of $p$ or $\neg p$.
If $a'$ reaches a program location and computes it as infeasible by using $p$, the abstract state $a$ will compute the same program location as feasible, if it reaches it.
Because of this, it seems legit to delete these obsolete constraints and only continue with one more abstract state instead of two more concrete ones by using the merge operator
\[ \cpaMerge (a, a', \pi) = \begin{dcases}
a' \setminus \neg Q & \text{ if } a \lesserEqual a' \setminus \neg Q \\
a' & \text{ otherwise}
\end{dcases} \]
with $\neg Q = \{\neg p |\ p \in a \logicAnd \neg p \in a' \}$ and $Q = \{ p |\ p \in a \logicAnd \neg p \in a' \}$.
It is not necessary that $a' \setminus \neg Q = a \setminus Q$.
If $a' \setminus \neg Q$ represents a super set of the concrete states represented by $a \setminus Q$, that is $a \setminus Q \lesserEqual a' \setminus \neg Q$, then the above condition is true, and $a \lesserEqual a \setminus Q$.

This condition is automatically checked by the $\mergeAgree$ operator, so we can simply use $\cpaMerge(a, a', \pi) = a' \setminus \neg Q$.

\section{Different Less-or-equal Operators}
%\subsubsection{Subset operator}
\label{sec:leqOperators}
The less-or-equal operator is the operator executed the most often during analyses as $\cpaStop^{sep}$ uses it once for every state in the reached set, at every iteration of the CPA algorithm.
In addition, it is responsible for determining whether a new state is already covered and analysis can be stopped at this point.
Although the implementation framework \cpaChecker\ only performs a termination check for reached states at the same location, 
its speed and precision can make a great difference for the performance of our analysis.

\paragraph*{Aliasing operator}
The  less-or-equal operators we used for \symbolicValueAnalysisCPA\ and \constraintsCPA\ in \cite{Lemberger2015} using an \aliasFunc\ function try to be more precise than a simple subset check.
Unfortunately, they can result in false behaviour because of their independent behaviour.
Consider the two pairs of value state and constraint state
$e = (v, a)$ with $v = \{x \rightarrow s1, y \rightarrow s2\}$, $a = \{s1 > 0\})$ and
$e' = (v', a')$ with $v' = \{x \rightarrow s2, y \rightarrow s1\}$, $a' = \{s1 > 0\})$.
When using the aliasing less-or-equal operators of the \symbolicValueAnalysisCPA\ and of the \constraintsCPA,
the \symbolicValueAnalysisCPA\ states
$v \lesserEqual v'$ for $\aliasFunc$ function $\aliasFunc(s1) = s2$, $\aliasFunc(s2) = s1$ and
the \constraintsCPA\ states
$a \lesserEqual a'$ for $\aliasFunc(s1) = s1$.
Because of this, $e \lesserEqual e'$, although the concrete states
$\llbracket e \rrbracket = \{ c \in C |\ c(x) > 0 \}$ and
$\llbracket e' \rrbracket = \{ c \in C |\ c(y) > 0 \}$
represented by $e$ and $e'$ are two different sets.
This violates the definition of the less-or-equal operator for abstract domains (Section~\ref{sec:abstractState}).
For this example, the less-or-equal operator of the \constraintsCPA\ actually behaves like the subset operator, since $\aliasFunc$ represents the identity.
This shows that the less-or-equal operator of the \symbolicValueAnalysisCPA\ cannot be used, regardless of the operator used by the \constraintsCPA.
Besides the default less-or-equal operator for the \constraintsCPA\ 
presented in Section~\ref{sec:constraintsCPA}, another operator might prove useful.

%\subsubsection{Implication operator}
\paragraph*{Implication operator}
Since a \constraintsCPA 's abstract state $a$ is interpreted as the conjunction of its constraints $\varphi_a$, it seems fit to use implication as the less-or-equal operator.
Remember that $\llbracket a \rrbracket = \{ c \in C |\ c \satisfies \varphi_a \}$.
If a formula $\varphi_a$ implies a formula $\varphi_{a'}$ and $c$ satisfies $\varphi_a$, then $c$ also satisfies $\varphi_{a'}$.
Because of this 
\[\llbracket a \rrbracket = \{ c \in C |\ c \satisfies \varphi_a \} \subseteq \{ c \in C |\ c \satisfies \varphi_{a'} \} = \llbracket a' \rrbracket \text{ if } \varphi_a \Rightarrow \varphi_{a'}.\]
The less-or-equal operator for the \constraintsCPA\ using implication is defined as $a \lesserEqualImpl a'$ if $\varphi_a \Rightarrow \varphi_{a'}$.
This operator has a higher precision than $\lesserEqualSub$ but requires SAT checks, which are definitely worse in performance than merely checking whether one set is the subset of another.


\subsection{Location/Frequency of SAT checks}
\subsubsection{After every assume}
\subsubsection{At every target location only}

\subsection{Basic CEGAR and its algorithm}
\subsubsection{CEGAR and interpolation in general}
Counterexample-guided abstraction refinement (CEGAR) \cite{Clarke2003} is a technique to find an abstraction that contains as few information as possible while retaining the possibility to prove or disprove a program's correctness.
This technique can greatly reduce the number of abstract states in a program's analysis and is considered ''the most general and flexible for handling the state explosion problem,''\cite{Clarke2003}\ the major problem we are facing with our \symbolicExecutionCPA.

The technique starts analysis with a coarse abstraction and refines it based on counterexamples. A counterexample is a witness of a property violation.\cite{Beyer2013}
If no error path is found by the analysis, it terminates and reports that no property violation exists.
If an error path is found, it is checked whether the path is feasible (i.e. a possible program execution) by repeating the analysis with full precision.
If the path is feasible, the analysis terminates and reports the found property violation.
If the error path is infeasible it was only found because the abstraction is too coarse. As a consequence, the abstraction is refined using the error path.
After this, the analysis starts again, using the new abstraction.

Since the problem of finding the coarsest possible refinement of an abstraction based on an error path is NP-hard, \cite{Clarke2003}\ good heuristics have to be used to find good refinements.
Interpolation \cite{Henzinger2004}\ is one such technique in a boolean context that is used for refinement of both the \predicateCPA\ and \valueAnalysisCPA.

\subsubsection{CEGAR and interpolation in the context of configurable software verification}
\label{sec:cegarBasics}
To apply CEGAR and interpolation to configurable software verification, a simple modification has to be made to the CPA algorithm.
Instead of passing it an initial state $e_0$ and an initial precision $\pi_0$, we use an initial reached set $R_0$ and initial waitlist $W_0$ (Alg. \ref{alg:cpaPlus}).
This way we can control at which point the analysis continues after a refinement was performed.

\begin{algorithm}[t]
\caption{$CPA(\cpaPlus, R_0, W_0)$, adapted from \cite{Beyer2013}}
\label{alg:cpaPlus}
\begin{algorithmic}[1]

\Input a CPA $\cpaPlus = (D, \Pi, \transfer, \cpaMerge, \cpaStop, \cpaPrec)$,
	    a set $R_0 \subseteq (E \times \Pi)$ of initial states with their precision and
	    a subset $W_0 \subseteq R_0$ of frontier abstract states with their precision,
	    with $E$ being the set of elements of $D$
\Output a set of abstract states reachable from $R_0$ with their precision and
	   a subset of frontier abstract states with their precision
\Variables \reachedSet\ and \waitlistSet , both subsets of $E \times \Pi$
\State $\reachedSet \assign R_0$
\State $\waitlistSet \assign W_0$
\While{$\waitlistSet \neq \varnothing$} \Comment from here on the same as before
\State ...
\EndWhile
\end{algorithmic}
\end{algorithm}

\begin{algorithm}[t]
\caption{$CEGAR(\cpaPlus, e_0, \pi_0)$, adapted from \cite{Beyer2013}}
\label{alg:cegar}
\begin{algorithmic}[1]
\Input a CPA $\cpaPlus = (D, \Pi, \transfer, \cpaMerge, \cpaStop, \cpaPrec)$ with dynamic precision adjustment,
	an initial abstract state $e_0 \in E$ with precision $\pi_0 \in \Pi$,
	with $E$ denoting the set of elements of the semi-lattice of $D$
\Output the verification result \safe\ or \unsafe
\Variables the sets \reachedSet\ and \waitlistSet\ of elements of $E \times \Pi$,
	      an error path $\sigma = \langle (op_1, l_1), ..., (op_n, l_n) \rangle$\\

\State $\reachedSet \assign \{ (e_0, \pi_0) \}$
\State $\waitlistSet \assign \{ e_0, \pi_0 \}$
\State $\pi \assign \pi_0$
\While{true}
	\State $(\reachedSet, \waitlistSet) \assign CPA(\cpaPlus, \reachedSet, \waitlistSet)$
	\If{$\waitlistSet = \varnothing$}
		\Return \safe
	\Else
		\State $\sigma \assign \extractErrorPath{\reachedSet}$ \label{alg:cegar:extraction}
		\If{\isFeasible{$\sigma$}} \Comment error path feasible: report bug \label{alg:cegar:feasibilityCheck}
			\State % empty state for new line after if
			\Return \unsafe 
		\Else \Comment error path infeasible: refine and restart from the beginning
			\State $\pi \assign \pi \cup \refine{\sigma}$
			\State $\reachedSet \assign (e_0, \pi)$
			\State $\waitlistSet \assign (e_0, \pi)$ \label{alg:cegar:end}
		\EndIf
	\EndIf
\EndWhile
\end{algorithmic}
\end{algorithm}

Now that the CPA algorithm is able to use precisions created in a refinement procedure, we use it as a part of our complete CEGAR algorithm.
Algorithm \ref{alg:cegar} uses a CPA using dynamic precision adjustment $\cpaPlus$,
an initial state $e_0$
and an initial precision $\pi_0$
to compute whether a property violation exists.

First, the $CPA$ algorithm is used to compute a set of reached abstract states ($\reachedSet$) and a subset of this set that contains all reached abstract states that have not been handled yet ($\waitlistSet$).
If $\waitlistSet$ is empty, the $CPA$ algorithm has handled all reachable states without encountering any target state.
If this is the case, no property violation was found and the algorithm can return \safe.
Otherwise, an error path is extracted from the reached set.
If the error path is reported as feasible, a property violation exists or the algorithm is not able to prove that none exists. It returns \unsafe.
If the error path is infeasible, the current precision is too abstract.
It is refined based on the infeasible error path by using $\refineFunc : \Sigma \rightarrow \Pi$ with $\Sigma$ being the set of all error paths, so that it can prove its infeasibility.
After this, the reached set and waitlist are reset to their initial values and the algorithm repeats analysis with the refined precision.
It is important to notice that the return type of $\refineFunc$ has to be equal to the precision type $\Pi$ used in $\cpaPlus$.
Because of this, CPAs are not exchangeable without changing refinement, too, in general.

%%%%%%%%%%%%%%%%%%%%%%%%%%
%%% Refinement in general
%%%%%%%%%%%%%%%%%%%%%%%%%%
For refinement, the priorly mentioned technique of interpolation is used to determine a location-specific precision that is strong enough for the CPA algorithm with precision adjustment to prove that a given error path is infeasible.
A boolean formula $\craigItp$ is a Craig interpolant \cite{Craig1957}\ for two boolean formulas $\prefix$ (called prefix) and $\suffix$ (called suffix), if the following three conditions are fulfilled:
\begin{enumerate}[label=\alph*)]
\item The prefix implies $\craigItp$, that is $\prefix \Rightarrow \craigItp$.
\item $\craigItp$ contradicts the suffix, that means $\craigItp \logicAnd \suffix$ is contradicting.
\item $\craigItp$ only contains variables occurring in \emph{both} prefix and suffix.
\end{enumerate}
It is proven that such an interpolant always exists in the domain of abstract variable assignments \cite{Beyer2013} as well as in the theory of linear arithmetics \cite{Craig1957}.

%\subsubsection{Refinement for the domain of abstract variable assignments}
Our work is strongly based on the refinement technique for abstract variable assignments.
The strongest-post operator $\strongestPostOp_{op}$ describes the semantics of an operation $op \in Ops$.
It is the analogy to the transfer relation in the domain of CPAs.
It maps a region of concrete states, implied by an abstract variable assignment, to the region of all concrete states that can be reached by executing $op$.
The semantics of a path $\sigma = \langle (l_1, op_1), ..., (l_n, op_n) \rangle$ is defined as the consecutive application of the strongest-post operator to its constraint sequence $\gamma_\sigma = \langle op_1, ..., op_n \rangle$:
$\strongestPostOp_{\gamma_\sigma}(v) = \strongestPostOp_{op_n}(\strongestPostOp_{op_{n-1}} (...\ \strongestPostOp_{op_1}(v) ... ))$.
We use strongest-post operators during interpolation and refinement to evaluate paths.

The strongest-post operator $\strongestPostOpExplicit_{op}$ is defined in the following way:
%\begin{enumerate}[label=\alph*)]
%\item
For an assignment operation $s \assign exp$, $\strongestPostOpExplicit_{s \assign exp}(v) = v_\restrictedTo{X \setminus \{ s \}} \logicAnd v_{s \assign exp}$ with $v_{s \assign exp} = \{ (s, exp_\using{v}) \}$ and $exp_\using{v}$ denoting the evaluation of $exp$ using the abstract variable assignment $v$, as defined in Section \ref{sec:valueAnalysis}.
%\item
For an assume operation $\assume(p)$, 
	$\strongestPostOpExplicit_{\assume(p)}(v) = v'$ with 
	\[ v'(x) = \begin{dcases}
		\bot & \text{ if } \exists y \in \defRange(v) : v(y) = \bot \text{ or } p_\using{v} \text{ is unsatisfiable}\\
		c & \text{ if $c$ is the only satisfying assignment of $p_\using{v}$ for $x$}\\
		v(x) & \text{ if none of the above and } x \in \defRange(v)
	\end{dcases}\]
	with $p_\using{v}$ as defined in Section \ref{sec:valueAnalysis}.
%\end{enumerate}

%\subsubsection{Interpolation for abstract variable assignments}
\begin{algorithm}[t]
\caption{$\interpolateExplicit(\prefix, \suffix)$, adapted from \cite{Beyer2013}}
\label{alg:interpolateExplicit}
\begin{algorithmic}[1]
\Input two constraint sequences $\prefix$ and $\suffix$, with $\prefix \logicAnd \suffix$ contradicting
\Output a constraint sequence $\Gamma$, which is an interpolant for $\prefix$ and $\suffix$
\Variables an abstract variable assignment $v$

\State $v \assign \strongestPostOpExplicit_\prefix(\varnothing)$
\ForAll{$x \in \defRange(v)$}
	\If{$\strongestPostOpExplicit_\suffix(v_\restrictedTo{\defRange(v) \setminus \{x\}})$ is contradicting}
		\State $v \assign v_\restrictedTo{\defRange(v) \setminus \{x\}}$ \Comment $x$ not relevant, should not occur in interpolant
	\EndIf
\EndFor
\State $\Gamma \assign \langle \rangle$
\ForAll{$x \in \defRange(v)$} \label{alg:interpolateExplicit:itpStart}
	\State $\Gamma \assign \Gamma \logicAnd \langle \assume(x = v(x))\rangle$
\EndFor\\ \label{alg:interpolateExplicit:itpFinish}
\Return $\Gamma$
\end{algorithmic}
\end{algorithm}

The algorithm for interpolation in the domain of abstract variable assignments is shown in Algorithm \ref{alg:interpolateExplicit}.
For a prefix $\prefix$ and a suffix $\prefix$, the abstract variable assignment $v$, that results from applying $\prefix$ to the initial abstract variable assignment $\varnothing$ is computed.
Next, for each variable assignment in $v$ it is checked whether the assignment is necessary to prove that $\suffix$ is contradicting.
If it is not, it can be removed from $v$.
After all variable assignments are checked, $v$ only contains variable assignments that are necessary to prove that $\suffix$ is contradicting.
From these, the interpolant is built (Lines \ref{alg:interpolateExplicit:itpStart} - \ref{alg:interpolateExplicit:itpFinish}).

\begin{algorithm}[t]
\caption{$\refineExplicit{\sigma}$, adapted from \cite{Beyer2015}}
\label{alg:refinementExplicit}
\begin{algorithmic}[1]
\Input infeasible error path $\sigma = \langle (op_1, l_1), ..., (op_n, l_n) \rangle$
\Output precision $\pi$
\Variables interpolating constraint sequence $\Gamma$
\State $\Gamma \assign \langle \rangle$
\State $\pi(l) \assign \varnothing$ for all program locations $l$
\For{$i \assign 1$ to $n - 1$}\label{alg:refinementExplicit:loopStart}
	\State $\suffix \assign \langle op_{i+1}, ..., op_n \rangle$
	\State $\Gamma \assign \interpolateExplicit(\Gamma \logicAnd \langle op_i \rangle, \suffix)$ \Comment inductive interpolation \label{alg:refinementExplicit:interpolation}
	\State $\pi(l_i) \assign \extractPrecision{\Gamma}$
\EndFor\\
\Return $\pi$
\end{algorithmic}
\end{algorithm}

The interpolants produced are used in the refinement of the precision (Alg. \ref{alg:refinementExplicit}).
We use a location-specific precision $\pi : L \rightarrow 2^X$ that returns for a program location $l \in L$ all program variables of $X$ which are relevant for the analysis at this location. This approach realizes the lazy abstraction technique \cite{Henzinger2002}.
The algorithm starts with an initial, empty interpolant $\Gamma$ and empty precision $\pi$ with $\pi(l) = \varnothing$ for all $l \in L$.
For each location $(l_i, op_i)$ on the error path, the suffix $\suffix$ of this location are set and the interpolant is computed inductively from the existing interpolant in conjunction with the current operation $op_i$ and the suffix (Line \ref{alg:refinementExplicit:interpolation}).
A precision for the current program location is then extracted from the interpolant.
One straightforward way to do this is by using all program variables with a valid assignment in the  abstract variable assignment resulting from the application of the strongest-post operator to our interpolant:
\[\extractPrecision{\Gamma} = \{ x |\ (x, z) \in \strongestPostOpExplicit_\Gamma (\varnothing ) \text{ and } z \neq \bot_\valueset \}.\]
It is not only sufficient, but also required to use $\Gamma \logicAnd \langle op_i \rangle$ instead of the full prefix $\prefix = \langle op_1, ..., op_1 \rangle$ for interpolation. The full prefix cannot be used as it has to be assured that the precision resulting from these consecutive interpolations proves the error path infeasible. All information necessary for proving the infeasibility of the remaining error path is present in the current interpolant and operation.

This refinement procedure can be used in CEGAR (Alg. \ref{alg:cegar}) in combination with a CPA with precision adjustment that expects these precision types, like the \valueAnalysisCPA\ in combination with refinement for abstract variable assignments.

%\subsubsection{Refinement for the domain of linear arithmetics}
Refinement in the domain of linear arithmetics, as used for the \predicateCPA, uses a standard approach to refinement based on lazy abstraction and Craig interpolation.
The task of interpolation is delegated to an off-the-shelf SMT solver.

%\subsubsection{\ValueAnalysisCPA\ with precision adjustment}
%The \valueAnalysisCPA\ with dynamic precision adjustment \cite{Beyer2013} \[\valCPAPlus = (D_\valCPA, \Pi_\valCPAPlus, \transfer_\valCPAPlus, \cpaMerge^{sep}, \cpaStop^{sep}, %\cpaPrec_\valCPAPlus)\] is a CPA that can be, and is, used with the refinement for abstract variable assignments as described above.
%It consists of:
%\begin{enumerate}[leftmargin=*, label=\arabic*.]
%\item The abstract domain $D_\valCPA$ as defined in Section \ref{sec:valueAnalysis}.
%\item The set of precisions $\Pi_\valCPAPlus = L \rightarrow 2^X$. A precision $\pi \in \Pi_\valCPAPlus$ specifies a subset of program variables of $X$ that are tracked.
%\item The transfer relation $\transfer_\valCPAPlus$ contains the transfer $v \transfer_\valCPAPlus (v', \pi)$ if $v \transfer_\valCPA v'$.
%\item The merge operator $\cpaMerge^{sep}$ that performs no merging.
%\item The termination check $\cpaStop^{sep}$ that checks every state individually.
%\item The precision adjustment $\cpaPrec_\valCPAPlus$. Given an abstract state $v$ and a precision $\pi$, all abstract assignments of variables that do not occur in $\pi$ are removed from %$v$. This is done by restricting the partial function: $\cpaPrec_\valCPAPlus(v, \pi) = (v_\restrictedTo{\pi}, \pi)$. The given precision is returned as it is.
%\end{enumerate}

In this chapter, we gave an overview of all theoretical concepts that are necessary to describe our own work. We introduced the concept of configurable software verification and configurable program analyses (CPAs), a very versatile approach to automated software verification. We introduced different CPAs we use in this work and CEGAR with precision refinement for both linear arithmetics and abstract variable assignments, which we will use when applying CEGAR to the \symbolicExecutionCPA.

%\subsection{Bounded loops with continuation after reached bound}
%\subsubsection{Idea}
%\subsubsection{Existing LoopstackCPA}
%\subsubsection{No continuation after reached bound: Existing AssumptionStorageCPA}
%\subsubsection{New: Jump out of loop at every possible exit and abstract information}


%%%%%%%%%%%%%%%%%%%%%%%%%%%%%%%%
\section{Definitions of newly introduced concepts}
\section{A different merge operator for \constraintsCPA}
\label{sec:newMerge}
For every operation $\assume(p)$ at a location $l$ that transfers the control flow to a location $l'$ there exists another operation $\assume(\neg p)$ at the same location transfering the control flow to a location $l'' \neq l'$.
In most programs it is probable that the two different program branches starting at $l'$ and $l''$ meet again, that means that for a later program location $l'''$ two abstract states $a, a'$ of the \constraintsCPA\ (in the following  called \emph{constraints states}) exist with $a$ containing $p$ and $a'$ containing $\neg p$.

If a constraint $p$ is part of an abstract state $a$, $p$ is true in all concrete states represented by $a$ (just like a predicate in an abstract state of the \predicateCPA\ \cite{Beyer2008}).
If for one program location $l$ two constraints states $a, a'$ exist with $p \in a$ and $\neg p \in a'$ and $a \setminus \{ p \} = a' \setminus \{ \neg p \}$,
then $a$ represents all concrete states for which $p \logicAnd a \setminus \{ p \}$ is true and $a'$ represents all concrete states for which $\neg p \logicAnd a \setminus \{ p \}$ is true.
At this point, the analysis will never be able to prove a program location as infeasible because of $p$ or $\neg p$.
If $a'$ reaches a program location and computes it as infeasible by using $p$, the abstract state $a$ will compute the same program location as feasible, if it reaches it.
Because of this, it seems legit to delete these obsolete constraints and only continue with one more abstract state instead of two more concrete ones by using the merge operator
\[ \cpaMerge (a, a', \pi) = \begin{dcases}
a' \setminus \neg Q & \text{ if } a \lesserEqual a' \setminus \neg Q \\
a' & \text{ otherwise}
\end{dcases} \]
with $\neg Q = \{\neg p |\ p \in a \logicAnd \neg p \in a' \}$ and $Q = \{ p |\ p \in a \logicAnd \neg p \in a' \}$.
It is not necessary that $a' \setminus \neg Q = a \setminus Q$.
If $a' \setminus \neg Q$ represents a super set of the concrete states represented by $a \setminus Q$, that is $a \setminus Q \lesserEqual a' \setminus \neg Q$, then the above condition is true, and $a \lesserEqual a \setminus Q$.

This condition is automatically checked by the $\mergeAgree$ operator, so we can simply use $\cpaMerge(a, a', \pi) = a' \setminus \neg Q$.

\section{Different Less-or-equal Operators}
%\subsubsection{Subset operator}
\label{sec:leqOperators}
The less-or-equal operator is the operator executed the most often during analyses as $\cpaStop^{sep}$ uses it once for every state in the reached set, at every iteration of the CPA algorithm.
In addition, it is responsible for determining whether a new state is already covered and analysis can be stopped at this point.
Although the implementation framework \cpaChecker\ only performs a termination check for reached states at the same location, 
its speed and precision can make a great difference for the performance of our analysis.

\paragraph*{Aliasing operator}
The  less-or-equal operators we used for \symbolicValueAnalysisCPA\ and \constraintsCPA\ in \cite{Lemberger2015} using an \aliasFunc\ function try to be more precise than a simple subset check.
Unfortunately, they can result in false behaviour because of their independent behaviour.
Consider the two pairs of value state and constraint state
$e = (v, a)$ with $v = \{x \rightarrow s1, y \rightarrow s2\}$, $a = \{s1 > 0\})$ and
$e' = (v', a')$ with $v' = \{x \rightarrow s2, y \rightarrow s1\}$, $a' = \{s1 > 0\})$.
When using the aliasing less-or-equal operators of the \symbolicValueAnalysisCPA\ and of the \constraintsCPA,
the \symbolicValueAnalysisCPA\ states
$v \lesserEqual v'$ for $\aliasFunc$ function $\aliasFunc(s1) = s2$, $\aliasFunc(s2) = s1$ and
the \constraintsCPA\ states
$a \lesserEqual a'$ for $\aliasFunc(s1) = s1$.
Because of this, $e \lesserEqual e'$, although the concrete states
$\llbracket e \rrbracket = \{ c \in C |\ c(x) > 0 \}$ and
$\llbracket e' \rrbracket = \{ c \in C |\ c(y) > 0 \}$
represented by $e$ and $e'$ are two different sets.
This violates the definition of the less-or-equal operator for abstract domains (Section~\ref{sec:abstractState}).
For this example, the less-or-equal operator of the \constraintsCPA\ actually behaves like the subset operator, since $\aliasFunc$ represents the identity.
This shows that the less-or-equal operator of the \symbolicValueAnalysisCPA\ cannot be used, regardless of the operator used by the \constraintsCPA.
Besides the default less-or-equal operator for the \constraintsCPA\ 
presented in Section~\ref{sec:constraintsCPA}, another operator might prove useful.

%\subsubsection{Implication operator}
\paragraph*{Implication operator}
Since a \constraintsCPA 's abstract state $a$ is interpreted as the conjunction of its constraints $\varphi_a$, it seems fit to use implication as the less-or-equal operator.
Remember that $\llbracket a \rrbracket = \{ c \in C |\ c \satisfies \varphi_a \}$.
If a formula $\varphi_a$ implies a formula $\varphi_{a'}$ and $c$ satisfies $\varphi_a$, then $c$ also satisfies $\varphi_{a'}$.
Because of this 
\[\llbracket a \rrbracket = \{ c \in C |\ c \satisfies \varphi_a \} \subseteq \{ c \in C |\ c \satisfies \varphi_{a'} \} = \llbracket a' \rrbracket \text{ if } \varphi_a \Rightarrow \varphi_{a'}.\]
The less-or-equal operator for the \constraintsCPA\ using implication is defined as $a \lesserEqualImpl a'$ if $\varphi_a \Rightarrow \varphi_{a'}$.
This operator has a higher precision than $\lesserEqualSub$ but requires SAT checks, which are definitely worse in performance than merely checking whether one set is the subset of another.


\subsection{Location/Frequency of SAT checks}
\subsubsection{After every assume}
\subsubsection{At every target location only}

\subsection{Basic CEGAR and its algorithm}
\subsubsection{CEGAR and interpolation in general}
Counterexample-guided abstraction refinement (CEGAR) \cite{Clarke2003} is a technique to find an abstraction that contains as few information as possible while retaining the possibility to prove or disprove a program's correctness.
This technique can greatly reduce the number of abstract states in a program's analysis and is considered ''the most general and flexible for handling the state explosion problem,''\cite{Clarke2003}\ the major problem we are facing with our \symbolicExecutionCPA.

The technique starts analysis with a coarse abstraction and refines it based on counterexamples. A counterexample is a witness of a property violation.\cite{Beyer2013}
If no error path is found by the analysis, it terminates and reports that no property violation exists.
If an error path is found, it is checked whether the path is feasible (i.e. a possible program execution) by repeating the analysis with full precision.
If the path is feasible, the analysis terminates and reports the found property violation.
If the error path is infeasible it was only found because the abstraction is too coarse. As a consequence, the abstraction is refined using the error path.
After this, the analysis starts again, using the new abstraction.

Since the problem of finding the coarsest possible refinement of an abstraction based on an error path is NP-hard, \cite{Clarke2003}\ good heuristics have to be used to find good refinements.
Interpolation \cite{Henzinger2004}\ is one such technique in a boolean context that is used for refinement of both the \predicateCPA\ and \valueAnalysisCPA.

\subsubsection{CEGAR and interpolation in the context of configurable software verification}
\label{sec:cegarBasics}
To apply CEGAR and interpolation to configurable software verification, a simple modification has to be made to the CPA algorithm.
Instead of passing it an initial state $e_0$ and an initial precision $\pi_0$, we use an initial reached set $R_0$ and initial waitlist $W_0$ (Alg. \ref{alg:cpaPlus}).
This way we can control at which point the analysis continues after a refinement was performed.

\begin{algorithm}[t]
\caption{$CPA(\cpaPlus, R_0, W_0)$, adapted from \cite{Beyer2013}}
\label{alg:cpaPlus}
\begin{algorithmic}[1]

\Input a CPA $\cpaPlus = (D, \Pi, \transfer, \cpaMerge, \cpaStop, \cpaPrec)$,
	    a set $R_0 \subseteq (E \times \Pi)$ of initial states with their precision and
	    a subset $W_0 \subseteq R_0$ of frontier abstract states with their precision,
	    with $E$ being the set of elements of $D$
\Output a set of abstract states reachable from $R_0$ with their precision and
	   a subset of frontier abstract states with their precision
\Variables \reachedSet\ and \waitlistSet , both subsets of $E \times \Pi$
\State $\reachedSet \assign R_0$
\State $\waitlistSet \assign W_0$
\While{$\waitlistSet \neq \varnothing$} \Comment from here on the same as before
\State ...
\EndWhile
\end{algorithmic}
\end{algorithm}

\begin{algorithm}[t]
\caption{$CEGAR(\cpaPlus, e_0, \pi_0)$, adapted from \cite{Beyer2013}}
\label{alg:cegar}
\begin{algorithmic}[1]
\Input a CPA $\cpaPlus = (D, \Pi, \transfer, \cpaMerge, \cpaStop, \cpaPrec)$ with dynamic precision adjustment,
	an initial abstract state $e_0 \in E$ with precision $\pi_0 \in \Pi$,
	with $E$ denoting the set of elements of the semi-lattice of $D$
\Output the verification result \safe\ or \unsafe
\Variables the sets \reachedSet\ and \waitlistSet\ of elements of $E \times \Pi$,
	      an error path $\sigma = \langle (op_1, l_1), ..., (op_n, l_n) \rangle$\\

\State $\reachedSet \assign \{ (e_0, \pi_0) \}$
\State $\waitlistSet \assign \{ e_0, \pi_0 \}$
\State $\pi \assign \pi_0$
\While{true}
	\State $(\reachedSet, \waitlistSet) \assign CPA(\cpaPlus, \reachedSet, \waitlistSet)$
	\If{$\waitlistSet = \varnothing$}
		\Return \safe
	\Else
		\State $\sigma \assign \extractErrorPath{\reachedSet}$ \label{alg:cegar:extraction}
		\If{\isFeasible{$\sigma$}} \Comment error path feasible: report bug \label{alg:cegar:feasibilityCheck}
			\State % empty state for new line after if
			\Return \unsafe 
		\Else \Comment error path infeasible: refine and restart from the beginning
			\State $\pi \assign \pi \cup \refine{\sigma}$
			\State $\reachedSet \assign (e_0, \pi)$
			\State $\waitlistSet \assign (e_0, \pi)$ \label{alg:cegar:end}
		\EndIf
	\EndIf
\EndWhile
\end{algorithmic}
\end{algorithm}

Now that the CPA algorithm is able to use precisions created in a refinement procedure, we use it as a part of our complete CEGAR algorithm.
Algorithm \ref{alg:cegar} uses a CPA using dynamic precision adjustment $\cpaPlus$,
an initial state $e_0$
and an initial precision $\pi_0$
to compute whether a property violation exists.

First, the $CPA$ algorithm is used to compute a set of reached abstract states ($\reachedSet$) and a subset of this set that contains all reached abstract states that have not been handled yet ($\waitlistSet$).
If $\waitlistSet$ is empty, the $CPA$ algorithm has handled all reachable states without encountering any target state.
If this is the case, no property violation was found and the algorithm can return \safe.
Otherwise, an error path is extracted from the reached set.
If the error path is reported as feasible, a property violation exists or the algorithm is not able to prove that none exists. It returns \unsafe.
If the error path is infeasible, the current precision is too abstract.
It is refined based on the infeasible error path by using $\refineFunc : \Sigma \rightarrow \Pi$ with $\Sigma$ being the set of all error paths, so that it can prove its infeasibility.
After this, the reached set and waitlist are reset to their initial values and the algorithm repeats analysis with the refined precision.
It is important to notice that the return type of $\refineFunc$ has to be equal to the precision type $\Pi$ used in $\cpaPlus$.
Because of this, CPAs are not exchangeable without changing refinement, too, in general.

%%%%%%%%%%%%%%%%%%%%%%%%%%
%%% Refinement in general
%%%%%%%%%%%%%%%%%%%%%%%%%%
For refinement, the priorly mentioned technique of interpolation is used to determine a location-specific precision that is strong enough for the CPA algorithm with precision adjustment to prove that a given error path is infeasible.
A boolean formula $\craigItp$ is a Craig interpolant \cite{Craig1957}\ for two boolean formulas $\prefix$ (called prefix) and $\suffix$ (called suffix), if the following three conditions are fulfilled:
\begin{enumerate}[label=\alph*)]
\item The prefix implies $\craigItp$, that is $\prefix \Rightarrow \craigItp$.
\item $\craigItp$ contradicts the suffix, that means $\craigItp \logicAnd \suffix$ is contradicting.
\item $\craigItp$ only contains variables occurring in \emph{both} prefix and suffix.
\end{enumerate}
It is proven that such an interpolant always exists in the domain of abstract variable assignments \cite{Beyer2013} as well as in the theory of linear arithmetics \cite{Craig1957}.

%\subsubsection{Refinement for the domain of abstract variable assignments}
Our work is strongly based on the refinement technique for abstract variable assignments.
The strongest-post operator $\strongestPostOp_{op}$ describes the semantics of an operation $op \in Ops$.
It is the analogy to the transfer relation in the domain of CPAs.
It maps a region of concrete states, implied by an abstract variable assignment, to the region of all concrete states that can be reached by executing $op$.
The semantics of a path $\sigma = \langle (l_1, op_1), ..., (l_n, op_n) \rangle$ is defined as the consecutive application of the strongest-post operator to its constraint sequence $\gamma_\sigma = \langle op_1, ..., op_n \rangle$:
$\strongestPostOp_{\gamma_\sigma}(v) = \strongestPostOp_{op_n}(\strongestPostOp_{op_{n-1}} (...\ \strongestPostOp_{op_1}(v) ... ))$.
We use strongest-post operators during interpolation and refinement to evaluate paths.

The strongest-post operator $\strongestPostOpExplicit_{op}$ is defined in the following way:
%\begin{enumerate}[label=\alph*)]
%\item
For an assignment operation $s \assign exp$, $\strongestPostOpExplicit_{s \assign exp}(v) = v_\restrictedTo{X \setminus \{ s \}} \logicAnd v_{s \assign exp}$ with $v_{s \assign exp} = \{ (s, exp_\using{v}) \}$ and $exp_\using{v}$ denoting the evaluation of $exp$ using the abstract variable assignment $v$, as defined in Section \ref{sec:valueAnalysis}.
%\item
For an assume operation $\assume(p)$, 
	$\strongestPostOpExplicit_{\assume(p)}(v) = v'$ with 
	\[ v'(x) = \begin{dcases}
		\bot & \text{ if } \exists y \in \defRange(v) : v(y) = \bot \text{ or } p_\using{v} \text{ is unsatisfiable}\\
		c & \text{ if $c$ is the only satisfying assignment of $p_\using{v}$ for $x$}\\
		v(x) & \text{ if none of the above and } x \in \defRange(v)
	\end{dcases}\]
	with $p_\using{v}$ as defined in Section \ref{sec:valueAnalysis}.
%\end{enumerate}

%\subsubsection{Interpolation for abstract variable assignments}
\begin{algorithm}[t]
\caption{$\interpolateExplicit(\prefix, \suffix)$, adapted from \cite{Beyer2013}}
\label{alg:interpolateExplicit}
\begin{algorithmic}[1]
\Input two constraint sequences $\prefix$ and $\suffix$, with $\prefix \logicAnd \suffix$ contradicting
\Output a constraint sequence $\Gamma$, which is an interpolant for $\prefix$ and $\suffix$
\Variables an abstract variable assignment $v$

\State $v \assign \strongestPostOpExplicit_\prefix(\varnothing)$
\ForAll{$x \in \defRange(v)$}
	\If{$\strongestPostOpExplicit_\suffix(v_\restrictedTo{\defRange(v) \setminus \{x\}})$ is contradicting}
		\State $v \assign v_\restrictedTo{\defRange(v) \setminus \{x\}}$ \Comment $x$ not relevant, should not occur in interpolant
	\EndIf
\EndFor
\State $\Gamma \assign \langle \rangle$
\ForAll{$x \in \defRange(v)$} \label{alg:interpolateExplicit:itpStart}
	\State $\Gamma \assign \Gamma \logicAnd \langle \assume(x = v(x))\rangle$
\EndFor\\ \label{alg:interpolateExplicit:itpFinish}
\Return $\Gamma$
\end{algorithmic}
\end{algorithm}

The algorithm for interpolation in the domain of abstract variable assignments is shown in Algorithm \ref{alg:interpolateExplicit}.
For a prefix $\prefix$ and a suffix $\prefix$, the abstract variable assignment $v$, that results from applying $\prefix$ to the initial abstract variable assignment $\varnothing$ is computed.
Next, for each variable assignment in $v$ it is checked whether the assignment is necessary to prove that $\suffix$ is contradicting.
If it is not, it can be removed from $v$.
After all variable assignments are checked, $v$ only contains variable assignments that are necessary to prove that $\suffix$ is contradicting.
From these, the interpolant is built (Lines \ref{alg:interpolateExplicit:itpStart} - \ref{alg:interpolateExplicit:itpFinish}).

\begin{algorithm}[t]
\caption{$\refineExplicit{\sigma}$, adapted from \cite{Beyer2015}}
\label{alg:refinementExplicit}
\begin{algorithmic}[1]
\Input infeasible error path $\sigma = \langle (op_1, l_1), ..., (op_n, l_n) \rangle$
\Output precision $\pi$
\Variables interpolating constraint sequence $\Gamma$
\State $\Gamma \assign \langle \rangle$
\State $\pi(l) \assign \varnothing$ for all program locations $l$
\For{$i \assign 1$ to $n - 1$}\label{alg:refinementExplicit:loopStart}
	\State $\suffix \assign \langle op_{i+1}, ..., op_n \rangle$
	\State $\Gamma \assign \interpolateExplicit(\Gamma \logicAnd \langle op_i \rangle, \suffix)$ \Comment inductive interpolation \label{alg:refinementExplicit:interpolation}
	\State $\pi(l_i) \assign \extractPrecision{\Gamma}$
\EndFor\\
\Return $\pi$
\end{algorithmic}
\end{algorithm}

The interpolants produced are used in the refinement of the precision (Alg. \ref{alg:refinementExplicit}).
We use a location-specific precision $\pi : L \rightarrow 2^X$ that returns for a program location $l \in L$ all program variables of $X$ which are relevant for the analysis at this location. This approach realizes the lazy abstraction technique \cite{Henzinger2002}.
The algorithm starts with an initial, empty interpolant $\Gamma$ and empty precision $\pi$ with $\pi(l) = \varnothing$ for all $l \in L$.
For each location $(l_i, op_i)$ on the error path, the suffix $\suffix$ of this location are set and the interpolant is computed inductively from the existing interpolant in conjunction with the current operation $op_i$ and the suffix (Line \ref{alg:refinementExplicit:interpolation}).
A precision for the current program location is then extracted from the interpolant.
One straightforward way to do this is by using all program variables with a valid assignment in the  abstract variable assignment resulting from the application of the strongest-post operator to our interpolant:
\[\extractPrecision{\Gamma} = \{ x |\ (x, z) \in \strongestPostOpExplicit_\Gamma (\varnothing ) \text{ and } z \neq \bot_\valueset \}.\]
It is not only sufficient, but also required to use $\Gamma \logicAnd \langle op_i \rangle$ instead of the full prefix $\prefix = \langle op_1, ..., op_1 \rangle$ for interpolation. The full prefix cannot be used as it has to be assured that the precision resulting from these consecutive interpolations proves the error path infeasible. All information necessary for proving the infeasibility of the remaining error path is present in the current interpolant and operation.

This refinement procedure can be used in CEGAR (Alg. \ref{alg:cegar}) in combination with a CPA with precision adjustment that expects these precision types, like the \valueAnalysisCPA\ in combination with refinement for abstract variable assignments.

%\subsubsection{Refinement for the domain of linear arithmetics}
Refinement in the domain of linear arithmetics, as used for the \predicateCPA, uses a standard approach to refinement based on lazy abstraction and Craig interpolation.
The task of interpolation is delegated to an off-the-shelf SMT solver.

%\subsubsection{\ValueAnalysisCPA\ with precision adjustment}
%The \valueAnalysisCPA\ with dynamic precision adjustment \cite{Beyer2013} \[\valCPAPlus = (D_\valCPA, \Pi_\valCPAPlus, \transfer_\valCPAPlus, \cpaMerge^{sep}, \cpaStop^{sep}, %\cpaPrec_\valCPAPlus)\] is a CPA that can be, and is, used with the refinement for abstract variable assignments as described above.
%It consists of:
%\begin{enumerate}[leftmargin=*, label=\arabic*.]
%\item The abstract domain $D_\valCPA$ as defined in Section \ref{sec:valueAnalysis}.
%\item The set of precisions $\Pi_\valCPAPlus = L \rightarrow 2^X$. A precision $\pi \in \Pi_\valCPAPlus$ specifies a subset of program variables of $X$ that are tracked.
%\item The transfer relation $\transfer_\valCPAPlus$ contains the transfer $v \transfer_\valCPAPlus (v', \pi)$ if $v \transfer_\valCPA v'$.
%\item The merge operator $\cpaMerge^{sep}$ that performs no merging.
%\item The termination check $\cpaStop^{sep}$ that checks every state individually.
%\item The precision adjustment $\cpaPrec_\valCPAPlus$. Given an abstract state $v$ and a precision $\pi$, all abstract assignments of variables that do not occur in $\pi$ are removed from %$v$. This is done by restricting the partial function: $\cpaPrec_\valCPAPlus(v, \pi) = (v_\restrictedTo{\pi}, \pi)$. The given precision is returned as it is.
%\end{enumerate}

In this chapter, we gave an overview of all theoretical concepts that are necessary to describe our own work. We introduced the concept of configurable software verification and configurable program analyses (CPAs), a very versatile approach to automated software verification. We introduced different CPAs we use in this work and CEGAR with precision refinement for both linear arithmetics and abstract variable assignments, which we will use when applying CEGAR to the \symbolicExecutionCPA.

%\subsection{Bounded loops with continuation after reached bound}
%\subsubsection{Idea}
%\subsubsection{Existing LoopstackCPA}
%\subsubsection{No continuation after reached bound: Existing AssumptionStorageCPA}
%\subsubsection{New: Jump out of loop at every possible exit and abstract information}


%%%%%%%%%%%%%%%%%%%%%%%%%%%%%%%%
\section{Definitions of newly introduced concepts}
\section{A different merge operator for \constraintsCPA}
\label{sec:newMerge}
For every operation $\assume(p)$ at a location $l$ that transfers the control flow to a location $l'$ there exists another operation $\assume(\neg p)$ at the same location transfering the control flow to a location $l'' \neq l'$.
In most programs it is probable that the two different program branches starting at $l'$ and $l''$ meet again, that means that for a later program location $l'''$ two abstract states $a, a'$ of the \constraintsCPA\ (in the following  called \emph{constraints states}) exist with $a$ containing $p$ and $a'$ containing $\neg p$.

If a constraint $p$ is part of an abstract state $a$, $p$ is true in all concrete states represented by $a$ (just like a predicate in an abstract state of the \predicateCPA\ \cite{Beyer2008}).
If for one program location $l$ two constraints states $a, a'$ exist with $p \in a$ and $\neg p \in a'$ and $a \setminus \{ p \} = a' \setminus \{ \neg p \}$,
then $a$ represents all concrete states for which $p \logicAnd a \setminus \{ p \}$ is true and $a'$ represents all concrete states for which $\neg p \logicAnd a \setminus \{ p \}$ is true.
At this point, the analysis will never be able to prove a program location as infeasible because of $p$ or $\neg p$.
If $a'$ reaches a program location and computes it as infeasible by using $p$, the abstract state $a$ will compute the same program location as feasible, if it reaches it.
Because of this, it seems legit to delete these obsolete constraints and only continue with one more abstract state instead of two more concrete ones by using the merge operator
\[ \cpaMerge (a, a', \pi) = \begin{dcases}
a' \setminus \neg Q & \text{ if } a \lesserEqual a' \setminus \neg Q \\
a' & \text{ otherwise}
\end{dcases} \]
with $\neg Q = \{\neg p |\ p \in a \logicAnd \neg p \in a' \}$ and $Q = \{ p |\ p \in a \logicAnd \neg p \in a' \}$.
It is not necessary that $a' \setminus \neg Q = a \setminus Q$.
If $a' \setminus \neg Q$ represents a super set of the concrete states represented by $a \setminus Q$, that is $a \setminus Q \lesserEqual a' \setminus \neg Q$, then the above condition is true, and $a \lesserEqual a \setminus Q$.

This condition is automatically checked by the $\mergeAgree$ operator, so we can simply use $\cpaMerge(a, a', \pi) = a' \setminus \neg Q$.

\section{Different Less-or-equal Operators}
%\subsubsection{Subset operator}
\label{sec:leqOperators}
The less-or-equal operator is the operator executed the most often during analyses as $\cpaStop^{sep}$ uses it once for every state in the reached set, at every iteration of the CPA algorithm.
In addition, it is responsible for determining whether a new state is already covered and analysis can be stopped at this point.
Although the implementation framework \cpaChecker\ only performs a termination check for reached states at the same location, 
its speed and precision can make a great difference for the performance of our analysis.

\paragraph*{Aliasing operator}
The  less-or-equal operators we used for \symbolicValueAnalysisCPA\ and \constraintsCPA\ in \cite{Lemberger2015} using an \aliasFunc\ function try to be more precise than a simple subset check.
Unfortunately, they can result in false behaviour because of their independent behaviour.
Consider the two pairs of value state and constraint state
$e = (v, a)$ with $v = \{x \rightarrow s1, y \rightarrow s2\}$, $a = \{s1 > 0\})$ and
$e' = (v', a')$ with $v' = \{x \rightarrow s2, y \rightarrow s1\}$, $a' = \{s1 > 0\})$.
When using the aliasing less-or-equal operators of the \symbolicValueAnalysisCPA\ and of the \constraintsCPA,
the \symbolicValueAnalysisCPA\ states
$v \lesserEqual v'$ for $\aliasFunc$ function $\aliasFunc(s1) = s2$, $\aliasFunc(s2) = s1$ and
the \constraintsCPA\ states
$a \lesserEqual a'$ for $\aliasFunc(s1) = s1$.
Because of this, $e \lesserEqual e'$, although the concrete states
$\llbracket e \rrbracket = \{ c \in C |\ c(x) > 0 \}$ and
$\llbracket e' \rrbracket = \{ c \in C |\ c(y) > 0 \}$
represented by $e$ and $e'$ are two different sets.
This violates the definition of the less-or-equal operator for abstract domains (Section~\ref{sec:abstractState}).
For this example, the less-or-equal operator of the \constraintsCPA\ actually behaves like the subset operator, since $\aliasFunc$ represents the identity.
This shows that the less-or-equal operator of the \symbolicValueAnalysisCPA\ cannot be used, regardless of the operator used by the \constraintsCPA.
Besides the default less-or-equal operator for the \constraintsCPA\ 
presented in Section~\ref{sec:constraintsCPA}, another operator might prove useful.

%\subsubsection{Implication operator}
\paragraph*{Implication operator}
Since a \constraintsCPA 's abstract state $a$ is interpreted as the conjunction of its constraints $\varphi_a$, it seems fit to use implication as the less-or-equal operator.
Remember that $\llbracket a \rrbracket = \{ c \in C |\ c \satisfies \varphi_a \}$.
If a formula $\varphi_a$ implies a formula $\varphi_{a'}$ and $c$ satisfies $\varphi_a$, then $c$ also satisfies $\varphi_{a'}$.
Because of this 
\[\llbracket a \rrbracket = \{ c \in C |\ c \satisfies \varphi_a \} \subseteq \{ c \in C |\ c \satisfies \varphi_{a'} \} = \llbracket a' \rrbracket \text{ if } \varphi_a \Rightarrow \varphi_{a'}.\]
The less-or-equal operator for the \constraintsCPA\ using implication is defined as $a \lesserEqualImpl a'$ if $\varphi_a \Rightarrow \varphi_{a'}$.
This operator has a higher precision than $\lesserEqualSub$ but requires SAT checks, which are definitely worse in performance than merely checking whether one set is the subset of another.


\subsection{Location/Frequency of SAT checks}
\subsubsection{After every assume}
\subsubsection{At every target location only}

\subsection{Basic CEGAR and its algorithm}
\subsubsection{CEGAR and interpolation in general}
Counterexample-guided abstraction refinement (CEGAR) \cite{Clarke2003} is a technique to find an abstraction that contains as few information as possible while retaining the possibility to prove or disprove a program's correctness.
This technique can greatly reduce the number of abstract states in a program's analysis and is considered ''the most general and flexible for handling the state explosion problem,''\cite{Clarke2003}\ the major problem we are facing with our \symbolicExecutionCPA.

The technique starts analysis with a coarse abstraction and refines it based on counterexamples. A counterexample is a witness of a property violation.\cite{Beyer2013}
If no error path is found by the analysis, it terminates and reports that no property violation exists.
If an error path is found, it is checked whether the path is feasible (i.e. a possible program execution) by repeating the analysis with full precision.
If the path is feasible, the analysis terminates and reports the found property violation.
If the error path is infeasible it was only found because the abstraction is too coarse. As a consequence, the abstraction is refined using the error path.
After this, the analysis starts again, using the new abstraction.

Since the problem of finding the coarsest possible refinement of an abstraction based on an error path is NP-hard, \cite{Clarke2003}\ good heuristics have to be used to find good refinements.
Interpolation \cite{Henzinger2004}\ is one such technique in a boolean context that is used for refinement of both the \predicateCPA\ and \valueAnalysisCPA.

\subsubsection{CEGAR and interpolation in the context of configurable software verification}
\label{sec:cegarBasics}
To apply CEGAR and interpolation to configurable software verification, a simple modification has to be made to the CPA algorithm.
Instead of passing it an initial state $e_0$ and an initial precision $\pi_0$, we use an initial reached set $R_0$ and initial waitlist $W_0$ (Alg. \ref{alg:cpaPlus}).
This way we can control at which point the analysis continues after a refinement was performed.

\begin{algorithm}[t]
\caption{$CPA(\cpaPlus, R_0, W_0)$, adapted from \cite{Beyer2013}}
\label{alg:cpaPlus}
\begin{algorithmic}[1]

\Input a CPA $\cpaPlus = (D, \Pi, \transfer, \cpaMerge, \cpaStop, \cpaPrec)$,
	    a set $R_0 \subseteq (E \times \Pi)$ of initial states with their precision and
	    a subset $W_0 \subseteq R_0$ of frontier abstract states with their precision,
	    with $E$ being the set of elements of $D$
\Output a set of abstract states reachable from $R_0$ with their precision and
	   a subset of frontier abstract states with their precision
\Variables \reachedSet\ and \waitlistSet , both subsets of $E \times \Pi$
\State $\reachedSet \assign R_0$
\State $\waitlistSet \assign W_0$
\While{$\waitlistSet \neq \varnothing$} \Comment from here on the same as before
\State ...
\EndWhile
\end{algorithmic}
\end{algorithm}

\begin{algorithm}[t]
\caption{$CEGAR(\cpaPlus, e_0, \pi_0)$, adapted from \cite{Beyer2013}}
\label{alg:cegar}
\begin{algorithmic}[1]
\Input a CPA $\cpaPlus = (D, \Pi, \transfer, \cpaMerge, \cpaStop, \cpaPrec)$ with dynamic precision adjustment,
	an initial abstract state $e_0 \in E$ with precision $\pi_0 \in \Pi$,
	with $E$ denoting the set of elements of the semi-lattice of $D$
\Output the verification result \safe\ or \unsafe
\Variables the sets \reachedSet\ and \waitlistSet\ of elements of $E \times \Pi$,
	      an error path $\sigma = \langle (op_1, l_1), ..., (op_n, l_n) \rangle$\\

\State $\reachedSet \assign \{ (e_0, \pi_0) \}$
\State $\waitlistSet \assign \{ e_0, \pi_0 \}$
\State $\pi \assign \pi_0$
\While{true}
	\State $(\reachedSet, \waitlistSet) \assign CPA(\cpaPlus, \reachedSet, \waitlistSet)$
	\If{$\waitlistSet = \varnothing$}
		\Return \safe
	\Else
		\State $\sigma \assign \extractErrorPath{\reachedSet}$ \label{alg:cegar:extraction}
		\If{\isFeasible{$\sigma$}} \Comment error path feasible: report bug \label{alg:cegar:feasibilityCheck}
			\State % empty state for new line after if
			\Return \unsafe 
		\Else \Comment error path infeasible: refine and restart from the beginning
			\State $\pi \assign \pi \cup \refine{\sigma}$
			\State $\reachedSet \assign (e_0, \pi)$
			\State $\waitlistSet \assign (e_0, \pi)$ \label{alg:cegar:end}
		\EndIf
	\EndIf
\EndWhile
\end{algorithmic}
\end{algorithm}

Now that the CPA algorithm is able to use precisions created in a refinement procedure, we use it as a part of our complete CEGAR algorithm.
Algorithm \ref{alg:cegar} uses a CPA using dynamic precision adjustment $\cpaPlus$,
an initial state $e_0$
and an initial precision $\pi_0$
to compute whether a property violation exists.

First, the $CPA$ algorithm is used to compute a set of reached abstract states ($\reachedSet$) and a subset of this set that contains all reached abstract states that have not been handled yet ($\waitlistSet$).
If $\waitlistSet$ is empty, the $CPA$ algorithm has handled all reachable states without encountering any target state.
If this is the case, no property violation was found and the algorithm can return \safe.
Otherwise, an error path is extracted from the reached set.
If the error path is reported as feasible, a property violation exists or the algorithm is not able to prove that none exists. It returns \unsafe.
If the error path is infeasible, the current precision is too abstract.
It is refined based on the infeasible error path by using $\refineFunc : \Sigma \rightarrow \Pi$ with $\Sigma$ being the set of all error paths, so that it can prove its infeasibility.
After this, the reached set and waitlist are reset to their initial values and the algorithm repeats analysis with the refined precision.
It is important to notice that the return type of $\refineFunc$ has to be equal to the precision type $\Pi$ used in $\cpaPlus$.
Because of this, CPAs are not exchangeable without changing refinement, too, in general.

%%%%%%%%%%%%%%%%%%%%%%%%%%
%%% Refinement in general
%%%%%%%%%%%%%%%%%%%%%%%%%%
For refinement, the priorly mentioned technique of interpolation is used to determine a location-specific precision that is strong enough for the CPA algorithm with precision adjustment to prove that a given error path is infeasible.
A boolean formula $\craigItp$ is a Craig interpolant \cite{Craig1957}\ for two boolean formulas $\prefix$ (called prefix) and $\suffix$ (called suffix), if the following three conditions are fulfilled:
\begin{enumerate}[label=\alph*)]
\item The prefix implies $\craigItp$, that is $\prefix \Rightarrow \craigItp$.
\item $\craigItp$ contradicts the suffix, that means $\craigItp \logicAnd \suffix$ is contradicting.
\item $\craigItp$ only contains variables occurring in \emph{both} prefix and suffix.
\end{enumerate}
It is proven that such an interpolant always exists in the domain of abstract variable assignments \cite{Beyer2013} as well as in the theory of linear arithmetics \cite{Craig1957}.

%\subsubsection{Refinement for the domain of abstract variable assignments}
Our work is strongly based on the refinement technique for abstract variable assignments.
The strongest-post operator $\strongestPostOp_{op}$ describes the semantics of an operation $op \in Ops$.
It is the analogy to the transfer relation in the domain of CPAs.
It maps a region of concrete states, implied by an abstract variable assignment, to the region of all concrete states that can be reached by executing $op$.
The semantics of a path $\sigma = \langle (l_1, op_1), ..., (l_n, op_n) \rangle$ is defined as the consecutive application of the strongest-post operator to its constraint sequence $\gamma_\sigma = \langle op_1, ..., op_n \rangle$:
$\strongestPostOp_{\gamma_\sigma}(v) = \strongestPostOp_{op_n}(\strongestPostOp_{op_{n-1}} (...\ \strongestPostOp_{op_1}(v) ... ))$.
We use strongest-post operators during interpolation and refinement to evaluate paths.

The strongest-post operator $\strongestPostOpExplicit_{op}$ is defined in the following way:
%\begin{enumerate}[label=\alph*)]
%\item
For an assignment operation $s \assign exp$, $\strongestPostOpExplicit_{s \assign exp}(v) = v_\restrictedTo{X \setminus \{ s \}} \logicAnd v_{s \assign exp}$ with $v_{s \assign exp} = \{ (s, exp_\using{v}) \}$ and $exp_\using{v}$ denoting the evaluation of $exp$ using the abstract variable assignment $v$, as defined in Section \ref{sec:valueAnalysis}.
%\item
For an assume operation $\assume(p)$, 
	$\strongestPostOpExplicit_{\assume(p)}(v) = v'$ with 
	\[ v'(x) = \begin{dcases}
		\bot & \text{ if } \exists y \in \defRange(v) : v(y) = \bot \text{ or } p_\using{v} \text{ is unsatisfiable}\\
		c & \text{ if $c$ is the only satisfying assignment of $p_\using{v}$ for $x$}\\
		v(x) & \text{ if none of the above and } x \in \defRange(v)
	\end{dcases}\]
	with $p_\using{v}$ as defined in Section \ref{sec:valueAnalysis}.
%\end{enumerate}

%\subsubsection{Interpolation for abstract variable assignments}
\begin{algorithm}[t]
\caption{$\interpolateExplicit(\prefix, \suffix)$, adapted from \cite{Beyer2013}}
\label{alg:interpolateExplicit}
\begin{algorithmic}[1]
\Input two constraint sequences $\prefix$ and $\suffix$, with $\prefix \logicAnd \suffix$ contradicting
\Output a constraint sequence $\Gamma$, which is an interpolant for $\prefix$ and $\suffix$
\Variables an abstract variable assignment $v$

\State $v \assign \strongestPostOpExplicit_\prefix(\varnothing)$
\ForAll{$x \in \defRange(v)$}
	\If{$\strongestPostOpExplicit_\suffix(v_\restrictedTo{\defRange(v) \setminus \{x\}})$ is contradicting}
		\State $v \assign v_\restrictedTo{\defRange(v) \setminus \{x\}}$ \Comment $x$ not relevant, should not occur in interpolant
	\EndIf
\EndFor
\State $\Gamma \assign \langle \rangle$
\ForAll{$x \in \defRange(v)$} \label{alg:interpolateExplicit:itpStart}
	\State $\Gamma \assign \Gamma \logicAnd \langle \assume(x = v(x))\rangle$
\EndFor\\ \label{alg:interpolateExplicit:itpFinish}
\Return $\Gamma$
\end{algorithmic}
\end{algorithm}

The algorithm for interpolation in the domain of abstract variable assignments is shown in Algorithm \ref{alg:interpolateExplicit}.
For a prefix $\prefix$ and a suffix $\prefix$, the abstract variable assignment $v$, that results from applying $\prefix$ to the initial abstract variable assignment $\varnothing$ is computed.
Next, for each variable assignment in $v$ it is checked whether the assignment is necessary to prove that $\suffix$ is contradicting.
If it is not, it can be removed from $v$.
After all variable assignments are checked, $v$ only contains variable assignments that are necessary to prove that $\suffix$ is contradicting.
From these, the interpolant is built (Lines \ref{alg:interpolateExplicit:itpStart} - \ref{alg:interpolateExplicit:itpFinish}).

\begin{algorithm}[t]
\caption{$\refineExplicit{\sigma}$, adapted from \cite{Beyer2015}}
\label{alg:refinementExplicit}
\begin{algorithmic}[1]
\Input infeasible error path $\sigma = \langle (op_1, l_1), ..., (op_n, l_n) \rangle$
\Output precision $\pi$
\Variables interpolating constraint sequence $\Gamma$
\State $\Gamma \assign \langle \rangle$
\State $\pi(l) \assign \varnothing$ for all program locations $l$
\For{$i \assign 1$ to $n - 1$}\label{alg:refinementExplicit:loopStart}
	\State $\suffix \assign \langle op_{i+1}, ..., op_n \rangle$
	\State $\Gamma \assign \interpolateExplicit(\Gamma \logicAnd \langle op_i \rangle, \suffix)$ \Comment inductive interpolation \label{alg:refinementExplicit:interpolation}
	\State $\pi(l_i) \assign \extractPrecision{\Gamma}$
\EndFor\\
\Return $\pi$
\end{algorithmic}
\end{algorithm}

The interpolants produced are used in the refinement of the precision (Alg. \ref{alg:refinementExplicit}).
We use a location-specific precision $\pi : L \rightarrow 2^X$ that returns for a program location $l \in L$ all program variables of $X$ which are relevant for the analysis at this location. This approach realizes the lazy abstraction technique \cite{Henzinger2002}.
The algorithm starts with an initial, empty interpolant $\Gamma$ and empty precision $\pi$ with $\pi(l) = \varnothing$ for all $l \in L$.
For each location $(l_i, op_i)$ on the error path, the suffix $\suffix$ of this location are set and the interpolant is computed inductively from the existing interpolant in conjunction with the current operation $op_i$ and the suffix (Line \ref{alg:refinementExplicit:interpolation}).
A precision for the current program location is then extracted from the interpolant.
One straightforward way to do this is by using all program variables with a valid assignment in the  abstract variable assignment resulting from the application of the strongest-post operator to our interpolant:
\[\extractPrecision{\Gamma} = \{ x |\ (x, z) \in \strongestPostOpExplicit_\Gamma (\varnothing ) \text{ and } z \neq \bot_\valueset \}.\]
It is not only sufficient, but also required to use $\Gamma \logicAnd \langle op_i \rangle$ instead of the full prefix $\prefix = \langle op_1, ..., op_1 \rangle$ for interpolation. The full prefix cannot be used as it has to be assured that the precision resulting from these consecutive interpolations proves the error path infeasible. All information necessary for proving the infeasibility of the remaining error path is present in the current interpolant and operation.

This refinement procedure can be used in CEGAR (Alg. \ref{alg:cegar}) in combination with a CPA with precision adjustment that expects these precision types, like the \valueAnalysisCPA\ in combination with refinement for abstract variable assignments.

%\subsubsection{Refinement for the domain of linear arithmetics}
Refinement in the domain of linear arithmetics, as used for the \predicateCPA, uses a standard approach to refinement based on lazy abstraction and Craig interpolation.
The task of interpolation is delegated to an off-the-shelf SMT solver.

%\subsubsection{\ValueAnalysisCPA\ with precision adjustment}
%The \valueAnalysisCPA\ with dynamic precision adjustment \cite{Beyer2013} \[\valCPAPlus = (D_\valCPA, \Pi_\valCPAPlus, \transfer_\valCPAPlus, \cpaMerge^{sep}, \cpaStop^{sep}, %\cpaPrec_\valCPAPlus)\] is a CPA that can be, and is, used with the refinement for abstract variable assignments as described above.
%It consists of:
%\begin{enumerate}[leftmargin=*, label=\arabic*.]
%\item The abstract domain $D_\valCPA$ as defined in Section \ref{sec:valueAnalysis}.
%\item The set of precisions $\Pi_\valCPAPlus = L \rightarrow 2^X$. A precision $\pi \in \Pi_\valCPAPlus$ specifies a subset of program variables of $X$ that are tracked.
%\item The transfer relation $\transfer_\valCPAPlus$ contains the transfer $v \transfer_\valCPAPlus (v', \pi)$ if $v \transfer_\valCPA v'$.
%\item The merge operator $\cpaMerge^{sep}$ that performs no merging.
%\item The termination check $\cpaStop^{sep}$ that checks every state individually.
%\item The precision adjustment $\cpaPrec_\valCPAPlus$. Given an abstract state $v$ and a precision $\pi$, all abstract assignments of variables that do not occur in $\pi$ are removed from %$v$. This is done by restricting the partial function: $\cpaPrec_\valCPAPlus(v, \pi) = (v_\restrictedTo{\pi}, \pi)$. The given precision is returned as it is.
%\end{enumerate}

In this chapter, we gave an overview of all theoretical concepts that are necessary to describe our own work. We introduced the concept of configurable software verification and configurable program analyses (CPAs), a very versatile approach to automated software verification. We introduced different CPAs we use in this work and CEGAR with precision refinement for both linear arithmetics and abstract variable assignments, which we will use when applying CEGAR to the \symbolicExecutionCPA.

%\subsection{Bounded loops with continuation after reached bound}
%\subsubsection{Idea}
%\subsubsection{Existing LoopstackCPA}
%\subsubsection{No continuation after reached bound: Existing AssumptionStorageCPA}
%\subsubsection{New: Jump out of loop at every possible exit and abstract information}


%%%%%%%%%%%%%%%%%%%%%%%%%%%%%%%%
\section{Definitions of newly introduced concepts}
\section{A different merge operator for \constraintsCPA}
\label{sec:newMerge}
For every operation $\assume(p)$ at a location $l$ that transfers the control flow to a location $l'$ there exists another operation $\assume(\neg p)$ at the same location transfering the control flow to a location $l'' \neq l'$.
In most programs it is probable that the two different program branches starting at $l'$ and $l''$ meet again, that means that for a later program location $l'''$ two abstract states $a, a'$ of the \constraintsCPA\ (in the following  called \emph{constraints states}) exist with $a$ containing $p$ and $a'$ containing $\neg p$.

If a constraint $p$ is part of an abstract state $a$, $p$ is true in all concrete states represented by $a$ (just like a predicate in an abstract state of the \predicateCPA\ \cite{Beyer2008}).
If for one program location $l$ two constraints states $a, a'$ exist with $p \in a$ and $\neg p \in a'$ and $a \setminus \{ p \} = a' \setminus \{ \neg p \}$,
then $a$ represents all concrete states for which $p \logicAnd a \setminus \{ p \}$ is true and $a'$ represents all concrete states for which $\neg p \logicAnd a \setminus \{ p \}$ is true.
At this point, the analysis will never be able to prove a program location as infeasible because of $p$ or $\neg p$.
If $a'$ reaches a program location and computes it as infeasible by using $p$, the abstract state $a$ will compute the same program location as feasible, if it reaches it.
Because of this, it seems legit to delete these obsolete constraints and only continue with one more abstract state instead of two more concrete ones by using the merge operator
\[ \cpaMerge (a, a', \pi) = \begin{dcases}
a' \setminus \neg Q & \text{ if } a \lesserEqual a' \setminus \neg Q \\
a' & \text{ otherwise}
\end{dcases} \]
with $\neg Q = \{\neg p |\ p \in a \logicAnd \neg p \in a' \}$ and $Q = \{ p |\ p \in a \logicAnd \neg p \in a' \}$.
It is not necessary that $a' \setminus \neg Q = a \setminus Q$.
If $a' \setminus \neg Q$ represents a super set of the concrete states represented by $a \setminus Q$, that is $a \setminus Q \lesserEqual a' \setminus \neg Q$, then the above condition is true, and $a \lesserEqual a \setminus Q$.

This condition is automatically checked by the $\mergeAgree$ operator, so we can simply use $\cpaMerge(a, a', \pi) = a' \setminus \neg Q$.

\section{Different Less-or-equal Operators}
%\subsubsection{Subset operator}
\label{sec:leqOperators}
The less-or-equal operator is the operator executed the most often during analyses as $\cpaStop^{sep}$ uses it once for every state in the reached set, at every iteration of the CPA algorithm.
In addition, it is responsible for determining whether a new state is already covered and analysis can be stopped at this point.
Although the implementation framework \cpaChecker\ only performs a termination check for reached states at the same location, 
its speed and precision can make a great difference for the performance of our analysis.

\paragraph*{Aliasing operator}
The  less-or-equal operators we used for \symbolicValueAnalysisCPA\ and \constraintsCPA\ in \cite{Lemberger2015} using an \aliasFunc\ function try to be more precise than a simple subset check.
Unfortunately, they can result in false behaviour because of their independent behaviour.
Consider the two pairs of value state and constraint state
$e = (v, a)$ with $v = \{x \rightarrow s1, y \rightarrow s2\}$, $a = \{s1 > 0\})$ and
$e' = (v', a')$ with $v' = \{x \rightarrow s2, y \rightarrow s1\}$, $a' = \{s1 > 0\})$.
When using the aliasing less-or-equal operators of the \symbolicValueAnalysisCPA\ and of the \constraintsCPA,
the \symbolicValueAnalysisCPA\ states
$v \lesserEqual v'$ for $\aliasFunc$ function $\aliasFunc(s1) = s2$, $\aliasFunc(s2) = s1$ and
the \constraintsCPA\ states
$a \lesserEqual a'$ for $\aliasFunc(s1) = s1$.
Because of this, $e \lesserEqual e'$, although the concrete states
$\llbracket e \rrbracket = \{ c \in C |\ c(x) > 0 \}$ and
$\llbracket e' \rrbracket = \{ c \in C |\ c(y) > 0 \}$
represented by $e$ and $e'$ are two different sets.
This violates the definition of the less-or-equal operator for abstract domains (Section~\ref{sec:abstractState}).
For this example, the less-or-equal operator of the \constraintsCPA\ actually behaves like the subset operator, since $\aliasFunc$ represents the identity.
This shows that the less-or-equal operator of the \symbolicValueAnalysisCPA\ cannot be used, regardless of the operator used by the \constraintsCPA.
Besides the default less-or-equal operator for the \constraintsCPA\ 
presented in Section~\ref{sec:constraintsCPA}, another operator might prove useful.

%\subsubsection{Implication operator}
\paragraph*{Implication operator}
Since a \constraintsCPA 's abstract state $a$ is interpreted as the conjunction of its constraints $\varphi_a$, it seems fit to use implication as the less-or-equal operator.
Remember that $\llbracket a \rrbracket = \{ c \in C |\ c \satisfies \varphi_a \}$.
If a formula $\varphi_a$ implies a formula $\varphi_{a'}$ and $c$ satisfies $\varphi_a$, then $c$ also satisfies $\varphi_{a'}$.
Because of this 
\[\llbracket a \rrbracket = \{ c \in C |\ c \satisfies \varphi_a \} \subseteq \{ c \in C |\ c \satisfies \varphi_{a'} \} = \llbracket a' \rrbracket \text{ if } \varphi_a \Rightarrow \varphi_{a'}.\]
The less-or-equal operator for the \constraintsCPA\ using implication is defined as $a \lesserEqualImpl a'$ if $\varphi_a \Rightarrow \varphi_{a'}$.
This operator has a higher precision than $\lesserEqualSub$ but requires SAT checks, which are definitely worse in performance than merely checking whether one set is the subset of another.


\subsection{Location/Frequency of SAT checks}
\subsubsection{After every assume}
\subsubsection{At every target location only}

\subsection{Basic CEGAR and its algorithm}
\subsubsection{CEGAR and interpolation in general}
Counterexample-guided abstraction refinement (CEGAR) \cite{Clarke2003} is a technique to find an abstraction that contains as few information as possible while retaining the possibility to prove or disprove a program's correctness.
This technique can greatly reduce the number of abstract states in a program's analysis and is considered ''the most general and flexible for handling the state explosion problem,''\cite{Clarke2003}\ the major problem we are facing with our \symbolicExecutionCPA.

The technique starts analysis with a coarse abstraction and refines it based on counterexamples. A counterexample is a witness of a property violation.\cite{Beyer2013}
If no error path is found by the analysis, it terminates and reports that no property violation exists.
If an error path is found, it is checked whether the path is feasible (i.e. a possible program execution) by repeating the analysis with full precision.
If the path is feasible, the analysis terminates and reports the found property violation.
If the error path is infeasible it was only found because the abstraction is too coarse. As a consequence, the abstraction is refined using the error path.
After this, the analysis starts again, using the new abstraction.

Since the problem of finding the coarsest possible refinement of an abstraction based on an error path is NP-hard, \cite{Clarke2003}\ good heuristics have to be used to find good refinements.
Interpolation \cite{Henzinger2004}\ is one such technique in a boolean context that is used for refinement of both the \predicateCPA\ and \valueAnalysisCPA.

\subsubsection{CEGAR and interpolation in the context of configurable software verification}
\label{sec:cegarBasics}
To apply CEGAR and interpolation to configurable software verification, a simple modification has to be made to the CPA algorithm.
Instead of passing it an initial state $e_0$ and an initial precision $\pi_0$, we use an initial reached set $R_0$ and initial waitlist $W_0$ (Alg. \ref{alg:cpaPlus}).
This way we can control at which point the analysis continues after a refinement was performed.

\begin{algorithm}[t]
\caption{$CPA(\cpaPlus, R_0, W_0)$, adapted from \cite{Beyer2013}}
\label{alg:cpaPlus}
\begin{algorithmic}[1]

\Input a CPA $\cpaPlus = (D, \Pi, \transfer, \cpaMerge, \cpaStop, \cpaPrec)$,
	    a set $R_0 \subseteq (E \times \Pi)$ of initial states with their precision and
	    a subset $W_0 \subseteq R_0$ of frontier abstract states with their precision,
	    with $E$ being the set of elements of $D$
\Output a set of abstract states reachable from $R_0$ with their precision and
	   a subset of frontier abstract states with their precision
\Variables \reachedSet\ and \waitlistSet , both subsets of $E \times \Pi$
\State $\reachedSet \assign R_0$
\State $\waitlistSet \assign W_0$
\While{$\waitlistSet \neq \varnothing$} \Comment from here on the same as before
\State ...
\EndWhile
\end{algorithmic}
\end{algorithm}

\begin{algorithm}[t]
\caption{$CEGAR(\cpaPlus, e_0, \pi_0)$, adapted from \cite{Beyer2013}}
\label{alg:cegar}
\begin{algorithmic}[1]
\Input a CPA $\cpaPlus = (D, \Pi, \transfer, \cpaMerge, \cpaStop, \cpaPrec)$ with dynamic precision adjustment,
	an initial abstract state $e_0 \in E$ with precision $\pi_0 \in \Pi$,
	with $E$ denoting the set of elements of the semi-lattice of $D$
\Output the verification result \safe\ or \unsafe
\Variables the sets \reachedSet\ and \waitlistSet\ of elements of $E \times \Pi$,
	      an error path $\sigma = \langle (op_1, l_1), ..., (op_n, l_n) \rangle$\\

\State $\reachedSet \assign \{ (e_0, \pi_0) \}$
\State $\waitlistSet \assign \{ e_0, \pi_0 \}$
\State $\pi \assign \pi_0$
\While{true}
	\State $(\reachedSet, \waitlistSet) \assign CPA(\cpaPlus, \reachedSet, \waitlistSet)$
	\If{$\waitlistSet = \varnothing$}
		\Return \safe
	\Else
		\State $\sigma \assign \extractErrorPath{\reachedSet}$ \label{alg:cegar:extraction}
		\If{\isFeasible{$\sigma$}} \Comment error path feasible: report bug \label{alg:cegar:feasibilityCheck}
			\State % empty state for new line after if
			\Return \unsafe 
		\Else \Comment error path infeasible: refine and restart from the beginning
			\State $\pi \assign \pi \cup \refine{\sigma}$
			\State $\reachedSet \assign (e_0, \pi)$
			\State $\waitlistSet \assign (e_0, \pi)$ \label{alg:cegar:end}
		\EndIf
	\EndIf
\EndWhile
\end{algorithmic}
\end{algorithm}

Now that the CPA algorithm is able to use precisions created in a refinement procedure, we use it as a part of our complete CEGAR algorithm.
Algorithm \ref{alg:cegar} uses a CPA using dynamic precision adjustment $\cpaPlus$,
an initial state $e_0$
and an initial precision $\pi_0$
to compute whether a property violation exists.

First, the $CPA$ algorithm is used to compute a set of reached abstract states ($\reachedSet$) and a subset of this set that contains all reached abstract states that have not been handled yet ($\waitlistSet$).
If $\waitlistSet$ is empty, the $CPA$ algorithm has handled all reachable states without encountering any target state.
If this is the case, no property violation was found and the algorithm can return \safe.
Otherwise, an error path is extracted from the reached set.
If the error path is reported as feasible, a property violation exists or the algorithm is not able to prove that none exists. It returns \unsafe.
If the error path is infeasible, the current precision is too abstract.
It is refined based on the infeasible error path by using $\refineFunc : \Sigma \rightarrow \Pi$ with $\Sigma$ being the set of all error paths, so that it can prove its infeasibility.
After this, the reached set and waitlist are reset to their initial values and the algorithm repeats analysis with the refined precision.
It is important to notice that the return type of $\refineFunc$ has to be equal to the precision type $\Pi$ used in $\cpaPlus$.
Because of this, CPAs are not exchangeable without changing refinement, too, in general.

%%%%%%%%%%%%%%%%%%%%%%%%%%
%%% Refinement in general
%%%%%%%%%%%%%%%%%%%%%%%%%%
For refinement, the priorly mentioned technique of interpolation is used to determine a location-specific precision that is strong enough for the CPA algorithm with precision adjustment to prove that a given error path is infeasible.
A boolean formula $\craigItp$ is a Craig interpolant \cite{Craig1957}\ for two boolean formulas $\prefix$ (called prefix) and $\suffix$ (called suffix), if the following three conditions are fulfilled:
\begin{enumerate}[label=\alph*)]
\item The prefix implies $\craigItp$, that is $\prefix \Rightarrow \craigItp$.
\item $\craigItp$ contradicts the suffix, that means $\craigItp \logicAnd \suffix$ is contradicting.
\item $\craigItp$ only contains variables occurring in \emph{both} prefix and suffix.
\end{enumerate}
It is proven that such an interpolant always exists in the domain of abstract variable assignments \cite{Beyer2013} as well as in the theory of linear arithmetics \cite{Craig1957}.

%\subsubsection{Refinement for the domain of abstract variable assignments}
Our work is strongly based on the refinement technique for abstract variable assignments.
The strongest-post operator $\strongestPostOp_{op}$ describes the semantics of an operation $op \in Ops$.
It is the analogy to the transfer relation in the domain of CPAs.
It maps a region of concrete states, implied by an abstract variable assignment, to the region of all concrete states that can be reached by executing $op$.
The semantics of a path $\sigma = \langle (l_1, op_1), ..., (l_n, op_n) \rangle$ is defined as the consecutive application of the strongest-post operator to its constraint sequence $\gamma_\sigma = \langle op_1, ..., op_n \rangle$:
$\strongestPostOp_{\gamma_\sigma}(v) = \strongestPostOp_{op_n}(\strongestPostOp_{op_{n-1}} (...\ \strongestPostOp_{op_1}(v) ... ))$.
We use strongest-post operators during interpolation and refinement to evaluate paths.

The strongest-post operator $\strongestPostOpExplicit_{op}$ is defined in the following way:
%\begin{enumerate}[label=\alph*)]
%\item
For an assignment operation $s \assign exp$, $\strongestPostOpExplicit_{s \assign exp}(v) = v_\restrictedTo{X \setminus \{ s \}} \logicAnd v_{s \assign exp}$ with $v_{s \assign exp} = \{ (s, exp_\using{v}) \}$ and $exp_\using{v}$ denoting the evaluation of $exp$ using the abstract variable assignment $v$, as defined in Section \ref{sec:valueAnalysis}.
%\item
For an assume operation $\assume(p)$, 
	$\strongestPostOpExplicit_{\assume(p)}(v) = v'$ with 
	\[ v'(x) = \begin{dcases}
		\bot & \text{ if } \exists y \in \defRange(v) : v(y) = \bot \text{ or } p_\using{v} \text{ is unsatisfiable}\\
		c & \text{ if $c$ is the only satisfying assignment of $p_\using{v}$ for $x$}\\
		v(x) & \text{ if none of the above and } x \in \defRange(v)
	\end{dcases}\]
	with $p_\using{v}$ as defined in Section \ref{sec:valueAnalysis}.
%\end{enumerate}

%\subsubsection{Interpolation for abstract variable assignments}
\begin{algorithm}[t]
\caption{$\interpolateExplicit(\prefix, \suffix)$, adapted from \cite{Beyer2013}}
\label{alg:interpolateExplicit}
\begin{algorithmic}[1]
\Input two constraint sequences $\prefix$ and $\suffix$, with $\prefix \logicAnd \suffix$ contradicting
\Output a constraint sequence $\Gamma$, which is an interpolant for $\prefix$ and $\suffix$
\Variables an abstract variable assignment $v$

\State $v \assign \strongestPostOpExplicit_\prefix(\varnothing)$
\ForAll{$x \in \defRange(v)$}
	\If{$\strongestPostOpExplicit_\suffix(v_\restrictedTo{\defRange(v) \setminus \{x\}})$ is contradicting}
		\State $v \assign v_\restrictedTo{\defRange(v) \setminus \{x\}}$ \Comment $x$ not relevant, should not occur in interpolant
	\EndIf
\EndFor
\State $\Gamma \assign \langle \rangle$
\ForAll{$x \in \defRange(v)$} \label{alg:interpolateExplicit:itpStart}
	\State $\Gamma \assign \Gamma \logicAnd \langle \assume(x = v(x))\rangle$
\EndFor\\ \label{alg:interpolateExplicit:itpFinish}
\Return $\Gamma$
\end{algorithmic}
\end{algorithm}

The algorithm for interpolation in the domain of abstract variable assignments is shown in Algorithm \ref{alg:interpolateExplicit}.
For a prefix $\prefix$ and a suffix $\prefix$, the abstract variable assignment $v$, that results from applying $\prefix$ to the initial abstract variable assignment $\varnothing$ is computed.
Next, for each variable assignment in $v$ it is checked whether the assignment is necessary to prove that $\suffix$ is contradicting.
If it is not, it can be removed from $v$.
After all variable assignments are checked, $v$ only contains variable assignments that are necessary to prove that $\suffix$ is contradicting.
From these, the interpolant is built (Lines \ref{alg:interpolateExplicit:itpStart} - \ref{alg:interpolateExplicit:itpFinish}).

\begin{algorithm}[t]
\caption{$\refineExplicit{\sigma}$, adapted from \cite{Beyer2015}}
\label{alg:refinementExplicit}
\begin{algorithmic}[1]
\Input infeasible error path $\sigma = \langle (op_1, l_1), ..., (op_n, l_n) \rangle$
\Output precision $\pi$
\Variables interpolating constraint sequence $\Gamma$
\State $\Gamma \assign \langle \rangle$
\State $\pi(l) \assign \varnothing$ for all program locations $l$
\For{$i \assign 1$ to $n - 1$}\label{alg:refinementExplicit:loopStart}
	\State $\suffix \assign \langle op_{i+1}, ..., op_n \rangle$
	\State $\Gamma \assign \interpolateExplicit(\Gamma \logicAnd \langle op_i \rangle, \suffix)$ \Comment inductive interpolation \label{alg:refinementExplicit:interpolation}
	\State $\pi(l_i) \assign \extractPrecision{\Gamma}$
\EndFor\\
\Return $\pi$
\end{algorithmic}
\end{algorithm}

The interpolants produced are used in the refinement of the precision (Alg. \ref{alg:refinementExplicit}).
We use a location-specific precision $\pi : L \rightarrow 2^X$ that returns for a program location $l \in L$ all program variables of $X$ which are relevant for the analysis at this location. This approach realizes the lazy abstraction technique \cite{Henzinger2002}.
The algorithm starts with an initial, empty interpolant $\Gamma$ and empty precision $\pi$ with $\pi(l) = \varnothing$ for all $l \in L$.
For each location $(l_i, op_i)$ on the error path, the suffix $\suffix$ of this location are set and the interpolant is computed inductively from the existing interpolant in conjunction with the current operation $op_i$ and the suffix (Line \ref{alg:refinementExplicit:interpolation}).
A precision for the current program location is then extracted from the interpolant.
One straightforward way to do this is by using all program variables with a valid assignment in the  abstract variable assignment resulting from the application of the strongest-post operator to our interpolant:
\[\extractPrecision{\Gamma} = \{ x |\ (x, z) \in \strongestPostOpExplicit_\Gamma (\varnothing ) \text{ and } z \neq \bot_\valueset \}.\]
It is not only sufficient, but also required to use $\Gamma \logicAnd \langle op_i \rangle$ instead of the full prefix $\prefix = \langle op_1, ..., op_1 \rangle$ for interpolation. The full prefix cannot be used as it has to be assured that the precision resulting from these consecutive interpolations proves the error path infeasible. All information necessary for proving the infeasibility of the remaining error path is present in the current interpolant and operation.

This refinement procedure can be used in CEGAR (Alg. \ref{alg:cegar}) in combination with a CPA with precision adjustment that expects these precision types, like the \valueAnalysisCPA\ in combination with refinement for abstract variable assignments.

%\subsubsection{Refinement for the domain of linear arithmetics}
Refinement in the domain of linear arithmetics, as used for the \predicateCPA, uses a standard approach to refinement based on lazy abstraction and Craig interpolation.
The task of interpolation is delegated to an off-the-shelf SMT solver.

%\subsubsection{\ValueAnalysisCPA\ with precision adjustment}
%The \valueAnalysisCPA\ with dynamic precision adjustment \cite{Beyer2013} \[\valCPAPlus = (D_\valCPA, \Pi_\valCPAPlus, \transfer_\valCPAPlus, \cpaMerge^{sep}, \cpaStop^{sep}, %\cpaPrec_\valCPAPlus)\] is a CPA that can be, and is, used with the refinement for abstract variable assignments as described above.
%It consists of:
%\begin{enumerate}[leftmargin=*, label=\arabic*.]
%\item The abstract domain $D_\valCPA$ as defined in Section \ref{sec:valueAnalysis}.
%\item The set of precisions $\Pi_\valCPAPlus = L \rightarrow 2^X$. A precision $\pi \in \Pi_\valCPAPlus$ specifies a subset of program variables of $X$ that are tracked.
%\item The transfer relation $\transfer_\valCPAPlus$ contains the transfer $v \transfer_\valCPAPlus (v', \pi)$ if $v \transfer_\valCPA v'$.
%\item The merge operator $\cpaMerge^{sep}$ that performs no merging.
%\item The termination check $\cpaStop^{sep}$ that checks every state individually.
%\item The precision adjustment $\cpaPrec_\valCPAPlus$. Given an abstract state $v$ and a precision $\pi$, all abstract assignments of variables that do not occur in $\pi$ are removed from %$v$. This is done by restricting the partial function: $\cpaPrec_\valCPAPlus(v, \pi) = (v_\restrictedTo{\pi}, \pi)$. The given precision is returned as it is.
%\end{enumerate}

In this chapter, we gave an overview of all theoretical concepts that are necessary to describe our own work. We introduced the concept of configurable software verification and configurable program analyses (CPAs), a very versatile approach to automated software verification. We introduced different CPAs we use in this work and CEGAR with precision refinement for both linear arithmetics and abstract variable assignments, which we will use when applying CEGAR to the \symbolicExecutionCPA.

%\subsection{Bounded loops with continuation after reached bound}
%\subsubsection{Idea}
%\subsubsection{Existing LoopstackCPA}
%\subsubsection{No continuation after reached bound: Existing AssumptionStorageCPA}
%\subsubsection{New: Jump out of loop at every possible exit and abstract information}



\backmatter
\appendix
\bibliography{softwareVerification}

\selectlanguage{ngerman}
\chapter*{Eidesstattliche Erkl\"{a}rung}
\thispagestyle{empty}
Hiermit versichere ich, dass ich diese Bachelorarbeit selbstst\"{a}ndig und ohne
Benutzung anderer als der angegebenen Quellen und Hilfsmittel angefertigt habe
und alle Ausf\"{u}hrungen, die w\"{o}rtlich oder sinngem\"{a}\ss{} \"{u}bernommen wurden, als
solche gekennzeichnet sind, sowie dass ich die Bachelorarbeit in gleicher oder
\"{a}hnlicher Form noch keiner anderen Pr\"{u}fungsbeh\"{o}rde vorgelegt habe.

\bigskip

\noindent Passau, den \today

\bigskip\bigskip\bigskip
\noindent \begin{tabular}{c}
\hspace*{0.5\linewidth}\\
\hline
Thomas Lemberger
\end{tabular}


\end{document}
