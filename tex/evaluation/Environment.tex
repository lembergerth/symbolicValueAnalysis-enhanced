\section{Evaluation setup}
We performed each run of our benchmarks on a dedicated, unloaded server with a Intel Xeon E5-2650 vs2 with 2.60 GHz and four CPU cores, using the Linux operating system Ubuntu 14.04 for the x86\_64 architecture.
The resource limits for each run were 15.00 GB (13.97 GiB) of memory (RAM), a Java heap limit of 10.49 GB (10000 MiB = 9.767 GiB),
a maximum use of two CPU cores,
a CPU time limit of 900 seconds (which equals 15 minutes), after which \cpaChecker\ is supposed to shut down,
and a hard time limit after which the run is killed of 1200 seconds (which equals 20 minutes).
We chose a relatively big difference between the time limit in \cpaChecker\ and the hard time limit to give the analysis enough time to shut down in case of a long taking SAT check during analysis.
As SAT checks are delegated to an SMT solver, \cpaChecker\ is unable to shut down during a check.

We took a subset of the benchmark repository\footnote{\url{https://svn.sosy-lab.org/software/sv-benchmarks/tags/svcomp15}}
of the SV-COMP 2015 verification tasks for our benchmarks.
A detailed explanation of all verification tasks present there can be found at \cite{SV15Benchmark}.
We used:
\begin{enumerate}
\item BitVectors. Requires treatment of bit-operations, which allows us to check the need for and performance of a bitvector-based theory for SMT solving.

\item Floats. Requires handling of floats, which allows us to check the need for and performance of a float-based theory for SMT solving. An alternative is to just use rationals as approximations, whose use increases performance of SMT solving in comparison to floats.

\item ControlFlowInteger, ECA, Loops, ProductLines and Simple.
All five of these sets are designed for testing the control flow and integer variable handling of analyses.
Path explosion should occur here a lot.

\item DeviceDrivers.
This set of tasks consists of problems that require analysis of pointer aliases and function pointers.
Since the \valueAnalysisCPA\ can't handle pointers, we expect the \symbolicExecutionCPA\ to perform poorly, also.
This set uses a simple memory model and the 64-bit architecture, the only task category we use that does so.

\item HeapManipulation. 
This set of tasks also consists of problems that require analysis of pointer aliases and function pointers, as well as data structures on the heap.
In contrast to the set DeviceDrivers, this set uses a precise memory model and a 32-bit architecture.

\item Sequentialized.
Different tasks derived from SystemC programs.
SystemC provides means to simulate concurrent processes.
Such programs were transformed to pure C programs so they can be analyzed by \cpaChecker.

\end{enumerate}
A simple memory model denotes that variables can only be modified using direct assignments or by using a pointer which was obtained by using $\&$ on the corresponding variable.
A precise memory model denotes that all memory cells can be written to, even by dereferencing uninitialized pointers.\cite{CITATION-TO-http://sv-comp.sosy-lab.org/2015/rules.php-NEEDED, OR BETTER, "Software Verification and VErifiable Witnesses, Beyer 2015", if it mentions these two models}
We told \cpaChecker\ whether to assume a 32-bit or 64-bit architecture with the command-line parameters \configOption{-32} (actually the default) and \configOption{-64}.

The external method \objectName{\_\_VERIFIER\_nondet\_X()} is used to introduce non-deterministic values of type \objectName{X} in a program.
Although \cpaChecker\ is able to handle recursive function calls using block-abstraction memoization (BAM) \cite{Friedberger2015},
we do not use this feature, but skip such calls to keep our analysis focused on the performance of the \symbolicExecutionCPA\ and its comparisons.
To skip recursive function calls, we use the command-line parameter \configOption{-skipRecursion}.

All verification tasks are batch executed using a benchmark script that performs all runs with above specifications and which returns for each run one of the following results:
\begin{itemize}
\item \resultTrue, if the program is safe, i.e. the specification holds
\item \resultFalse, if the program is potentially unsafe, i.e. the analysis found a specification violation and can't prove that the specification holds due to this
\item \resultUnknown, if the result is unknown, due to an error, a crash, or exhausted resources (e.g. time or memory)
\end{itemize}
For each such result, the script assigns a number of points depending on the received and the expected result, and presents the sum of these points as a general indicator for the performance of the used program/analyses.
The points assigned are:\\
\ \\
% empty line above necessary so tabular environment is on new line
\begin{tabular}{r | l}
\hline
Points & Result \\ \hline
0 & \resultUnknown \\
+1 & \resultFalse, correct\\
-6 & \resultFalse, incorrect (false alarm)\\
+2 & \resultTrue, correct\\
-12 & \resultTrue, incorrect (unsound analysis) \\\hline
\end{tabular}\\
\ \\
This point scale rewards correct results, but punishes wrong ones stronger, especially unsoundness, the worst property a verifier can have.
The script rewards correctly found bugs with less points then proving that a specification holds, since the latter is more complicated.
In addition, the script does not check whether the error found by the analysis is an actual specification violation or just a lucky coincidence due to a too high level of abstraction, when a correct \resultFalse\ is returned.
That means that simple analyses like the \valueAnalysisCPA, which don't track much information, can get points for finding a potential bug that not really is one in case another real bug exists.
Keeping these points in mind, the resulting score for an analysis can give a fair general overview of its performance.
